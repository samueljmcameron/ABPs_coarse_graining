\documentclass[twocolumn,amsmath,amssymb,aps]{revtex4-1}%{article}
%\usepackage[hmargin=1.25cm,vmargin=2.5cm]{geometry}
\usepackage[utf8]{inputenc}
\usepackage{bm}
%\usepackage{amsmath}
%\usepackage{amssymb}
\usepackage{comment}
\usepackage{graphicx}
\usepackage{dcolumn}
\usepackage[caption=false]{subfig}
\usepackage{braket}
\usepackage{subfiles}

\begin{document}
 
\title{Steady State probability distribution of ABPs}
\author{Sam Cameron}
\affiliation{%
  University of Bristol}%
\date{\today}

\begin{abstract}
  In equilibrium systems, collective properties are calculated
  through knowledge of the Boltzmann probability density,
  $e^{-\beta \mathcal{H}(\{\bm{x}_i)\}}$ and the corresponding partition
  function. In contrast, non-equilibrium/active systems do not obey
  Boltzmann statistics, and so the functional form of the steady-state
  probability density (if it exists) is not known a priori. In this letter,
  we present a systematic way of calculating the steady-state
  probability density of an active systems. As a proof of concept, we use
  this method to calculate the steady-state probability density of
  interacting active brownian particles. Finally, we compare our calculation
  to Molecular Dynamics simulations.
\end{abstract}


\maketitle


\section{Introduction.}
Classical equilibrium statistical mechanics works because of the existence of
a unique probability distribution, which allows scientists to compute
all equilibrium features about a system of particles. For example,
in the canonical ensemble all we need to know about our system is the
form of the interactions between particles (i.e. the Hamiltonian
$\beta U(\{\bm{x}_i\})$) and the temperature $(\beta)^{-1}$ of
the system. With this knowledge, we can immediately write down the probability
distribution of the system as $\rho_{ss}(\{\bm{x}_i\})=\exp(-\beta U)/Z$
(the Boltzmann distribution) and the partition function
\begin{align}
  Z = \int \bigg(\prod_{i=1}^Nd^dx_i\bigg)e^{-\beta U}
\end{align}
where $d$ is the dimension of space. With knowledge of $Z$, we can then
go on to calculate the equation of state for the system, any phase-transitions,
response functions, and other \textit{collective behaviours} of the
system.

Of course, there are numerous practical difficulties in this procedure that
we are omitting, such as how one computes a $dN$ dimensional integral
and whether the Hamiltonian specified really reflects the true nature of
the system one is modelling. But, all of these difficulties can be (at least
partially) overcome via crafty analytical techniques, reasonable
approximations, and numerical methods. The importance of the Boltzmann
distribution, then, is that it gives us a solid foundation, on-top of which we
can build a significantly less sturdy (but good enough) structure.

In contrast, non-equilibrium statistical mechanics does not have a nice
foundation to compute collective behaviour. There is no reason that the
steady-state of a non-equilibrium system should follow a Boltzmann
distribution, and in fact, there is no reason why a steady-state should even
exist!

However, in this letter, we highlight recent work demonstrating the
possibility of calculating steady-state probability distributions for
systems which includes non-conservative forces. This is possible only
in certain non-equilibrim systems, and requires a full dynamical
description (i.e. the stochastic equations of motion). We then showcase
the utility of this procedure by perturbatively computing the
steady-state probability distribution for a system of active brownian
particles (ABPs) with interactions.



\section{Generic steady-state algorithm.}
The procedure only works if we already know the equations of motion.
We restrict our attention to
systems of stochastic differential equations having the form
\footnote{We use summation notation, so all repeated indices are summed
  over.}
\begin{align}\label{eq:stochasticEOMs}
  \dot{\bm{x}}_i = \bm{F}_i(\{\bm{x}_j\})+\sqrt{2}(D^{1/2})\cdot\bm{\xi}_i
\end{align}
$\bm{x}_i$ is the generalised coordinates in phase space for particle
$i$ and $\bm{F}_i$ is the force that is exerted on this particle.
The set of vectors $\{\bm{\xi}_i\}$ generate additive Gaussian white
noise with zero mean and variance $D_{\alpha\beta}$, i.e.
\footnote{The ensemble average of the system of an observable
  $\mathcal{O}(\{\bm{\xi}_i\})$ is the average over the noise,
  \unexpanded{
  \begin{align}
    <\mathcal{O}>
    = \int \mathcal{D}[\{\bm{\xi}_i\}]\mathcal{O}
    \exp\bigg(-\frac{1}{T}\int_0^T\sum_{j=1}^N\bm{\xi}_j(t)^2 dt'\bigg).
  \end{align}
  }
}
\begin{align}
  <\bm{\xi}_i>=0,\:\:<(\bm{\xi}_i)_{\alpha}(t)(\bm{\xi}_j)_{\beta}(t')>
  =\delta_{ij}\delta_{\alpha\beta}\delta(t-t'),
\end{align}
and $D_{\alpha,\beta}$ is a multiplicative tensor (typically a diffusion
constant which is set as a parameter) which is for our purposes independent
of position and time, and also diagonal.

A conservative force $\bm{F}_{i,c}$ is one which can be written as the gradient
of a scalar. We can decompose the total force then into two vectors,
\begin{align}
  \bm{F}_{i}
  & = \bm{F}_{i,c}+\bm{F}_{i,a} = -\nabla_{i}U(\{\bm{x}_j\}) + \bm{F}_{i,a}
\end{align}
where $\bm{F}_{i,a}$ cannot be written as the gradient of a scalar.

The above equations of motion can be re-cast using standard techniques into
the Fokker-Planck equation
\begin{align}
  \frac{\partial P(\bm{x})}{\partial t} + \nabla\cdot\bm{J}(\bm{x},t)
  = 0
\end{align}
where $\bm{x}$ is an $nD$ dimensional vector, $P(\bm{x},t)$ is the
probability of finding the set of $N$ particles in configuration $\bm{x}$
at time $t$, and the current is given by
\begin{align}
  J_{\alpha}(\bm{x},t)
  = \bigg[D_{\alpha\beta}F_{\beta}(\bm{x})
    -\sqrt{D_{\alpha\beta}D_{\gamma\beta}}
    \partial_{\gamma}\bigg]P(\bm{x},t).
\end{align}
As before, repeated indices imply summation.
Note that now $\alpha=1,\cdots,dN$ and $\beta=1,\cdots,dN$ (as opposed to
in eqn \ref{eq:stochasticEOMs} where $\alpha=1,\cdots,d$).

We are interested in steady-states, so that $\partial P/\partial t=0$.
If we write the steady-state as $\rho_{ss}(\bm{x})\propto \exp(-h(\bm{x}))$,
we can determine $h$ through the set of equations
\begin{align}\label{eq:generic_ss}
  0
  =&D_{\alpha\beta}\big[\partial_{\alpha}F_{\beta}
    -F_{\beta}\partial_{\alpha}h\big]\nonumber\\
  &+\sqrt{D_{\alpha\beta}D_{\gamma\beta}}
    \big[\partial_{\alpha}h\partial_{\gamma}h
      -\partial_{\alpha}\partial_{\gamma}h\big].
\end{align}
Notably, if the force $\bm{F}$ is conservative so that
$\bm{F}=-\nabla U(\bm{x})$, and the tensor $D_{\alpha\beta}$ is symmetric,
then $h=U(\bm{x})$ and we recover the Boltzmann steady-state distribution.

However, the general case results in a non-linear PDE for $h$,
and so finding a closed-form solution seems unlikely at best. But, if
we assume that the non-conservative force is small, we should be able
to find a perturbative solution for $h$. That is what we now present for
the case of ABPs.

\subsection{Perturbative calculation of ABP steady state.}
The equations of motion for a $d=2$ system of ABPs (neglecting hydrodynamics)
in contact with a thermal bath are
\begin{subequations}
  \label{eqs:ABPsEOMs}
  \begin{align}
    \dot{\bm{r}}_i
    &= -\nabla_i U+\mathrm{Pe}\:\bm{u}_i + \sqrt{2}\bm{\xi}_{i}^r,
    \label{eq:pos_EOM}\\
    \dot{\theta}_i &= \sqrt{4\Pi}\xi_i^{\theta}\label{eq:theta_EOM},
  \end{align}
\end{subequations}
where $\Pi=\sigma^2D^r/(2D^t)$ and $\mathrm{Pe} = v\beta\sigma$ are the
(dimensionless) rotational diffusion constant and Peclet number,
respectively. We measure length in units of a microscopic length
scale specified by $ U$, and time in units of $\sigma^2/D^t$.
$\bm{u}_i = (\cos\theta_i,\sin\theta_i)$ is the direction
of self-propulsion for particle $i$. We assume that the conservative
forces can be written as a sum over
pair potentials which only depend on the distance between particles,
\begin{align}\label{eq:HtoV}
   U(\bm{x}) = \sum_{i=1}^N\sum_{j=1}^{i-1}\mathcal{V}(r_{ij}).
\end{align}

We re-write eqn \ref{eq:generic_ss} as
\begin{align}\label{eq:perturbODE}
  \sum_{i=1}^N[\mathcal{L}_i^{(0)}(h)+\mathrm{Pe}\:\mathcal{L}_i^{(1)}(h)]
  = 0
\end{align}
where
\begin{align}\label{eq:perturb0th}
  \mathcal{L}_i^{(0)}(h)=&(\nabla_i h)^2 + \nabla_i^2 U
  -\nabla_i U\cdot\nabla_ih-\nabla_i^2h\nonumber\\
  &+2\Pi\big[(\partial_{\theta_i}h)^2-\partial_{\theta_i}^2h\big],
\end{align}
and
\begin{align}\label{eq:perturb1st}
  \mathcal{L}_i^{(1)}(h)=\bm{u}_i\cdot\nabla_ih.
\end{align}
Up until eqn \ref{eq:perturb1st}, our result is still exact. We now
introduce approximations in order to make the problem tractable.

Defining $\bm{r}_{ij}=\bm{r}_j-\bm{r}_i$, $\bm{u}_{ij}=\bm{u}_j-\bm{u}_i$,
we expand
\begin{align}\label{eq:hgeneric}
  h= U+\mathrm{Pe}\:\sum_{i=1}^N\sum_{j=1}^{i-1}w(r_{ij})
  \bm{r}_{ij}\cdot\bm{u}_{ij}
\end{align}
as the lowest order rotationally invariant form, assuming terms with
$\bm{u}_{ij}\cdot\bm{u}_{ij}$ are negligible.

Inserting this into eqn \ref{eq:perturbODE}, we arrive at the equation
\begin{widetext}
\begin{align}\label{eq:bigeq}
  0=&\sum_{i=1}^N\sum_{j\neq i}^N\bm{r}_{ij}\cdot\bm{u}_{ij}
  \bigg\{\sum_{k\neq i}^N\bigg[
    \frac{\mathcal{V}^{\prime}(r_{ik})}{r_{ik}}
    w'(r_{ij})\frac{\bm{r}_{ik}\cdot\bm{r}_{ij}}{r_{ij}}\bigg]
  -w''(r_{ij})-3\frac{w'(r_{ij})}{r_{ij}}
  +\Pi w(r_{ij})
  +\frac{\mathcal{V}^{\prime}(r_{ij})}{2r_{ij}}\bigg\}\nonumber\\
  &+\sum_{i=1}^N\sum_{j\neq i}^N\bm{r}_{ij}\cdot\sum_{k\neq i}^N
  \bm{u}_{ik}\frac{\mathcal{V}^{\prime}(r_{ik})}{r_{ik}}w(r_{ik}).
\end{align}
\end{widetext}

To simplify this further, we note that
any terms  involving $\sum_{i\neq k}f(\bm{r}_{ik})\bm{r}_{ik}$, with no
coupling to orientation $\bm{u}_{ij}$, can be written as
\begin{align}\label{eq:g2_hat}
  \sum_{i\neq k}f(\bm{r}_{ik})\bm{r}_{ik}
  \propto\int\hat{g}_2(\bm{r})f(-\bm{r})\bm{r}d^dr,
\end{align}
where $\hat{g}_2(\bm{r})=\sum_{k\neq i}\delta(\bm{r}-\bm{r}_{ik})$ is the
instantaneous value of the radial distribution function. On average, we
expect this to be isotropic, i.e. $\braket{\hat{g}_2(\bm{r})}=g_2(r)$.
Then, any terms in the form of eqn \ref{eq:g2_hat} should, on average, be
zero due to radial symmetry \footnote{Strictly speaking, this implies
  that we are averaging over some fluctuations in the system, and this
  assumption could be problematic. However, we expect that any error
  should vanish as the number of particles $N\to\infty$.}.
With these assumptions, we get a second order linear ODE,
\begin{align}\label{eq:w_ODE}
  w''(r)+\frac{3w'(r)}{r}-\Pi w(r)=\frac{\mathcal{V}'(r)}{2r}=0.
\end{align}

In appendix \ref{app:solve_w_ODE}, we suggest that the boundary conditions
which accompany eqn \ref{eq:w_ODE} are
that $w(r)$ vanishes at large $r$, so that the boundary conditions
\begin{subequations}
  \label{eqs:w_BCs}
  \begin{align}
    w'(r_0)r_0+w(r_0)+\frac{1}{3} = 0,\label{eq:w_BC_r0}\\
    w(\infty) = 0,\label{eq:w_BC_inf}.
  \end{align}
\end{subequations}
The former arises from a force balance of three
particles converging on each other and forming a (temporarily) stable
cluster at separation $r_0$. The latter is assuming no long-range effects
are present.

In order to solve eqns \ref{eq:w_ODE} and \ref{eqs:w_BCs}, a form of
$\mathcal{V}$ must be selected. In appendix \ref{app:solve_w_ODE},
we determine $w(r)$ for both the hard sphere potential and the repulsive
Weeks-Chandler-Anderson (WCA) potential. In Figure \ref{fig:1} we
show the functional form of $w(r)$ for the WCA potential at different
$\mathrm{Pe}$ values, with $\Pi=3/2$ fixed.

\begin{figure}[t!]
  \includegraphics[width=0.5\textwidth]{Figures/2020_05_15_soft_sphere.pdf}
  \caption{Functional form of $w(r)$ with $\Pi=3/2$ held constant and
    a repulsive $\mathcal{V}$. The value of $r_0$,
    when an isolated trimer experiences zero force, is shown in black dots
    for each Peclet number.
  }\label{fig:1}
\end{figure}

\section{Statistical Mechanics of ABPs.}
If we assume only that the steady state of the ABPs takes the form
$\frac{1}{\tilde{Z}}\exp(-h)$, where
$h(\{\bm{r}_i\},\{\bm{\theta}_i\})$ is given by eqn \ref{eq:hgeneric},
we can write, in analogy to
equilibrium thermodynamics, a ``partition function'' for the position and
orientation degrees of freedom, $\tilde{Z}$. Just as in equilibrium
thermodynamics,
the partition function is just a normalising factor
$\tilde{Z}$, defined as
\begin{align}\label{eq:Zori}
  \tilde{Z} =& \int\prod_{i=1}^N(d^2r_id\theta_i)\nonumber\\
  &\times\exp\bigg(-\sum_{i,j<i}[\mathcal{V}(r_{ij})
    +\mathrm{Pe}\:w(r_{ij})\bm{r}_{ij}\cdot\bm{u}_{ij}]\bigg).
\end{align}
We have shown in appendix \ref{app:Z}
that it is possible to integrate out all orientational degrees
of freedom in $\tilde{Z}$, such that
\begin{align}\label{eq:Z}
  Z = \int\prod_{i=1}^N(d^2r_i)\exp[-V_{\mathrm{eff}}(\{\bm{r}_i\})],
\end{align}
where $V_{\mathrm{eff}}$ is a new, effective potential for this ABP
steady-state, which only depends on the positions of the particles.
We can then write our steady-state probability density as
$\rho_{ss}=1/Ze^{-V_{\mathrm{eff}}}$. Furthermore, this effective
potential can be expanded in powers of $\mathrm{Pe}^2$,
and to $O(\mathrm{Pe}^4)$ (see \ref{app:Z}),
\begin{align}
  V_{\mathrm{eff}}(\{\bm{r}_i\};\mathrm{Pe})
  &\approx\sum_{j<i}W_2(r_{ij})
  +\sum_{k<j<i}W_3(r_{ij},r_{ik},\theta)
\end{align}
where
\begin{subequations}
  \label{eqs:2and3body}
  \begin{align}
    W_2(r;\mathrm{Pe})
    &= \mathcal{V}(r)
      -\frac{1}{2}\mathrm{Pe}^2 r^2w^2(r),\label{eq:2body}\\
    W_3(r_1,r_2,\theta;\mathrm{Pe})
    &= -\frac{3}{2}\mathrm{Pe}^2w(r_1)w(r_2)r_1r_2\cos\theta
      \label{eq:3body}
  \end{align}
\end{subequations}
are the effective two-body and three-body interactions for the ABP
steady state. The presence of three-body interactions at $O(\mathrm{Pe}^2)$
is a consequence of eqn \ref{eq:hgeneric}, regardless of the functional
form of $w(r)$. This highlights that in
general, we expect that many-body effects will play a role in the
steady-state behaviour of ABPs. We investigate this further in section
\ref{subsec:collbeh} below.

We have calculated these effective interactions,
taking the true two-body potential $\mathcal{V}$
in eqn \ref{eq:HtoV} to be WCA potential.
The resulting forms of $W_2(r)$ and $W_3(r_{12},r_{13},\cos\theta)$ are
presented in Figures
\ref{fig:2} and \ref{fig:3}, respectively. Notably,
a minimum develops in the effective two-body interaction, $W_2$,
\textit{even though} the original WCA potential is strictly repulsive.
Extrapolating
results from liquid-state theory, we expect that presence of this
effective attractive interaction potential will allow the ABPs to phase
separate into a low-density and a high-density phase. This phase
separation has been verified to occur in simulation, and is known as
Motility Induced Phase Separation (MIPS).

\begin{figure}[t!]
  \includegraphics[width=0.5\textwidth]{Figures/2020_05_15_W_2.pdf}
  \caption{Functional form of the effective potential two-body interaction,
    $W_2(r)$, with $\Pi=3/2$ held constant and a repulsive $\mathcal{V}$.
    The value of $r_0$,
    when an isolated trimer experiences zero force, is shown in black dots
    for both Peclet numbers. The inset highlights the existence of a shallow
    minimum for $\mathrm{Pe}=0.1$.
  }\label{fig:2}
\end{figure}

\begin{figure}[t!]
  \includegraphics[width=0.5\textwidth]{{Figures/2020_05_15_W_3_Pe_10.0}.pdf}
  \caption{Functional form of the effective potential three-body
    interaction, $W_3(r)$, with $\Pi=3/2$, $\mathrm{Pe}=10$ held constant
    and a repulsive $\mathcal{V}$. $\bm{r}_{12}$ is the distance between
    the first and second particle in the three-body interaction, and
    $\bm{r}_{13}$ is the distance between the first and third particle.
    $\bm{r}_{13}\cdot\bm{r}_{12}=r_{12}r_{13}\cos\theta$, and we have
    chosen $\theta=\pi/3$. Therefore, when $r_{12}=r_{13}$ in this
    figure, the three
    particles are in an equilateral triangle configuration.
  }\label{fig:3}
\end{figure}

\subsection{Collective behaviour of ABPs using $\rho_{ss}=e^{-h}/\tilde{Z}$.
  \label{subsec:collbeh}}

In this section, we make predictions about collective behaviour of ABPs
using eqns \ref{eq:Zori}, \ref{eq:Z}, \ref{eq:2body}, and,
when necessary, \ref{eq:3body}.
As in the case of equilibrium statistical mechanics, we generate various
ensemble averages through derivatives of $\ln \tilde{Z}$. We compare our
predictions with simulations of eqn \ref{eqs:ABPsEOMs}, with $N=10000$
ABPs and periodic boundary conditions.

To begin with, we consider the analog of the equilibrium pressure for the
ABP system,
\begin{align}
  P_{ss} = \frac{\partial \ln Z}{\partial A}
\end{align}
where $\rho\equiv N/A$ is the number density of ABPs.
We label this pressure as $P_{ss}$ to differentiate from the pressure we
measure in simulation, which we will denote $P$
\footnote{Computation of the simulation pressure $P$ is discussed in
  appendix \ref{app:pressure}}.

If we keep terms to $O(\mathrm{Pe}^2)$, this pressure
can equivalently be expressed as
$P_{ss}=P_{\mathrm{id}}+P_{ss}^{(2)}+P_{ss}^{(3)}$, where
\begin{subequations}
  \label{eqs:P2andP3}
  \begin{align}
    P_{ss}^{(2)}
    &\equiv-\frac{\pi N^2}{2A^2}\int du W_2'(u)u^2 g^{(2)}(u),
    \label{eq:P2}\\
    P_{ss}^{(3)}
    &\equiv-\frac{2\pi N^3}{3A^3}\int dudvd\theta u^2v
    \frac{\partial W_3(u,v,\theta)}{\partial u}
    g^{(3)}(u,v,\theta),\label{eq:P3}
  \end{align}
\end{subequations}
are the two-body and three-body contributions to the steady-state pressure
at $O(\mathrm{Pe}^2)$, and $P_{\mathrm{id}}=\rho$ is the ideal gas pressure.
The main assumptions in deriving eqn \ref{eqs:P2andP3} are that the two-body
and three-body correlation functions, $g^{(2)}(u)$ and
$g^{(3)}(u,v,\theta)$, respectively, are invariant
under rotations, translations, and permutation of particle positions
\footnote{see appendix \ref{app:press} for details}. We use our simulations
to compute both $g^{(2)}(u)$ and $g^{(3)}(u,v,\theta)$. We justify
this by noting that we are investigating whether our steady-state gives
a reasonable approximation to the measured simulation pressure, and so
any further approximations employed to compute $g^{(i)}$ would be a
distraction \footnote{Though they can in principle be calculated,
  e.g. via a virial expansion.}.


\begin{figure}[t!]
  \includegraphics[width=0.5\textwidth]{{Figures/2020_05_19_threebody_correction_P_vs_fp_rho_is_0.05_and_0.5}.pdf}
  \caption{Steady-pressure of the ABP system for $\rho=0.05$ (top) and
    $\rho=0.5$ (bottom). Orange circles denote the simulation pressure
    as measured in LAMMPS, $P_L$, less the ideal gas term
    $P_{\mathrm{id}}=\rho$. Blue arrows denote the same simulation, but
    subtracting out both $P_{\mathrm{id}}$ and the ideal swim pressure,
    defined as $P_{\mathrm{swim}}=\rho\mathrm{Pe}^2/(4\Pi)$. The green
    squares denote the pressure calculated using only the two-body
    distribution function, $P_{ss}^{(2)}$ (eqn \ref{eq:P2} in the text),
    whereas the red triangles denote the pressure calculated using both
    $P_{ss}^{(2)}$ and the three-body distribution function, $P_{ss}^{(3)}$
    (eqn \ref{eq:P3} in the text).
  }\label{fig:4}
\end{figure}

To begin with, we note that the ABP pressure for a $d$ dimensional
system can be written as
\begin{align}
  dVp^i=&
  dN + \frac{1}{2}\sum_{i,j\neq i}\sum_{\bm{n}}
  \braket{\bm{F}_{ij}^{\bm{n}}\cdot(\bm{r}_i-\bm{r}_j-\bm{R}_{\bm{n}})}
  \nonumber\\
  &+\mathrm{Pe}\:\sum_{i=1}^N\braket{\bm{u}_i\cdot\bm{r}_i},
\end{align}
where $\braket{\cdots}$ denotes the steady-state ensemble average,
which we are taking as $\rho_{ss}=\exp(-h)/Z$.




\bibliographystyle{plain}
\bibliography{steadystate}
\end{document}
