\documentclass[twocolumn,amsmath,amssymb,aps]{revtex4-1}%{article}
%\usepackage[hmargin=1.25cm,vmargin=2.5cm]{geometry}
\usepackage[utf8]{inputenc}
\usepackage{bm}
%\usepackage{amsmath}
%\usepackage{amssymb}
\usepackage{comment}
\usepackage{graphicx}
\usepackage{dcolumn}
\usepackage[caption=false]{subfig}

\begin{document}
 
\title{calculation: coarse-graining active brownian particles}
\author{Sam Cameron}
\affiliation{%
  University of Bristol}%
\date{\today}

\begin{abstract}
\end{abstract}

\maketitle

\section{Estimation of standard error in correlated data.}

This section follows ref. \cite{doi:10.1063/1.457480}.

To calculate a time-averaged quantity $<x>$ from an MD simulation, one
must first take instantaneous measurements $x_i$ (here $i$ indicates the
time at which the measurement was taken, and then computes the average in
the standard way,
\begin{align}\label{eq:xav}
  <x>\approx m \equiv \frac{1}{n}\sum_{i=1}^nx_i
\end{align}
where $m$ is the estimate of $<x>$ given the data.

An estimator of the variance of $m$ is
\begin{align}\label{eq:sigma_est}
  \sigma^2(m) = <m^2>-<m>^2,
\end{align}
and by inserting eqn \ref{eq:xav} and defining
$\gamma_{i,j}=<x_ix_j>-<x_i><x_j>$, this becomes
\begin{align}
  \sigma^2(m)
  &= \frac{1}{n^2}\sum_{i=1}^n\sum_{j=1}^n(<x_ix_j>-<x_i><x_j>)
  \nonumber\\
  &= \frac{1}{n^2}\sum_{i=1}^n\sum_{j=1}^n\gamma_{i,j}\nonumber\\
  &=\frac{1}{n^2}\bigg(\sum_{i=1}^n\sum_{j\neq i}^n\gamma_{i,j}
  + \sum_{i=1}^n\gamma_{i,i}\bigg)\nonumber\\
  &=\frac{1}{n^2}\bigg(2\sum_{i=1}^n\sum_{j=1}^{i-1}\gamma_{i,j}
  + \sum_{i=1}^n\gamma_{i,i}\bigg)\nonumber\\
  &=\frac{1}{n}\bigg[\gamma_0
    +2\sum_{t=1}^{n-1}\bigg(1-\frac{t}{n}\bigg)\gamma_t\bigg]
\end{align}
where in the last line we have used $\gamma_{i,j}=\gamma_t$ where $t=|i-j|$.

It is suggested in the paper that the naive measure of $\gamma_t$ is
\begin{align}
  c_t = \frac{1}{n-t}\sum_{k=1}^{n-t}(x_{k+1}-m)(x_k-m).
\end{align} 
However, this is not a biased estimator of $\gamma_t$, as can be seen
by formally taking the ensemble average of $c_t$, and recalling eqn
\ref{eq:xav},
\begin{widetext}
\begin{align}
  <c_t>
  =& \frac{1}{n-t}\sum_{k=1}^{n-t}<(x_{k+t}-m)(x_k-m)>\nonumber\\
  =& \frac{1}{n-t}\sum_{k=1}^{n-t}(<x_{k+t}x_k>-<mx_k>
  -<mx_{k+t}>+<m^2>)\nonumber\\
  =& \frac{1}{n-t}\sum_{k=1}^{n-t}(<x_{k+t}x_k>
  +\sigma^2(m)+<m>^2-<m>^2+<m>^2
  -\frac{1}{n}\sum_{i=1}^n<x_ix_k>
  -\frac{1}{n}\sum_{i=1}^n<x_ix_{k+t}>)\nonumber\\
  =&\gamma_t+\sigma^2(m)+2\sigma^2(m)-2\sigma^2(m)
  -\frac{1}{n-t}\sum_{k=1}^{n-t}
  \frac{1}{n}\sum_{i=1}^n(\gamma_{i,k}+\gamma_{i,k+t})
  \nonumber\\
  =&\gamma_t-\sigma^2(m)+2\frac{1}{n^2}\sum_{i=1}^n\sum_{j=1}^n\gamma_{i,j}
  -\sum_{i=1}^n\sum_{k=1}^{n-t}
  \frac{1}{n(n-t)}(\gamma_{i,k}+\gamma_{i,k+t})\nonumber\\
  =&\gamma_t-\sigma^2(m)+\Delta_t
\end{align}
where
\begin{align}
  \Delta_t
  &=\sum_{i=1}^n\bigg[\frac{2}{n^2}\sum_{j=1}^n\gamma_{i,j}
  -\frac{1}{n(n-t)}\sum_{k=1}^{n-t}(\gamma_{i,k}+\gamma_{i,k+t})\bigg].
\end{align}
\end{widetext}
Note that this is NOT equal to equation 10 of ref.\cite{doi:10.1063/1.457480}.
It would only be equivalent for
\begin{align}
  \sum_{i=1}^n\sum_{k=1}^{n-t}(\gamma_{i,k}+\gamma_{i,k+t})
  =2\sum_{i=1}^n\sum_{k=1}^{n-t}\gamma_{i,k}
\end{align}
which I cannot see as being true. I'm not sure where my error is.
\section{Microscopic definition of pressure.}
\subsection{System with walls.}

Following appendix C of \cite{C5SM01412C}.

Define the pressure, $p^{e,\alpha}$ on a surface of area
$\Delta S^{\alpha}$ as
\begin{align}\label{eq:pfund}
  p^{e,\alpha}\Delta S^{\alpha}
  \equiv -\sum_{i\in\Delta S^{\alpha}}
  <\bm{F}_i^{s,{\alpha}}\cdot\bm{n}_i^{\alpha}>.
\end{align}
$\bm{F}_i^{s,{\alpha}}$ is the force exerted on the surface
(e.g. through some potential energy of the surface) due to
the $i^{th}$ particle, and $\bm{n}_i^{\alpha}$ is the normal
to the surface at the location of the $i^{th}$ particle. The
centre of $\Delta S^{\alpha}$ is at $\bm{r}^{\alpha}$, and
the normal at $\bm{r}^{\alpha}$ is $\bm{n}^{\alpha}$.

Assume $\Delta S^{\alpha}$ is smooth (curvature variations
are small compared to some particle length scale $\sigma$).
More concretely, this is saying that the change in curvature
of the wall at any two points within $\Delta S^{\alpha}$ is
small, $\sigma d\bm{n}/dr^{\alpha}|_{\Delta S^{\alpha}}\ll 1$.
This should be valid for small enough $\Delta S^{\alpha}$.
Then, we can say that
$\bm{n}_i^{\alpha}\approx\bm{n}^{\alpha}$. We can now re-write
eqn \ref{eq:pfund} as
\begin{align}
  p^{e,\alpha}\Delta S^{\alpha}\bm{n}^{\alpha}
  \cdot\bm{r}^{\alpha}
  \approx -\sum_{i\in\Delta S^{\alpha}}<\bm{F}_i^{s,\alpha}>
\end{align}
where we have multiplied both sides by the dot product
$\bm{r}^{\alpha}\cdot\bm{n}^{\alpha}$.

Next, define $d\bm{S}\equiv\Delta S^{\alpha}\bm{n}^{\alpha}$,
and use the assumption that for small enough $d\bm{S}$,
$\bm{r}^{\alpha}\approx\bm{r}_i^{\alpha}$ so that the sum over
all $\alpha$ is
\begin{align}
  \sum_{\alpha}p^{e,\alpha}\Delta S\bm{n}^{\alpha}
  \cdot\bm{r}^{\alpha}
  \approx -\sum_{\alpha}\sum_{i\in\Delta S^{\alpha}}
  <\bm{F}_i^{s,\alpha}\cdot\bm{r}_i^{\alpha}>
  \approx-\sum_{i=1}^{N}<\bm{F}_i^{s}\bm{r}_i>.
\end{align}
In the last line, I have assumed that each particle only
exerts forces on one surface element. This is not the case
at e.g. corners, but errors introduced from this assumption
should be negligible in the limit of large particle number,
as only a small number of particles can be within a close
enough distance to a wall corner to feel multiple surface
elements. In the limit of $\Delta S\to 0$, the sum over
$\alpha$ becomes an integral, and the approximations above
become exact (aside from the corner discussion).
replacing the discrete $\bm{r}^{\alpha}$ to a continuous
vector $\bm{r}$, and using the divergence theorem,
\begin{align}
  \int_{S} p^e(\bm{r})\bm{r}\cdot d\bm{S}
  =\int_V\nabla\cdot(p^e(\bm{r})\bm{r})d^dr
  =-\sum_{i=1}^{N}<\bm{F}_i^{s}\bm{r}_i>
\end{align}
where $d$ is the volume element in $d$ dimensions.

Finally, if $p^e$ is independent of position, then in $d$
dimensions the pressure on a wall is
\begin{align}\label{eq:pwall}
  p^e
  =-\frac{1}{dV}\sum_{i=1}^{N}<\bm{F}_i^{s}\bm{r}_i>,
\end{align}
and so the pressure on the wall is the virial over all
surface forces.

\subsection{Periodic boundary conditions.}

\section{Standard method of pressure computation in LAMMPS.}

In LAMMPS, the pressure is computed by first assuming the formula
\begin{align}
  P_{\alpha\beta}V = \bigg<\sum_{i=1}^N(m_i(\bm{v}_i)_{\alpha}
  (\bm{v}_i)_{\beta}+W_{\alpha\beta}(\{\bm{r}_i\}))\bigg>
\end{align}
is correct, where
\begin{align}
  W_{\alpha\beta}(\{\bm{r}_i\})\equiv\sum_{k\in\bm{0}}
  \sum_{w=1}^{N_k}(\bm{r}_w^k)_{\alpha}
    (\bm{F}_w^k)_{\beta}
\end{align}
is the virial tensor, where the system has been broken into
$k$ groups of atoms which interact with each other (i.e.
are within cutoff distance of each other, and so an atom can
belong to multiple groups), and each group has $N_k$ atoms in
it. So $\bm{r}_w^k$ is the position of the $w^{th}$ atom in
group $k$, $\bm{F}_w^k$ is the force due to all atoms in group
$k$ on the $w^{th}$ atom in group $k$. The first summation,
with $k\in\bm{0}$, just states that the sum is over all group
images associated with the local lattice cell, and so it is
taken that $\bm{r}_1^k=\bm{r}_{i\bm{0}}$ is always in the local
cell. This is actually equivalent to the second in term in ref
Gompperant so can remain in the system.

Just need to figure out the term with summation over
$\sum_{i}\sum_{j}\sum_{\bm{n}}\bm{F}_{ij\bm{n}}\cdot\bm{e}_i$,
probably using same style as Thomson calculation for lammps.
(look at lammps code for insight??).

of interacting atoms having
  . Unfortunately, it is not. The first term can be easily
neglected by telling lammps not to compute it, but the second
term might still be some use if we can understand where it comes
from.

\section{Writing the active brownian particles equations of motion into
  the correct form}

The equations of motion for active brownian particles are:
\begin{subequations}
  \label{eqs:basicABPsODEs_dimensional}
  \begin{align}
    \frac{d\tilde{\bm{r}}_i}{d\tilde{t}}&=\tilde{\beta}
    \tilde{D}^t\big(\tilde{\bm{F}}_i
    +\tilde{f}^P\bm{P}_i\big)
    +\sqrt{2\tilde{D}^t}\tilde{\bm{\xi}}^r_i,\\
    \frac{d\theta_i}{dt}&=\sqrt{2\tilde{D}^r}
    \tilde{\bm{\xi}}^{\theta}_i.
  \end{align}
\end{subequations}
where $\tilde{\bm{r}}_i=(x_i,y_i)$ are the particle positions,
$\bm{P}_i=(\cos\theta_i,\sin\theta_i)$ are the particle orientations,
and $\tilde{\bm{\xi}}^{a}_i$, $a=t,r$ are Gaussian white noise
with zero mean and unit variance, i.e.
\begin{subequations}
  \begin{align}
    \big<(\tilde{\xi}^a_i(t))_{\gamma}\big>&=0,\\
    \big<(\tilde{\xi}^a_i(t))_{\gamma}(\tilde{\xi}^b_j(t'))_{\beta}\big>
    &=\delta_{\beta\gamma}\delta_{ij}\delta_{ab}\delta(t-t^{\prime}).
  \end{align}
\end{subequations}

If I take $\sigma$ (the length scale of the force $\bm{F}_i$) as the unit of
length, $\beta$ as the unit of energy, and $\tau = \sigma^2/D^t$ as the unit
of time, then eqn \ref{eqs:basicABPsODEs_dimensional} can be re-written as
\begin{subequations}
  \label{eqs:basicABPsODEs}
  \begin{align}
    \dot{\bm{r}}_i&=\bm{F}_i
    +f^P\bm{P}_i
    +\sqrt{2}\bm{\xi}^{r}_i,\label{eq:micro_pos}\\
    \dot\theta_i&=\sqrt{4\Pi}\xi^{\theta}_i,\label{eq:micro_theta}
  \end{align}
\end{subequations}
where dots over variables are derivatives with respect to (dimensionless)
time $t$ and all variables without tildes are dimensionless, i.e.
\begin{subequations}
  \label{eqs:dimensionless}
  \begin{align}
    \tilde{t} &= t\frac{\sigma^2}{D^t},\\
    \tilde{\bm{r}}_i &= \bm{r}_i\sigma,\\
    \tilde{\bm{F}}_i & = \bm{F}_i\frac{1}{\beta\sigma},\\
    \tilde{f}^P &= f^P\frac{1}{\beta\sigma},\\
    \tilde{D}^r &= \Pi \frac{2D^t}{\sigma^2}.
  \end{align}
\end{subequations}
\subsection{Ensemble averaging.}
We define the ensemble average of some observable $\mathcal{O}(\bm{\xi}(t))$
as
\begin{align}
  <\mathcal{O}>=\frac{1}{Z}\int \mathcal{D}\bm{\xi}
  \mathcal{O}(\bm{\xi}(t))
  \exp\bigg(-\frac{1}{2}\int_0^Tdt(\bm{\xi}(t))^2\bigg),
\end{align}
where the components of $\bm{\xi}$ include all particles, with each particle
having $3$ ($x$, $y$, and $\theta$) noise terms (and so $3N$ terms total).

To begin with, we derive the ensemble averaged orientation of eqn
\ref{eq:micro_theta}. Formally,
\begin{align}
  \theta_i(t)-\theta(0)=\sqrt{4\Pi}\int_0^tdt_1\xi_i^{\theta}(t_1).
\end{align}
Then, the mean-squared angular fluctuations between two particles $i$
and $j$ at times $t$ and $t'$ are
\begin{align}
  <[\theta_i(t)-\theta_j(t')]^2>
  &= 4\Pi\int_{t'}^tdt_1\int_{t'}^tdt_2
  <\xi_i^{\theta}(t_1)\xi_j^{\theta}(t_2)>\nonumber\\
  &= \delta_{ij}4\Pi(t-t').
\end{align}
I will take $t'=0$ without loss of generality, and $i=j$ as from this
calculation all particle orientations are uncorrelated with each other.

If $\bm{P}(t)=\cos\theta(t)\bm{\hat{e}}_x+\sin\theta(t)\bm{\hat{e}}_y$
is the orientation of a particle at time $t$, then the autocorrelation
function of a particle's orientation is
\begin{align}
  <\bm{P}(t)\cdot\bm{P}(0)>
  &=<\cos\theta(t)\cos\theta(0)+\sin\theta(t)\sin\theta(0)>
  \nonumber\\
  &=<\cos[\theta(t)-\theta(0)]>\nonumber\\
  &=\bigg<\cos\bigg[\sqrt{4\Pi}\int_0^tdt_1\xi^{\theta}(t_1)\bigg]\bigg>.
\end{align}
Now, since $\cos x = (e^{ix}+e^{-ix})/2$, and using the explicit
definition of the ensemble average,
\begin{widetext}
  \begin{align}
    \bigg<\exp\bigg(\pm 4\Pi i\int_0^tdt_1\xi^{\theta}(t_1)\bigg)\bigg>
    &=\frac{1}{Z}\int \mathcal{D}\bm{\xi}
    \exp\bigg(-\frac{1}{2}\int_0^Tdt(\bm{\xi}(t))^2
    \pm \sqrt{4\Pi} i\int_0^tdt_1\xi^{\theta}(t_1)\bigg)\bigg>
    \nonumber\\
    &=\frac{1}{Z}\int \mathcal{D}\bm{\xi}
    \exp\bigg(-\frac{1}{2}\int_0^Tdt(\bm{\xi}(t)\pm\sqrt{4\Pi}i)^2
    \bigg)\bigg>e^{-2\Pi t}\nonumber\\
    &=e^{-2\Pi t}
  \end{align}
\end{widetext}
and so
\begin{align}
  <\bm{P}(t)\cdot\bm{P}(0)>=e^{-2\Pi t}.
\end{align}
which holds for all models with angular diffusion described by eqn
\ref{eq:micro_theta}.

\newpage
\appendix
\begin{widetext}
\section{Ensemble-averaged quantities}

In this section, we evaluate ensemble averaged quantities by using the formal
solution to eqn \ref{eq:micro_pos}, namely
\begin{align}\label{appeq:rformal}
  \bm{r}_i(t)=
  &\bm{r}_i(0)
  +\int_{0}^tdt_1[\bm{F}_i(\{\bm{r}_j\})\nonumber\\
    &+f^P\bm{P}_i(t_1)
    +\sqrt{2}\bm{\xi}_i(t_1)].
\end{align}
Before evaluating, it is also useful to note that
\begin{align}
  <\bm{P}_i(t)\bm{P}_i(t')> = e^{-2(d-1)\Pi |t-t'|}.
\end{align}
in $d$ dimensions.
\subsection{$<\bm{P}_i\cdot\bm{r}_i>$}

\begin{align}\label{appeq:Pdotr}
  <\bm{P}_i(t)\cdot\bm{r}_i(t)>=
  &<\bm{r}_i(0)\cdot\bm{P}_i(t)>
  +<\int_{0}^tdt_1\big[\bm{F}_i(\{\bm{r}_j(t_1)\})
    \cdot\bm{P}_i(t)+f^P\bm{P}_i(t_1)\cdot\bm{P}_i(t)
    +\sqrt{2}\bm{\xi}_i(t_1)\cdot\bm{P}_i(t)\big]>\nonumber\\
  =&\int_{0}^tdt_1\big[<\bm{F}_i(\{\bm{r}_j(t_1)\})
    \cdot\bm{P}_i(t)>+f^P<\bm{P}_i(t_1)\cdot\bm{P}_i(t)>
    +\sqrt{2}<\bm{\xi}_i(t_1)\cdot\bm{P}_i(t)>\big]\nonumber\\
  =&\int_{0}^tdt_1\big[<\bm{F}_i(\{\bm{r}_j(t_1)\})
    \cdot\bm{P}_i(t)>+f^Pe^{-2(d-1)\Pi|t-t_1|}
    \big]\nonumber\\
  =&\int_{0}^tdt_1\big[<\bm{F}_i(\{\bm{r}_j(t_1)\})
    \cdot\bm{P}_i(t)>\big]+\frac{f^P}{2(d-1)\Pi}(1-e^{-2(d-1)\Pi t})
\end{align}


\subsection{$<\bm{\xi}_i(t)\cdot\bm{r}_j(t)>$}
\begin{align}\label{appeq:xidotr}
  <\bm{\xi}_i(t)\cdot\bm{r}_i(t)>=
  &<\bm{r}_i(-\infty)\cdot\bm{\xi}_i(t)>
  +<\int_{-\infty}^tdt_1\big[\bm{F}_i(\{\bm{r}_j\})
    \cdot\bm{\xi}_i(t)+f^P\bm{P}_i(t_1)\cdot\bm{\xi}_i(t)
    +\sqrt{2}\bm{\xi}_i(t_1)\cdot\bm{\xi}_i(t)\big]>\nonumber\\
  &=<\int_{-\infty}^tdt_1\big[
    +\sqrt{2}\bm{\xi}_i(t_1)\cdot\bm{\xi}_i(t)\big]>\nonumber\\
  &=\int_{-\infty}^tdt_1\big[
    +\sqrt{2}d\delta(t-t')\big]\nonumber\\
  &=\sqrt{2}\frac{d}{2}
\end{align}

\subsection{$<\dot{\bm{r}_i}(t)\cdot\bm{P}_i(t)>$}
Taking the product of eqn \ref{eq:micro_pos} with $\bm{P}_i(t')$,
\begin{align}
  <\dot{\bm{r}_i}(t)\cdot\bm{P}_i(t')>
  =&<\bm{F}_i(t)\cdot\bm{P}_i(t')>+<f^P\bm{P}_i(t)\cdot\bm{P}_i(t')>
  +\sqrt{2}<\bm{\xi}_i^r\cdot\bm{P}_i(t')>\nonumber\\
  =&<\bm{F}_i(t)\cdot\bm{P}_i(t')>+f^Pe^{-2\Pi|t-t'|}.
\end{align}
Following \cite{C5SM01412C}, when $t=t'$ this equation is considered to be an
average particle velocity, $v$, projected onto its propulsion direction,
\begin{align}
  v=\frac{1}{N}\sum_{i=1}^N<\dot{\bm{r}_i}(t)\cdot\bm{P}_i(t)>.
\end{align}
Next, I will assume \footnote{AND LATER HOPEFULLY BE ABLE TO SHOW} that
\begin{align}
  <\dot{\bm{r}_i}(0)\cdot\bm{P}_i(t)>
  = <\dot{\bm{r}_i}(0)\cdot\bm{P}_i(0)>e^{-2(d-1)\Pi t}
\end{align}
Now, I will further assume that in the stationary state
\begin{align}
  <\bm{P}_i(t)\cdot\bm{r}_i(t)>
  &=\int_{-\infty}^t<\bm{P}_i(t)\cdot\dot{\bm{r}_i}(t')>dt'
  +<\bm{P}_i(t)\cdot\bm{r}_i(-\infty)>\nonumber\\
  &=\int_0^{\infty}<\bm{P}_i(t')\cdot\dot{\bm{r}_i}(0)>dt'
\end{align}
and so
\begin{align}
  \sum_{i=1}^N<\bm{P}_i(t)\cdot\bm{r}_i(t)>
  &=\sum_{i=1}^N<\dot{\bm{r}_i}(0)\cdot\bm{P}_i(0)>
  \int_0^{\infty}dt'e^{-2(d-1)\Pi t}\nonumber\\
  &=\sum_{i=1}^N<\dot{\bm{r}_i}(0)\cdot\bm{P}_i(0)>
  \frac{1}{2(d-1)\Pi}\nonumber\\
  &=\frac{Nv}{2(d-1)\Pi}\nonumber\\
  &=\frac{N}{2(d-1)\Pi}
  \bigg(f^P+\frac{1}{N}\sum_{i=1}^N<\bm{F}_i\cdot\bm{e}_i>\bigg).
\end{align}
\subsection{Averaged ODE}
If we take the dot product of eqn \ref{eq:micro_pos} with $\bm{r}_i$,
then we get
\begin{align}
  \frac{1}{2}\frac{d}{dt}<\bm{r}_i^2>
  =& <\bm{F}_i\cdot\bm{r}_i>+f^P<\bm{P}_i\cdot\bm{r}_i>
  +\sqrt{2}<\bm{\xi}^r\cdot\bm{r}_i>\nonumber\\
  =& <\bm{F}_i\cdot\bm{r}_i>
  +f^P\int_{-\infty}^tdt_1\big[<\bm{F}_i(\{\bm{r}_j(t_1)\})
    \cdot\bm{P}_i(t)>\big]+\frac{(f^P)^2}{2(d-1)\Pi}
  +d
\end{align}
If you take the sum over all particles, then
\begin{align}
  \sum_{i=1}^N\frac{1}{2}\frac{d}{dt}<\bm{r}_i^2>
  =&\sum_{i=1}^N <\bm{F}_i\cdot\bm{r}_i>
  +f^P\sum_{i=1}^N\int_{-\infty}^tdt_1\big[<\bm{F}_i(\{\bm{r}_j(t_1)\})
    \cdot\bm{P}_i(t)>\big]+\frac{N(f^P)^2}{2(d-1)\Pi}
  +dN\nonumber\\
  =&\sum_{i=1}^N <\bm{F}_i\cdot\bm{r}_i>
  +\frac{f^P}{2(d-1)\Pi}\sum_{i=1}^N<\bm{F}_i(\{\bm{r}_j(t)\})
    \cdot\bm{P}_i(t)>+\frac{N(f^P)^2}{2(d-1)\Pi}
  +dN.
\end{align}
Using the fact that in steady state
$<\bm{r}_i^2(t)>=<\bm{r}_i^2(0)>+2dD_{\mathrm{eff}}t$, this can be written as
\begin{align}\label{appeq:generic_av}
  dND_{\mathrm{eff}}
  =&\sum_{i=1}^N <\bm{F}_i\cdot\bm{r}_i>
  +\frac{f^P}{2(d-1)\Pi}\sum_{i=1}^N<\bm{F}_i
    \cdot\bm{P}_i>+\frac{N(f^P)^2}{2(d-1)\Pi}
  +dN.
\end{align}

\section{Periodic boundary conditions.}
Eqn \ref{appeq:generic_av} is valid for any forces and system setup. In order to
look at periodic boundary conditions, we first note that for pair-wise
interactions in a periodic system, the forces on particle $i$ are
\begin{align}
  \bm{F}_i = \sum_{j=1;j\neq i}^N\sum_{\bm{n}}
  \bm{F}_{ij}(\bm{r}_i-\bm{r}_j-\bm{R}_n)
\end{align}
where $\bm{n}$ is some indexing of which unit cell is being referred to,
e.g. in the main (``local'') unit cell which is arbitrarily chosen,
$\bm{n}=(0,0,0)$ (in $d=3$). $\bm{R}_n$ is then the lattice vector for
cell $\bm{n}$. More general expressions are available for many-body forces
and could be applied here \cite{doi:10.1063/1.3245303}.
We will use the shorthand
$\bm{F}_{ij}^{\bm{n}}\equiv\bm{F}_{ij}(\bm{r}_i-\bm{r}_j-\bm{R}_{\bm{n}})$.

Then, with this definition, eqn \ref{appeq:generic_av} becomes
\begin{align}\label{appeq:PBC_av}
  dND_{\mathrm{eff}}
  =&\sum_{i=1}^N\sum_{j=1;j\neq i}^N\sum_{\bm{n}}
  <\bm{F}_{ij}^{\bm{n}}\cdot\bm{r}_i>
  +\frac{f^P}{2(d-1)\Pi}
  \sum_{i=1}^N\sum_{j=1;j\neq i}^N\sum_{\bm{n}}
  <\bm{F}_{ij}^{\bm{n}}\cdot\bm{P}_i>
  +\frac{N(f^P)^2}{2(d-1)\Pi}+dN.
  =&\frac{1}{2}\sum_{i=1}^N\sum_{j=1;j\neq i}^N\sum_{\bm{n}}
  <\bm{F}_{ij}^{\bm{n}}\cdot(\bm{r}_i-\bm{r}_j)>
  +\frac{f^P}{2(d-1)\Pi}
  \frac{1}{2}\sum_{i=1}^N\sum_{j=1;j\neq i}^N\sum_{\bm{n}}
  <\bm{F}_{ij}^{\bm{n}}\cdot(\bm{P}_i-\bm{P}_j)>
  +\frac{N(f^P)^2}{2(d-1)\Pi}+dN.
\end{align}
Now, in order to get a pressure, we follow again \cite{C5SM01412C} and
subtract the term
\begin{align}
  \frac{1}{2}\sum_{i=1}^N\sum_{j=1;j\neq i}^N\sum_{\bm{n}}
  <\bm{F}_{ij}^{\bm{n}}\cdot\bm{R}_{\bm{n}}>
\end{align}
from both sides of eqn \ref{appeq:PBC_av} to obtain the external pressure
\begin{align}
  dVp^e = dND_{\mathrm{eff}}
  -\frac{1}{2}\sum_{i=1}^N\sum_{j=1;j\neq i}^N\sum_{\bm{n}}
  <\bm{F}_{ij}^{\bm{n}}\cdot\bm{R}_{\bm{n}}>
\end{align}
and the internal pressure
\begin{align}
  dVp^i=
  &\frac{1}{2}\sum_{i=1}^N\sum_{j=1;j\neq i}^N\sum_{\bm{n}}
  <\bm{F}_{ij}^{\bm{n}}\cdot(\bm{r}_i-\bm{r}_j-\bm{R}_{\bm{n}})>
  +\frac{f^P}{2(d-1)\Pi }
  \frac{1}{2}\sum_{i=1}^N\sum_{j=1;j\neq i}^N\sum_{\bm{n}}
  <\bm{F}_{ij}^{\bm{n}}\cdot(\bm{P}_i-\bm{P}_j)>
  +\frac{N(f^P)^2}{2(d-1)\Pi}+dN.
\end{align}
\end{widetext}  
\section{Fokker-Planck equations\label{app:fokkerplanck}}

Now, I can define a vector of length $3N$ with components
\begin{align}
  \bm{r}=(\bm{r}^{(1)},\bm{r}^{(2)},\cdots,\bm{r}^{(N)},\theta_1,\theta_2,\cdots,
  \theta_N).
\end{align}

This will then give the differential equation
\begin{align}
  \frac{d\bm{r}}{dt}=\beta\bm{D}\cdot(\bm{F}(\bm{r})+\bm{v}(\bm{r}))
  +\sqrt{2D_{\alpha\beta}}u_{\beta}(t),
\end{align}
where $D_{\alpha\beta}$ is diagonal, and the values of the diagonal are
$D^t$ for the first $2N$ terms, and $D^r$ for the last $N$ terms.

Next, I will define
\begin{align}
  \bm{x}&\equiv(x_1,x_2,\cdots,x_{3N}),\\
  \nabla&\equiv(\partial_{x_1},\partial_{x_2},\cdots,\partial_{x_{3N}}).
\end{align}

as well as the stochastic probability density operator
\begin{align}
  \hat{\rho}(\bm{x},t)=\delta(\bm{x}-\bm{r}(t)).
\end{align}

Taking the partial derivative of this with respect to time gives
\begin{align}
  \frac{\partial\hat{\rho}}{\partial t}
  &=-(\nabla_{\bm{x}}\delta(\bm{x}-\bm{r}(t))
  \cdot\frac{d\bm{r}}{dt}\nonumber\\
  &=-\nabla_{\bm{x}}\cdot\bigg(\frac{d\bm{r}}{dt}
  \delta(\bm{x}-\bm{r}(t)\bigg)\nonumber\\
  &=-\nabla_{\bm{x}}\cdot\hat{\bm{j}}(\bm{x},t),
\end{align}
where I have defined the stochastic current operator
\begin{align}
  \hat{\bm{j}}(\bm{x},t)\equiv
  &\beta\bm{D}\cdot\bm{F}(\bm{x})\delta(\bm{x}-\bm{r}(t))
  +\beta\bm{D}\bm{v}(\bm{x})\delta(\bm{x}-\bm{r}(t))\nonumber\\
  &+\sqrt{2D_{\alpha\beta}}u_{\beta}(t)
  \delta(\bm{x}-\bm{r}(t))\nonumber\\
  =&\beta\bm{D}\cdot\bm{F}(\bm{x})\hat{\rho}(\bm{x},t)
  +\beta\bm{D}\cdot\bm{v}(\bm{x})\hat{\rho}(\bm{x},t)\nonumber\\
  &+\sqrt{2D_{\alpha\beta}}u_{\beta}(t)\hat{\rho}(\bm{x},t).
\end{align}

You can average $\hat{\rho}(\bm{x},t)$ over the fluctuations $\bm{u}(t)$, and
get the resulting density
\begin{align}
  P(\bm{x},t)=<\hat{\rho}(\bm{x},t)>,
\end{align}
where I have used the definition that for some observable $\hat{\mathcal{O}}$,
\begin{align}
  <\hat{\mathcal{O}}(\bm{u}(t))>=\frac{1}{Z}\int\mathcal{D}\bm{u}
  \hat{\mathcal{O}}(\bm{u}(t))
  \exp\bigg(-\frac{1}{2}\int_0^Tdt(\bm{u}(t))^2\bigg).
\end{align}
From this, I can calculate that noise-averaged current as
\begin{align}
  \bm{J}(\bm{x},t)=
  &\beta\bm{D}\cdot(\bm{F}(\bm{x})P(\bm{x},t)
  +\bm{v}(\bm{x})P(\bm{x},t))\nonumber\\
  &+\sqrt{2D_{\alpha\beta}}<u_{\beta}(t)\hat{\rho}(\bm{x},t)>.
\end{align}

In order to write the noise term above in terms of $P(\bm{x},t)$, I will use
the properties of the Dirac-delta function, namely,
\begin{align}
  \delta(\bm{x})=\int\frac{d^dk}{(2\pi)^d}\exp(-ik_jx_j),
\end{align}
to write
\begin{align}
  <u_{\beta}\hat{\rho}(\bm{x},t)>=\int\frac{d^dk}{(2\pi)^d}u_{\beta}(t)
  \exp(-ik_j(x_j-r_j(t))).
\end{align}

Since correlations between the noise
$<u_{\alpha}(t)u_{\beta}(t)>=\delta_{\alpha\beta}\delta(t-t^{\prime})$ only
occur for fluctuations happening at the same time, we can write
\begin{align}
  \bm{r}(t)=
  &\bm{r}(t-\Delta t)
  +\int_{t-\Delta t}^tdt_1[\beta\bm{D}(\bm{F}(\bm{r}(t_1))
    +\bm{v}(\bm{r}(t_1)))\nonumber\\
    &+\sqrt{2D_{j\gamma}}u_{\gamma}(t_1)]
\end{align}
as $\Delta t\to0$. Then, since
\begin{align}
  \exp\bigg[\int_{t-\Delta t}^tdt_1(\bm{F}(\bm{r}(t_1)
  +\bm{v}({\bm{r}(t_1)})\bigg]=1+O(\Delta t),
\end{align}
this can be written as
\begin{widetext}
  \begin{align}
    <u_{\beta}(t)\exp(-ik_jr_j(t))>=
    &\bigg<u_{\beta}(t)\exp\bigg[-ik_jr_j(t-\Delta t)-ik_j
      \int_{t-\Delta t}^tdt_1(F_j(\bm{r}(t_1)+v_j{\bm{r}(t_1)}
      +\sqrt{2D_{j\gamma}}u_{\gamma}(t_1))\bigg]\bigg>\nonumber\\
    =&\bigg<u_{\beta}(t)\exp(-ik_jr_j(t-\Delta t))
    \bigg[1-ik_j\int_{t-\Delta t}^tdt_1
      \sqrt{2D_{j\gamma}}u_{\gamma}(t_1)\bigg]\bigg>+O(\Delta t).
  \end{align}
\end{widetext}
Next, since $t-\Delta t<t_1,t$, there won't be any correlations between the
fluctuations at earlier times and those at $t$, so using the fact that for uncorrelated
functions $<AB>=<A><B>$, I get
\begin{widetext}
\begin{align}
  <u_{\beta}(t)\exp(-ik_jr_j(t))>
  &=\bigg<\exp(-ik_jr_j(t-\Delta t))\bigg>
  \bigg<u_{\beta}(t)\bigg[1-ik_j\int_{t-\Delta t}^tdt_1
    \sqrt{2D_{j\gamma}}u_{\gamma}(t_1)\bigg]\bigg>
  +O(\Delta t)\nonumber\\
  &=\bigg<\exp(-ik_jr_j(t-\Delta t))\bigg>
  \bigg(<u_{\beta}(t)>
  -\bigg<ik_j\int_{t-\Delta t}^tdt_1
  \sqrt{2D_{j\gamma}}u_{\gamma}(t_1)
  u_{\beta}(t)\bigg>\bigg)
  +O(\Delta t)\nonumber\\
  &=-\bigg<ik_j\sqrt{2D_{j\gamma}}\int_{t-\Delta t}^tdt_1
  \delta_{\beta \gamma}\delta(t-t_1)\bigg>
  \bigg<\exp(-ik_jr_j(t-\Delta t))\bigg>
  +O(\Delta t)\nonumber\\
  &=-\bigg<ik_j\sqrt{2D_{j\beta}}\int_{t-\Delta t}^tdt_1
  \delta(t-t_1)\bigg>
  \bigg<\exp(-ik_jr_j(t-\Delta t))\bigg>
  +O(\Delta t)\nonumber\\
  &=-\bigg<ik_j\sqrt{2D_{j\beta}}\int_0^{\Delta t}du
  \delta(u)\bigg>
  \bigg<\exp(-ik_jr_j(t-\Delta t))\bigg>
  +O(\Delta t)\nonumber\\
  &=-\frac{1}{2}ik_j\sqrt{2D_{j\beta}}
  \bigg<\exp(-ik_jr_j(t-\Delta t))\bigg>
  +O(\Delta t)\nonumber\\.
\end{align}
\end{widetext}
Taking the limit as $\Delta t\to0$, this becomes
\begin{align}
  <u_{\beta}(t)&\exp(-ik_jr_j(t)>=\nonumber\\
  &-\frac{1}{2}ik_j\sqrt{2D_{j\beta}}\bigg<\exp(-ik_jr_j(t))\bigg>.
\end{align}
Taking the inverse Fourier transform then gives
\begin{align}
  <u_{\beta}(t)\hat{\rho}(\bm{x},t)>=-\frac{1}{2}\sqrt{2D_{j\beta}}
  \partial_jP(\bm{x},t).
\end{align}
Finally, inserting this into the current I get
\begin{align}
  J_{\alpha}(\bm{x},t)=
  &\beta D_{\alpha\beta}F_{\beta}(\bm{x})P(\bm{x},t)
  +\beta D_{\alpha\beta}v_{\beta}(\bm{x})P(\bm{x},t)\nonumber\\
  &+\sqrt{2D_{\alpha\beta}}\bigg(-\frac{1}{2}\sqrt{2D_{j\beta}}
  \partial_jP(\bm{x},t)\bigg)\nonumber\\
  =&\beta D_{\alpha\beta}F_{\beta}(\bm{x})P(\bm{x},t)
  +\beta D_{\alpha\beta}v_{\beta}(\bm{x})P(\bm{x},t)\nonumber\\
  &-\sqrt{D_{\alpha\beta}D_{\gamma\beta}}
  \partial_{\gamma}P(\bm{x},t))
\end{align}

Since in our case, $D_{\alpha\beta}$ is diagonal, this just gives eqn
\ref{eq:Nparticlecurrent} as desired.

\bibliographystyle{plain}
\bibliography{bib}

\end{document}
