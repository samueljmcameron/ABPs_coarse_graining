\documentclass[twocolumn,amsmath,amssymb,aps]{revtex4-1}%{article}
%\usepackage[hmargin=1.25cm,vmargin=2.5cm]{geometry}
\usepackage[utf8]{inputenc}
\usepackage{bm}
%\usepackage{amsmath}
%\usepackage{amssymb}
\usepackage{comment}
\usepackage{graphicx}
\usepackage{dcolumn}
\usepackage[caption=false]{subfig}

\begin{document}
 
\title{Three-body correlation function.}
\author{Sam Cameron}
\affiliation{%
  University of Bristol}%
\date{\today}

\begin{abstract}
\end{abstract}

\maketitle

\section{N-body correlation function definition.}

The $n$ body correlation function is defined as
\begin{align}
  g^(n)(\bm{r}_1,\bm{r}_2,\cdots,\bm{r}_n)
  = \frac{A^n}{N^n}p^(n)(\bm{r}_1,\bm{r}_2,\cdots,\bm{r}_n)
\end{align}
where $p^(n)$ is the probability of finding one particle within a
volume $d^2r_1$ centred at $\bm{r}_1$, a second particle within a
volume $d^2r_2$ centred at $\bm{r}_2$, and a third particle at the
volume $d^2r_3$. For example, in equilibrium for a system in the
canonical ensemble, this probability is given by
\begin{align}\label{eq:sub_prob}
  p^(n)(\bm{r}_1,\bm{r}_2,\cdots,\bm{r}_n)
  =\frac{N!}{(N-n)!}\frac{1}{Z}
  \int\prod_{k=n+1}^Nd^2r_k\exp[-\beta U(\{\bm{r}_i\})]
\end{align}
where $Z$ is the partition function of the system, and the prefactors
account for the (assumed) indistinguishability of particles.

However, for our purposes we are only really concerned with measured or
\textit{objective} probabilities. For discrete processes, these
probabilities are conceptually simple. Namely, if a random process is repeated
N times, then the probability of an event $A$ occuring is given by
\begin{align}\label{eq:obj_prob}
  p(A) = \lim_{N\to\infty} \frac{N_A}{N}
\end{align}
where $N_A$ is the number of times in which the event occurs. This is in
contrast to the statistical mechanics probability written in eqn
\ref{eq:sub_prob} above, which is
a theoretical or \textit{subjective} probability.

For us, we must adjust eqn \ref{eq:obj_prob} to suit our needs. In our case,
the event $A$ will be more complex.

First of all, we will assume translational invariance within the system.
So, our events (such as $A$ in eqn \ref{eq:obj_prob}) will be the event
that $n$ particles are at relative displacements from each other, e.g.
\begin{align}
  p^(n)(\bm{r}_1,\bm{r}_2,\cdots,\bm{r}_n)d^2r_1d^2r_2\cdots d^2r_n
  = p^(n)(\bm{r}_{12},\bm{r}_{13},\cdots,\bm{r}_{n-1,n})d^2r_1d^2r_2
  \cdots d^2r_n
\end{align}

In our simulations, we will iterate over the particles in the system once,
so the origin at iteration $i$ will be $\bm{r}_i$. Then, given that we are
at position $\bm{r}_i$, we need to determine the probability of another
particle being at 

\section{2-body correlation function.}


\bibliographystyle{plain}
\bibliography{bib}

\end{document}
