\documentclass[../main.tex]{subfiles}

\begin{document}

Consider three interacting spheres, with their positions and orientations as
\begin{subequations}
  \label{eqs:3hardspherepositions}
  \begin{align}
    \bm{r}_1 &= \bm{0},\\
    \bm{r}_2 &= r_0\bm{\hat{e}}_x,\\
    \bm{r}_3 &= \frac{r_0}{2}(\bm{\hat{e}}_x+\sqrt{3}\bm{\hat{e}}_y),
  \end{align}
\end{subequations}
and
\begin{subequations}
  \label{eqs:3hardsphereorientations}
  \begin{align}
    \bm{u}_1 &= \frac{1}{2}(\sqrt{3}\bm{\hat{e}}_x+\bm{\hat{e}}_y),\\
    \bm{u}_2 &= \frac{1}{2}(-\sqrt{3}\bm{\hat{e}}_x+\bm{\hat{e}}_y),\\
    \bm{u}_3 &= -\bm{\hat{e}}_y,
  \end{align}
\end{subequations}
respectively. Then, the differences in positions and orientations are
\begin{subequations}
  \label{eqs:3hardsphererelativepositions}
  \begin{align}
    \bm{r}_{12} &= r_0\bm{\hat{e}}_x-\bm{0} = r_0\bm{\hat{e}}_x,\\
    \bm{r}_{13} &= \frac{r_0}{2}(\bm{\hat{e}}_x+\sqrt{3}\bm{\hat{e}}_y)-\bm{0}
    = \frac{r_0}{2}(\bm{\hat{e}}_x+\sqrt{3}\bm{\hat{e}}_y),\\
    \bm{r}_{23} &= \frac{r_0}{2}(\bm{\hat{e}}_x+\sqrt{3}\bm{\hat{e}}_y)
                  -r_0\bm{\hat{e}}_x
    =\frac{r_0}{2}(-\bm{\hat{e}}_x+\sqrt{3}\bm{\hat{e}}_y),
  \end{align}
\end{subequations}
and
\begin{subequations}
  \label{eqs:3hardsphererelativeorientations}
  \begin{align}
    \bm{u}_{12} &= \frac{1}{2}(-\sqrt{3}\bm{\hat{e}}_x+\bm{\hat{e}}_y)
                  -\frac{1}{2}(\sqrt{3}\bm{\hat{e}}_x+\bm{\hat{e}}_y)
    = -\sqrt{3}\bm{\hat{e}}_x,\\
    \bm{u}_{13} &= -\bm{\hat{e}}_y-\frac{1}{2}(\sqrt{3}\bm{\hat{e}}_x
                  +\bm{\hat{e}}_y)
    = -\sqrt{3}\frac{1}{2}(\bm{\hat{e}}_x+\sqrt{3}\bm{\hat{e}}_y),\\
    \bm{u}_{23} &=  \bm{\hat{e}}_y-\frac{1}{2}(-\sqrt{3}\bm{\hat{e}}_x
                  +\bm{\hat{e}}_y)
    = \sqrt{3}\frac{1}{2}(\bm{\hat{e}}_x-\sqrt{3}\bm{\hat{e}}_y),
  \end{align}
\end{subequations}
respectively. Then, from eqns \ref{eqs:3hardsphererelativepositions}
and \ref{eqs:3hardsphererelativeorientations}, one arrives at the generic
relationships
\begin{align}
  \bm{\hat{r}}_{ij} &= -\frac{\bm{u}_{ij}}{\sqrt{3}},\\
  \bm{u}_{ij}\cdot\bm{r}_{ij} &= -\sqrt{3}r_0.
\end{align}
Inserting these into eqn \ref{eq:3hardspherevelocities}
\begin{align}
  \frac{\bm{V}_i}{\lambda}
  &=\bm{\hat{u}}_i-\sum_{j\neq i}^N \bigg[w'(r_{ij})
    (\bm{u}_{ij}\cdot\bm{r}_{ij})\frac{\bm{r}_{ij}}{r_{ij}}
    +w(r_{ij})\bm{u}_{ij}\bigg]\nonumber\\
  &=\bm{\hat{u}}_i-\sum_{j\neq i}^N \bigg[w'(r_0)
    r_0\bm{u}_{ij}+w(r_0)\bm{u}_{ij}\bigg]
\end{align}
and then taking the difference of two (out of three) velocities
\begin{align}
  \frac{\bm{V}_i}{\lambda}-\frac{\bm{V}_j}{\lambda}
  =&-\bm{u}_{ij}-\sum_{k\neq i}^N \bigg[w'(r_0)
     r_0\bm{u}_{ik}+w(r_0)\bm{u}_{ik}\bigg]\nonumber\\
   &+\sum_{k\neq j}^N \bigg[w'(r_0)
     r_0\bm{u}_{jk}+w(r_0)\bm{u}_{jk}\bigg].
\end{align}
Taking $i=1$ and $j=2$ gives
\begin{align}
  \frac{\bm{V}_1}{\lambda}-\frac{\bm{V}_2}{\lambda}
  =&-\bm{u}_{12}-w'(r_0)r_0\bm{u}_{12}-w(r_0)\bm{u}_{12}
  \nonumber\\
  &-w'(r_0)r_0\bm{u}_{13}-w(r_0)\bm{u}_{13}\nonumber\\
  &+w'(r_0)r_0\bm{u}_{21}+w(r_0)\bm{u}_{21}\nonumber\\
  &+w'(r_0)r_0\bm{u}_{23}+w(r_0)\bm{u}_{23}\nonumber\\
  =&-\bm{u}_{12}-2w'(r_0)r_0\bm{u}_{12}-2w(r_0)\bm{u}_{12}
  \nonumber\\
  &-(w'(r_0)r_0+w(r_0))(\bm{u}_{13}-\bm{u}_{23})\nonumber\\
  =&\bm{u}_{12}\big[-1-3w'(r_0)r_0-3w(r_0)\big].
\end{align}
Setting the lefthand side of the above equation then gives
\ref{eq:hardsphereBC}. The same result occurs for any choices of $i\neq j$
for $N=3$ particles in this trimer configuration. Since we are in
dimensionless variables we can set $r_0=1$.

Inserting the homogeneous solution, eqn \ref{eq:Whomo}, into eqn
\ref{eq:hardsphereBC} gives the coeffiecient of $w(r)$ as
\begin{align}
  c_2 = -\frac{1}{3\sqrt{\Pi}K'(\sqrt{\Pi})}>0.
\end{align}
     

\end{document}
