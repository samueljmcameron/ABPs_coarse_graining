\documentclass[../main.tex]{subfiles}

\begin{document}


In this section, we will derive the partition function of the
generalized steady state. Since we are also interested in the
correlation functions of the couplings between orientation and
position, we also include an extra parameter $J$, which will
then serve as a way to generate this couplings (in the language
of probability, $Z(J)$ is a characteristic function, and the
partition function is $Z(0)$). With this, we get
\begin{widetext}
\begin{align}\label{appeq:genfunc}
  Z(J)
  =&\int\prod_{i=1}^N\big(d^2r_id\theta_i)
  \times\exp\bigg(-\sum_{i=1}^N\sum_{j=1}^{i-1}\big[
    \mathcal{V}(r_{ij})+\lambda (w(r_{ij})+J)\bm{r}_{ij}
    \cdot\bm{u}_{ij}\big]\bigg)\nonumber\\
  =&\int\prod_{i=1}^N\big(d^2r_id\theta_i)
  \times\exp\bigg(-\frac{1}{2}\sum_{i,j;j\neq i}\big[
    \mathcal{V}(r_{ij})+\lambda (w(r_{ij})+J)\bm{r}_{ij}
    \cdot\bm{u}_{ij}\big]\bigg)\nonumber\\
  =&\int\prod_{i=1}^N\big(d^2r_id\theta_i)
  \exp\bigg(-\frac{1}{2}\sum_{i=1}^N\sum_{j\neq i}^N
  \mathcal{V}(r_{ij})
  +\lambda\sum_{i=1}^N\sum_{j\neq i}^N\bm{u}_{i}
  \cdot \bm{r}_{ij}(w(r_{ij})+J)\bigg).
\end{align}
\end{widetext}
where we have used eqn \ref{eq:u_itou_ij}. With the definition of
\begin{align}\label{appeq:b_i}
  \tilde{b}_i(J) = |\tilde{\bm{b}}_i(J)|\equiv
  \bigg|\sum_{j\neq i}^N(w(r_{ij})+J)
  \bm{r}_{ij}\bigg|,
\end{align}
or equivalently,
\begin{align}\label{appeq:b_i2}
  \sum_{i=1}^N\tilde{b}_i^2(J)
  =&\sum_{i=1}^N\sum_{j\neq i}^N(w(r_{ij})+J)^2r_{ij}^2\nonumber\\
  &+\sum_{i=1}^N\sum_{j\neq i}^N\sum_{\substack{k\neq i \\ k\neq j}}^N
  (w(r_{ij})+J)(w(r_{ik})+J)\bm{r}_{ij}\cdot\bm{r}_{ik},
\end{align}
this is re-written as
\begin{widetext}
\begin{align}
  Z=&\int\prod_{i=1}^N\big(d^2r_id\theta_i)
     \exp\bigg(-\frac{1}{2}\sum_{i=1}^N\sum_{j\neq i}^N
     \mathcal{V}(r_{ij})+2\lambda\sum_{i=1}^N\bm{u}_{i}
     \cdot \bm{b}_{i}\bigg)\nonumber\\
  =&\int\prod_{i=1}^N\big(d^2r_i)
     \exp\bigg(-\frac{1}{2}\sum_{i=1}^N\sum_{j\neq i}^N
     \mathcal{V}(r_{ij})\bigg)\int\prod_{i=1}^N\bigg[d\theta_i
     \exp(\lambda\bm{u}_{i}\cdot \bm{b}_{i})\bigg].
\end{align}
\end{widetext}
The angular integrals can be re-written as
\begin{widetext}
\begin{align}
  \int_0^{2\pi} d\theta_u\exp(\bm{u}\cdot\lambda\bm{b})
  &=\int_0^{2\pi} d\theta_u\exp(\lambda|\bm{b}|\cos(\theta_b-\theta_u))
    \nonumber\\
  &=\int_{-\theta_b}^{2\pi-\theta_b} d\phi\exp(\lambda|\bm{b}|\cos(\phi))
    \nonumber\\
  &=\bigg(\int_{-\theta_b}^0
    +\int_0^{2\pi}
    -\int_{2\pi-\theta_b}^{2\pi}\bigg)
    d\phi\exp(\lambda|\bm{b}|\cos(\phi))
    \nonumber\\
  &=\bigg(\int_{-\theta_b}^0
    -\int_{2\pi-\theta_b}^{2\pi}\bigg)
    d\phi\exp(\lambda|\bm{b}|\cos(\phi))
    +\int_0^{2\pi}d\phi\exp(|\bm{b}|\cos(\phi))
    \nonumber\\
    &=\int_{-\theta_b}^0d\phi\exp(\lambda|\bm{b}|\cos(\phi))
    -\int_{2\pi-\theta_b}^{2\pi}d\phi'\exp(\lambda|\bm{b}|\cos(\phi'))
    +\int_0^{2\pi}d\phi\exp(\lambda|\bm{b}|\cos(\phi))
    \nonumber\\
  &=\int_{-\theta_b}^0d\phi\exp(\lambda|\bm{b}|\cos(\phi))
    +\int_{\theta_b}^{0}d\phi''\exp(\lambda|\bm{b}|\cos(2\pi-\phi''))
    +\int_0^{2\pi}d\phi\exp(\lambda|\bm{b}|\cos(\phi))
    \nonumber\\
  &=\int_{-\theta_b}^0d\phi\exp(\lambda|\bm{b}|\cos(\phi))
    -\int_{-\theta_b}^{0}d\phi\exp(\lambda|\bm{b}|\cos(\phi))
    +\int_0^{2\pi}d\phi\exp(\lambda|\bm{b}|\cos(\phi))
    \nonumber\\
  &=\int_0^{2\pi}d\phi\exp(\lambda|\bm{b}|\cos(\phi))
\end{align}
\end{widetext}
which is $I_0(\lambda|b|)$ using the definition of eqn \ref{eq:Iintegral}.
Inserting these results back into the partition function gives eqn
\ref{eq:partitionfunction}.

From the definition of eqn \ref{appeq:genfunc}, one can see that
\begin{align}
  \bigg<\sum_{i,j;j\neq i} \bm{u}_{ij}\cdot\bm{r}_{ij}
  \bigg>= -\frac{\partial \ln Z}{\partial \lambda J}.
\end{align}

\end{document}
