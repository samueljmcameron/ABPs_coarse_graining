\documentclass[../main.tex]{subfiles}

\begin{document}
In this section, I would like to determine the many body terms
of $b_i^{2n}$, where $b_i$ is defined in eqn \ref{eq:b_i}.
The squared term is
\begin{align}
  b_i^2
  =&\sum_{j\neq i}^Nw^2(r_{ij})r_{ij}^2\nonumber\\
   &+\sum_{j\neq i}^N\sum_{\substack{k\neq i \\ k\neq j}}^N
  w(r_{ij})w(r_{ik})\bm{r}_{ij}\cdot\bm{r}_{ik}.
\end{align}
I will define
\begin{subequations}
  \begin{align}
    r &= \sum_{j\neq i}^Nw^2(r_{ij})r_{ij}^2,\\
    s &=\sum_{j\neq i}^N\sum_{\substack{k\neq i \\ k\neq j}}^N
    w(r_{ij})w(r_{ik})\bm{r}_{ij}\cdot\bm{r}_{ik},
  \end{align}
\end{subequations}
so we can utilize the binomial theorem
\begin{align}
  b_i^{2n} = \sum_{k=0}^n{ n \choose k} r^{n-k}s^k.
\end{align}
We would like to limit the expansion to just including two-body
and three-body terms. Limiting to just a two body expansion is
straightforward, as any product $r^{n-k}s^k$ with $k\neq 0$ will
have a product of a two-body and a three-body term, which is
irreducible to anything less than a three-body term (if this
wasn't the case, then the three-body term itself would be
reducible to a two-body term, leading to a contradiction of us
assuming it was a three-body term). Therefore, we start by
using the multinomial formula,
\begin{align}
  (\sum_{j\neq i}^Nx_j)^k = \sum_{j_1+j_2+\cdots+j_N=k}
  { k \choose j_1,j_2,\cdots j_N }\prod_{t=1,t\neq i}^N x_t^{j_t}.
\end{align}
For us, this simplifies to 
\begin{widetext}
  \begin{align}
    r^k &= \bigg(\sum_{j\neq i}^Nw^2(r_{ij})r_{ij}^2\bigg)^k
    \nonumber\\
    &=\sum_{j\neq i}^Nw^{2k}(r_{ij})r_{ij}^{2k}
    + \frac{1}{2}\sum_{l\neq i}\sum_{\substack{m\neq i \\ m\neq l}}
    \sum_{\substack{j_l+j_m=k \\ j_l\neq 0 \\ j_m\neq 0}}
        { k \choose j_l,j_m }
        (w(r_{il})r_{il})^{2j_l}(w(r_{im})r_{im})^{2j_m}
        +\textrm{four (and higher) body terms}.
  \end{align}
\end{widetext}
From the looks of this, it seems like three-body terms will be
difficult to include in any sort of coherent way...

Continuing on with just two body terms then, $b_i^{2k}$ simplifies
substantially to
\begin{align}
  b_i^{2k}=\sum_{j\neq i}^N(w(r_{ij})r_{ij})^{2k}
  +\textrm{three (and higher) body terms}.
\end{align}
we can write the effective
potential as
\begin{widetext}
  \begin{align}
    V_{\mathrm{eff}}(\{\bm{r}_i\})
    =&\frac{1}{2}\sum_{i=1}^N\sum_{j\neq i}^N\mathcal{V}(r_{ij})
    -\sum_{i=1}^N\ln(I_0(\lambda b_i))\nonumber\\
    =&\frac{1}{2}\sum_{i=1}^N\sum_{j\neq i}^N\mathcal{V}(r_{ij})
    -\sum_{i=1}^N\ln\bigg(\sum_{k=0}^{\infty}
    \frac{\bigg(\frac{1}{4}(\lambda^2 b_i^2)\bigg)^k}{(k!)^2}\bigg)\nonumber\\
    =&\frac{1}{2}\sum_{i=1}^N\sum_{j\neq i}^N\mathcal{V}(r_{ij})
    -\sum_{i=1}^N\ln\bigg(\sum_{k=0}^{\infty}
    \frac{\bigg(\frac{1}{4}\lambda^2\bigg)^k}{(k!)^2}b_i^{2k}\bigg)\nonumber\\
    \approx&\frac{1}{2}\sum_{i=1}^N\sum_{j\neq i}^N\mathcal{V}(r_{ij})
    -\sum_{i=1}^N\ln\bigg(\sum_{j\neq i}\sum_{k=0}^{\infty}
    \frac{\bigg(\frac{1}{4}\lambda^2\bigg)^k}{(k!)^2}
    (w(r_{ij})r_{ij})^{2k}\bigg)\nonumber\\
    \approx&\frac{1}{2}\sum_{i=1}^N\sum_{j\neq i}^N\mathcal{V}(r_{ij})
    -\sum_{i=1}^N\ln\bigg(1+\sum_{j\neq i}\sum_{k=1}^{\infty}
    \frac{\bigg(\frac{1}{4}\lambda^2\bigg)^k}{(k!)^2}
    (w(r_{ij})r_{ij})^{2k}\bigg)\nonumber\\
    \approx&\frac{1}{2}\sum_{i=1}^N\sum_{j\neq i}^N\mathcal{V}(r_{ij})
    -\sum_{i=1}^N\sum_{n=0}^{\infty}\frac{(-1)^{n+1}}{n}
    \bigg(\sum_{j\neq i}\sum_{k=1}^{\infty}
    \frac{\bigg(\frac{1}{4}\lambda^2\bigg)^k}{(k!)^2}
    (w(r_{ij})r_{ij})^{2k}\bigg)^n\nonumber\\
    \approx&\frac{1}{2}\sum_{i=1}^N\sum_{j\neq i}^N\mathcal{V}(r_{ij})
    -\sum_{i=1}^N\sum_{j\neq i}\sum_{n=0}^{\infty}\frac{(-1)^{n+1}}{n}
    \bigg(\sum_{k=1}^{\infty}
    \frac{\bigg(\frac{1}{4}\lambda^2\bigg)^k}{(k!)^2}
    (w(r_{ij})r_{ij})^{2k}\bigg)^n\nonumber\\
    =&\frac{1}{2}\sum_{i=1}^N\sum_{j\neq i}^N\mathcal{V}(r_{ij})
    -\sum_{i=1}^N\sum_{j\neq i}\ln(I_0(\lambda w(r_{ij})r_{ij}))
    +\textrm{three (and higher) body terms}.
  \end{align}
\end{widetext}

\end{document}
