\documentclass[../main.tex]{subfiles}

\begin{document}

Now, according to \cite{LEE198895}
\footnote{The section of this book is riddled with
  typos, so I should cross-reference with another source or check myself.},
the virial expansion for a three body potential with the form of eqn
\ref{eq:Veff} can be written in terms of the two-body and three body
Mayer F-functions in the usual way, i.e.
\begin{align}
  \beta P = \rho + B_2\rho^2 +B_3\rho^3+\cdots,
\end{align}
where $\rho\equiv N/A$, the second virial coefficient is
\begin{align}\label{eq:B2}
  B_2=-\frac{1}{2A}\int d^2r_1 d^2r_2 f_{12},
\end{align}
the third virial coefficient is
\begin{widetext}
  \begin{align}\label{eq:B3}
    B_3=-\frac{1}{3A}\int d^2r_1 d^2r_2 d^2r_3
    [f_{12}f_{23}f_{31}+f_{123}(1+f_{12})(1+f_{23})(1+f_{31})],
  \end{align}
  \end{widetext}
the two-body Mayer function is
\begin{align}\label{eq:f12}
  f_{12} = \exp(-W_2(r_{12}))-1,
\end{align}
and the three-body Mayer function is
\begin{align}\label{eq:f13}
  f_{123} = \exp(-W_3(\bm{r}_1,\bm{r}_2,\bm{r}_3))-1.
\end{align}

\end{document}
