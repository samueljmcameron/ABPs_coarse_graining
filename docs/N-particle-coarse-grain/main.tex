\documentclass[twocolumn,amsmath,amssymb,aps]{revtex4-1}%{article}
%\usepackage[hmargin=1.25cm,vmargin=2.5cm]{geometry}
\usepackage[utf8]{inputenc}
\usepackage{bm}
%\usepackage{amsmath}
%\usepackage{amssymb}
\usepackage{comment}
\usepackage{graphicx}
\usepackage{dcolumn}
\usepackage[caption=false]{subfig}

\begin{document}
 
\title{calculation: coarse-graining active brownian particles}
\author{Sam Cameron}
\affiliation{%
  University of Bristol}%
\date{\today}

\begin{abstract}
\end{abstract}

\maketitle

\section{Writing the active brownian particles equations of motion into
  the correct form}

The equations of motion for active brownian particles are:
\begin{subequations}
  \label{eqs:basicABPsODEs}
  \begin{align}
    \frac{d\bm{r}_i}{dt}&=\beta D^t\big(\bm{F}_i
    +f^P\bm{P}_i\big)
    +\sqrt{2D^t}\tilde{\bm{u}}_i,\label{eq:micro_pos}\\
    \frac{d\theta_i}{dt}&=\sqrt{2D^r}w_i.\label{eq:micro_theta}
  \end{align}
\end{subequations}
where $\bm{r}_i=(x_i,y_i)$ are the particle positions,
$\bm{P}_i=(\cos\theta_i,\sin\theta_i)$ are the particle orientations,
and $\tilde{\bm{u}}_i=(\tilde{u}_i,v_i)$, $w_i$ are Gaussian white noise
with zero mean and unit variance, i.e. for any two components of the vector
$\bm{s}=(u_i,v_i,w_i)$,
\begin{align}
  \big<s_i(t)\big>=0;\;\;\;\;\; \big<s_i(t)s_j(t^{\prime})\big>=
  \delta_{ij}\delta(t-t^{\prime}).
\end{align}.

This is equivalent to a Fokker-Planck equation for the $N$-body Fokker-Planck
equation (see appendix \ref{app:fokkerplanck})
\begin{align}\label{eq:NparticleFP}
  \frac{\partial P}{\partial t} +
  \nabla_{\bm{x}}\cdot\bm{J}=0.
\end{align}
The current $\bm{J}$ is
\begin{align}\label{eq:Nparticlecurrent}
  \bm{J}(\bm{x},t)=
  &\bm{D}\cdot(\beta\bm{F}(\bm{x})P(\bm{x},t)
  +\beta\bm{v}(\bm{x})P(\bm{x},t)\nonumber\\
  &-\nabla P(\bm{x},t)).
\end{align}
where $\bm{D}$ is a diagonal, and equal to $D^t$ for the first $2N$ components,
and equal to $D^r$ for the last $N$.
The vector $\bm{F}$ is for simplicity just two-body interactions, and so is a
concatenation of the two-component vectors $\bm{F}_i$ from eqn
\ref{eq:micro_pos} along with a final $N$ components all equal to zero (so the
total number of components of $\bm{F}$ is $3N$, but only the first $2N$ are
non-zero). The two body interactions will be written as
\begin{align}\label{eq:general_conservative_force}
  \bm{F}_i&=-\nabla_{i}\mathcal{H}(\bm{x})\nonumber\\
          &=-\nabla_{i}\sum_{j=1}^N\sum_{k=1}^{j-1}\mathcal{V}(r_{jk}),
\end{align}
where $\bm{r}_{ij}=\bm{r}_j-\bm{r}_i$.
The vector $\bm{v}$ cannot be represented
as the gradient of a scalar, and is explicitly given by a concatenation of two-
component vectors $\bm{v}_i$, and then a final $N$ zeros added on. The
$\bm{v}_i$ take the form
\begin{align}\label{eq:dissipative_force}
  \bm{v}_i =f^P\bm{u}(\theta_i)
\end{align}
where $\bm{u}(\theta)=(\cos\theta,\sin\theta)$.

It can be shown \cite{liverpool2018nonequilibrium} that there exists a
generalized steady state of the form
\begin{align}
  P_{ss}=\exp(-h(\bm{x})).
\end{align}
For ABPs, $h(\bm{x})$ satisfies the equation
\begin{align}\label{eq:steadystatecondition}
  D^t\beta\sum_{i=1}^{N}L_i(h)
  +D^r\sum_{i=1}^{N}\big[(\partial_{\theta_i}h)^2
  -\partial_{\theta_i}^2h\big]=0
\end{align}
where
\begin{align}\label{eq:L_i}
  L_i(h) =& \beta^{-1}(\nabla_ih)^2+\nabla_i^2\mathcal{H}
  +(\bm{v}_i-\nabla_i\mathcal{H})\cdot\nabla_ih\nonumber\\
  &-\beta^{-1}\nabla_i^2h-\nabla_i\cdot\bm{v}_i.
\end{align}

I can explicitly calculate the terms not involving $h$ in eqn \ref{eq:L_i}
using eqn \ref{eq:general_conservative_force}, which gives the relevant
derivatives of $\mathcal{H}$ and $\bm{v}_i$,
\begin{subequations}
  \label{eqs:partials_of_L_i}
  \begin{align}
    \nabla_{\alpha}\mathcal{H}
    &=-\sum_{i\neq\alpha}^N\mathcal{V}^{\prime}
    (r_{\alpha i})\frac{\bm{r}_{\alpha i}}{r_{\alpha i}},
    \label{eq:partial_H_of_L_i}\\
    \nabla_{\alpha}^2\mathcal{H}
    &=\sum_{i\neq\alpha}^N\bigg[\frac{\mathcal{V}^{\prime}
        (r_{\alpha i})}{r_{\alpha i}}
      +\mathcal{V}^{\prime\prime}(r_{\alpha i})\bigg],
    \label{eq:partialsq_H_of_L_i}\\
    \nabla_i\cdot \bm{v}_i
    &= 0.\label{eq:partial_v_of_L_i}
  \end{align}
\end{subequations}

\section{Computing the steady $h(\bm{x})$ function.}

Generically, we expect that the microscopic $h$ function will be invariant
under translations, so
\begin{align}
  h(\{\bm{r}_i+\bm{T}\},\{\theta_i\}) = h(\{\bm{r}_i\},\{\theta_i\}).
\end{align}
Further to this invariance, the system must also be invariant
under arbitrary rotations, and so
\begin{align}
  h(\{\bm{Q}(\phi)\bm{r}_i\},\{\theta_i+\phi\})
  =h(\{\bm{r}_i\},\{\theta_i\}).
\end{align}
The system should also have periodic rotational invariance, so
\begin{align}
  h(\{\bm{r}_i\},\{\theta_i+2\pi\})
    =h(\{\bm{r}_i\},\{\theta_i\}).
\end{align}
And finally, interchanging particles should not vary the results, so swapping
$\bm{r}_i,\theta_i$ with $\bm{r}_j,\theta_j$ should leave $h$ invariant.

To generically have all of these features, we require
\begin{align}
  h(\{r_{ij}\},\{\bm{\hat{u}}_{ij}\cdot\bm{\hat{u}}_{kl}\},
  \{\bm{r}_{ij}\cdot\bm{\hat{u}}_{kl}\})
\end{align}
where $\bm{\hat{u}}_i = (\cos\theta_i,\sin\theta_i)$.

\subsection{Non-dimensionalisation.}

Before doing this calculation, I will write everything in dimensionless form.
If $\tilde{x}$ are the dimensional variables, then
\begin{align}
  \tilde{\mathcal{H}} &= \beta^{-1}\mathcal{H},\\
  \tilde{h} &= h,\\
  \tilde{\bm{r}} &= \sigma\bm{r},\\
  \tilde{\nabla} &= \sigma^{-1}\nabla.
\end{align}
Here, $\sigma$ is some microscopic length scale defined in the Hamiltonian
$\mathcal{H}$. I will also define
\begin{align}
  \Pi &= D^r\sigma^2/(2D^t),\\
  \lambda &= f^P\sigma\beta.
\end{align}
From this point forward, all variables will be held in their dimensionless
forms.


Then, I can re-write eqn \ref{eq:steadystatecondition} as a sum of terms
linear and non-linear in $\mathrm{h}$,
\begin{align}\label{eq:dim_ss}
  \sum_{i=1}^N\big[\mathcal{L}_i^{(0)}(h)
    +\lambda\mathcal{L}_i^{(1)}(h)\big] = 0
\end{align}
where
\begin{align}
  \mathcal{L}_i^{(0)}(h)=&(\nabla_i h)^2 + \nabla_i^2\mathcal{H}
  -\nabla_i\mathcal{H}\cdot\nabla_ih-\nabla_i^2h\nonumber\\
  &+2\Pi\big[(\partial_{\theta_i}h)^2-\partial_{\theta_i}^2h\big],
\end{align}
and
\begin{align}
  \mathcal{L}_i^{(1)}(h)=\bm{u}_i\cdot\nabla_ih.
\end{align}
When $\lambda = 0$ ($f^P=0$), the solution is just $h=\mathcal{H}$ as expected.

\subsection{Perturbative guess for coupled $h$ function.}

I will take a perturbative guess to the solution, with $\lambda=f^P\sigma\beta$
as my perturbation parameter. So, my guess will be
\begin{align}\label{eq:hperturb}
  h(\lambda) = \mathcal{H}
  + \lambda\frac{\partial h}{\partial \lambda}\bigg|_{\lambda=0}.
\end{align}

Differentiating eqn \ref{eq:dim_ss} with respect to $\lambda$ gives
\begin{align}\label{eq:first_perturb_DE}
  0
  =&\sum_{i=1}^N\big\{\partial_{\lambda}\mathcal{L}_i^{(0)}(h)
    + \mathcal{L}_i^{(1)}(h(0))
    +\lambda\partial_{\lambda}\mathcal{L}_i^{(1)}
    (h)\big\}\bigg|_{\lambda=0}\nonumber\\
  =&\sum_{i=1}^N\big\{2\nabla_i \mathcal{H}\cdot\nabla_i\zeta
    - \nabla_i\mathcal{H}\cdot\nabla_i\zeta
    -\nabla_i^2\zeta\nonumber\\
    &+2\Pi\big[2\partial_{\theta_i}\mathcal{H}\partial_{\theta_i}\zeta
    -\partial_{\theta_i}^2\zeta\big]
    + \bm{u}_i\cdot\nabla_i\mathcal{H}
    +(0)\bm{u}_i\cdot\nabla_i\zeta\big\}\nonumber\\
  =&\sum_{i=1}^N\big[\nabla_i \mathcal{H}\cdot\nabla_i\zeta
    -\nabla_i^2\zeta - 2\Pi
    \partial_{\theta_i}^2\zeta
    + \bm{u}_i\cdot\nabla_i\mathcal{H}\big]
\end{align}
where $\zeta\equiv\partial_{\lambda}h|_{\lambda=0}$.

Assuming $h$ is analytic in its variables, I will Taylor expand it to
first order as
\begin{align}\label{eq:zetaform}
  \zeta &= \sum_{i=1}^N\sum_{j=1}^{i-1}
      w(r_{ij})\bm{r}_{ij}\cdot\bm{u}_{ij}.
\end{align}
Here, I am neglecting terms of the form $\bm{u}_{ij}\cdot\bm{u}_{ij}$ as
they will only be (on average) non-zero when a non-zero polarity is present
in the system (and MIPS is not a polar phase).


Since, for $i\neq j$,
\begin{align}
  \nabla_i(\bm{r}_{jk}\cdot\bm{u}_{jk})
  &= \bm{u}_{jk}(\delta_{ik}-\delta_{ij}),\\
  \nabla_i r_{jk}
  &=\frac{\bm{r}_{jk}}{r_{jk}}(\delta_{ik}-\delta_{ij}),\\
  \partial_{\theta_i}^2(\bm{u}_{ij}\cdot\bm{r}_{ij})
  &=\bm{u}_i\cdot\bm{r}_{ij},\\
  \partial_{\theta_j}^2(\bm{u}_{ij}\cdot\bm{r}_{ij})
  &=-\bm{u}_j\cdot\bm{r}_{ij},
\end{align}
the terms are
\begin{align}
  \nabla_{i}\zeta = -\sum_{j\neq i}^N \bigg[w'(r_{ij})
  (\bm{u}_{ij}\cdot\bm{r}_{ij})\frac{\bm{r}_{ij}}{r_{ij}}
  +w(r_{ij})\bm{u}_{ij}\bigg],
\end{align}

\begin{align}
  \nabla_i^2\zeta=\sum_{j\neq i}^N\bigg[w''(r_{ij})
  +3\frac{w'(r_{ij})}{r_{ij}}\bigg](\bm{u}_{ij}\cdot\bm{r}_{ij}).
\end{align}

\begin{align}
  \partial_{\theta_i}^2\zeta=\sum_{j\neq i}w(r_{ij})\bm{u}_i\cdot\bm{r}_{ij}
\end{align}

Inserting these into eqn \ref{eq:first_perturb_DE}
\begin{widetext}
\begin{align}
  0=
  &\sum_{i=1}^N\sum_{j\neq i}^N\sum_{k\neq i}^N
  \mathcal{V}^{\prime}(r_{ij})\frac{\bm{r}_{ij}}{r_{ij}}
  \cdot\bigg[w'(r_{ik})
    (\bm{u}_{ik}\cdot\bm{r}_{ik})\frac{\bm{r}_{ik}}{r_{ik}}
    +w(r_{ik})\bm{u}_{ik}\bigg]
  -\sum_{i=1}^N\sum_{j\neq i}^N\bigg[w''(r_{ij})
    +3\frac{w'(r_{ij})}{r_{ij}}\bigg](\bm{u}_{ij}\cdot\bm{r}_{ij})\nonumber\\
  &-2\Pi\sum_{i=1}^N\sum_{j\neq i}^Nw(r_{ij})\bm{u}_i\cdot\bm{r}_{ij}
  -\sum_{i=1}^N\sum_{j\neq i}^N 
  \mathcal{V}^{\prime}(r_{ij})\frac{\bm{u}_i\cdot\bm{r}_{ij}}{r_{ij}}
\end{align}
\end{widetext}

You can factor this equation a bit, using the result that
\begin{align}\label{eq:u_itou_ij}
  \sum_{i=1}^N\sum_{j\neq i}f(r_{ij}) \bm{u}_i\cdot\bm{r}_{ij}
  =-\frac{1}{2}\sum_{i=1}^N\sum_{j\neq i}f(r_{ji})\bm{u}_{ij}\cdot\bm{r}_{ij}
\end{align}
to get
\begin{widetext}
\begin{align}\label{eq:WODE_exact}
  0=
  &\sum_{i=1}^N\sum_{j\neq i}^N\bm{r}_{ij}\cdot\bigg\{\sum_{k\neq i}^N\bigg[
    \frac{\mathcal{V}^{\prime}(r_{ik})}{r_{ik}}
    w'(r_{ij})
    \cdot\frac{\bm{r}_{ik}\cdot\bm{r}_{ij}}{r_{ij}}\bm{u}_{ij}
    +\frac{\mathcal{V}^{\prime}(r_{ik})}{r_{ik}}w(r_{ik})\bm{u}_{ik}\bigg]
  -\bigg[w''(r_{ij})
    +3\frac{w'(r_{ij})}{r_{ij}}\bigg]\bm{u}_{ij}\nonumber\\
  &+\Pi w(r_{ij})\bm{u}_{ij}
  +\mathcal{V}^{\prime}(r_{ij})\frac{\bm{u}_{ij}}{2r_{ij}}\bigg\}\nonumber\\
  =&\sum_{i=1}^N\sum_{j\neq i}^N\bm{r}_{ij}\cdot\bm{u}_{ij}
  \bigg\{\sum_{k\neq i}^N\bigg[
    \frac{\mathcal{V}^{\prime}(r_{ik})}{r_{ik}}
    w'(r_{ij})\frac{\bm{r}_{ik}\cdot\bm{r}_{ij}}{r_{ij}}\bigg]
  -w''(r_{ij})-3\frac{w'(r_{ij})}{r_{ij}}\nonumber\\
  &+\Pi w(r_{ij})
  +\frac{\mathcal{V}^{\prime}(r_{ij})}{2r_{ij}}\bigg\}
  +\sum_{i=1}^N\sum_{j\neq i}^N\bm{r}_{ij}\cdot\sum_{k\neq i}^N
  \bm{u}_{ik}\frac{\mathcal{V}^{\prime}(r_{ik})}{r_{ik}}w(r_{ik})
\end{align}
\end{widetext}
So far, the approximations we have made in calculating $h$ are that for
small activity, $h$ is perturbatively close to $\mathcal{H}$, and we have
taken a lowest order guess in what this perturbative term is, in terms of some
arbitrary function $w(r_{ij})$. Now, to be able to get some sort of solution
for eqn \ref{eq:WODE_exact}, we need to make some further approximation. The
way forward is to assume that the system has translational and rotational
invariance in its pairwise-correlation function.

The standard definition of the pairwise-correlation operator is
\begin{align}
  <n(\bm{x}_1)>\hat{g}_2(\bm{x}_1,\bm{x}_2)<(\bm{x}_2)>
  &=\sum_{i=1}^N\sum_{j\neq i}^N\delta(\bm{x}_1-\bm{r}_i)
    \delta(\bm{x}_2-\bm{r}_j).
\end{align}

If the system is translationally invariant, then $\bm{r}=\bm{x}_1-\bm{x}_2$
and
\begin{align}
  \hat{g}_2(\bm{x}_1,\bm{x}_2)
  &=\hat{g}_2(\bm{r})\nonumber\\
  &=\frac{1}{n^2}\sum_{i=1}^N\sum_{j\neq i}^N\delta(\bm{r}+\bm{x}_2-\bm{r}_i)
  \delta(\bm{x}_2-\bm{r}_j)\nonumber\\
  &=\frac{1}{n^2}\frac{1}{V}\int d^2x_2 \sum_{i=1}^N\sum_{j\neq i}^N
  \delta(\bm{r}+\bm{x}_2-\bm{r}_i)\delta(\bm{x}_2-\bm{r}_j)\nonumber\\
  &=\frac{V}{N^2}\sum_{i=1}^N\sum_{j\neq i}
  \delta(\bm{r}+\bm{r}_{ij})\nonumber\\
  &=\frac{V}{N}\sum_{j=1,j\neq i}^N\delta(\bm{r}-\bm{r}_{ij})\nonumber\\
\end{align}
where in the last line I have used the fact that the sum over $j$ runs over
all possible values of the difference $\bm{r}_i-\bm{r}_j$, and so each of the
$N-1\approx N$ terms are identical.

Then, any term of the form
\begin{align}
  \sum_{j\neq i} \bm{f}(\bm{r}_{ij})
  &= \sum_{j\neq i}\int \delta(\bm{r}+\bm{r}_{ij})\bm{f}(-\bm{r})
  d^2r\nonumber\\
  &=\frac{N}{V}\int\hat{g}_2(\bm{r})\bm{f}(-\bm{r})d^2r.
\end{align}

Now, for $g_2(r)=<\hat{g}_2(\bm{r})>$ is the average pair-correlation function,
and is radially symmetric, any integrals over functions of the form
$\bm{\hat{e}}_r f(r)$ will be zero by symmetry. With this in mind, I can see
from both three-sum terms in eqn \ref{eq:WODE_exact} that both have vector
functions which are spherically symmetric, the first being 
\begin{align}
  \frac{\mathcal{V}^{\prime}(r_{ik})}{r_{ik}}w'(r_{ij})\bm{r}_{ik},
\end{align}
and the second simply $\bm{r}_{ij}$. 

Eqn \ref{eq:WODE_exact} is then reducible to
\begin{align}\label{eq:WODE_isotropic}
  w''(r)+3\frac{w'(r)}{r}-\Pi w(r)-\frac{\mathcal{V}'(r)}{2r}=0.
\end{align}
\subsection{Hard-sphere result.}
For the case of a hard sphere potential, the homogeneous solution to eqn
\ref{eq:WODE_isotropic} (plus boundary conditions) gives the final solution.
The homogeneous solution can be found analytically, and is
\begin{align}
  w_h(r)=a_1\frac{J_1(i\sqrt{\Pi}r)}{r}-a_2\frac{Y_1(-i\sqrt{\Pi}r)}{r},
\end{align}
where
\begin{align}
  J_n(z) =& \frac{1}{\pi}\int_0^{\pi}\cos(z\sin\theta-n\theta)d\theta,\\
  Y_n(z) =& -\frac{2\bigg(\frac{1}{2}z\bigg)^{-n}}{\sqrt{\pi}
    \Gamma\bigg(\frac{1}{2}-n\bigg)}
  \int_1^{\infty}\frac{\cos(zt)}{(t^2-1)^{n+1/2}}dt.
\end{align}
From the latter equation, I can see that $Y(-i\sqrt{\Pi}r)=-Y(i\sqrt{\Pi}r)$,
and so
\begin{align}
  w_h(r)=a_1\frac{J_1(i\sqrt{\Pi}r)}{r}+a_2\frac{Y_1(i\sqrt{\Pi}r)}{r}.
\end{align}

I can re-write this in terms of the modified bessel functions of the first and
second kinds, using
\begin{align}
  J_1(ix) &= i I_1(x),\\
  Y_1(ix) &= \frac{2i}{\pi}K_1(x)-I_1(x).
\end{align}
Therefore, the homogeneous solution of eqn \ref{eq:WODE_isotropic} can be
written as
\begin{align}
  w_h(r) = c_1\frac{I_1(\sqrt{\Pi}r)}{r} + c_2\frac{K_1(\sqrt{\Pi}r)}{r}.
\end{align}
Now, I want $w_h(r\to\infty)\to0$, so $c_1=0$. This leaves
\begin{align}\label{eq:Whomo}
  w_h(r) = c_2\frac{K_1(\sqrt{\Pi}r)}{r}.
\end{align}

To uniquely solve the ODE for $w(r)$, a second boundary conditions is required.
One physically motivated option is considering the minimum distance of
interaction between particles.

Consider three particles which are all moving toward a single, central point at
the origin.
The (dimensionless) velocity of each particle is
\begin{align}\label{eq:3hardspherevelocities}
  \bm{V}_i
  &=\nabla h -\nabla_i\mathcal{H} + \lambda \bm{\hat{u}}_i\nonumber\\
  &=\lambda \bigg(\bm{\hat{u}}_i+\nabla_i\sum_{j=1}^N\sum_{k=1}^{j-1}w(r_{jk})
  \bm{r}_{jk}\cdot\bm{u}_{jk}\bigg)\nonumber\\
  &=\lambda \bigg(\bm{\hat{u}}_i-\sum_{j\neq i}^N \bigg[w'(r_{ij})
    (\bm{u}_{ij}\cdot\bm{r}_{ij})\frac{\bm{r}_{ij}}{r_{ij}}
    +w(r_{ij})\bm{u}_{ij}\bigg]\bigg),
\end{align}
where I have used eqns \ref{eq:hperturb} and \ref{eq:zetaform}. When all three
particles are hitting each other simultaneously, they should all have zero
relative velocity with respect to each other. Therefore, for this three
particle system, the boundary condition on $w(r)$ at $r_0=1$ (see appendix
\ref{app:hardsphereBC}) is
\begin{align}\label{eq:hardsphereBC}
  -1-3w'(1)-3w(1)=0.
\end{align}


Thus, inserting eqn \ref{eq:Whomo} into the boundary condition, $c_2$ is
uniquely determined and the resulting solution is
\begin{align}
  w_h(r) &= -\frac{1}{3\sqrt{\Pi}K_1'(\sqrt{\Pi})}\frac{K_1(\sqrt{\Pi}r)}{r}.
\end{align}
Since $K_1(x)>0$, $K_1'(x)<0$ for all $x>0$, this implies that $w_h(r)>0$
for all $r>0$.

Analytical and numerical solutions of eqn \ref{eq:Whomo} subject to eqn
\ref{eq:hardsphereBC} are shown in Figure \ref{fig:1}.

\begin{figure}[!t]
  \subfloat{
    \includegraphics[width=0.5\textwidth]{Figures/fig1.pdf}
  \label{fig:1a}}
  \subfloat{\label{fig:1b}}
  \caption{\protect\subref{fig:1a} Effective potential $w_h$ and
    \protect\subref{fig:1b} its gradient for hard-sphere interactions between
    active brownian particles, with $\Pi=0.1$. The numerical result has not
    been fine-tuned as it only approximately solves the boundary conditions
    (i.e. it was found by eye using a shooting method).}\label{fig:1}
\end{figure}

\subsection{Soft potential.}
For a different, soft, two-body potential such as the Weeks-Chandler-Anderson
(WCA) interaction in dimensionless form\footnote{With $\epsilon$ re-scaled
  by temperature and length in terms of the potential minimum (\textbf{not}
  the zero of the potential conventionally denoted by $\sigma$).}
\begin{equation}\label{eq:WCAapprox}
  \mathcal{V}^{WCA}_{ij} =
  \begin{cases}
    \epsilon[(r_{ij}^{-12}
      -2r_{ij}^{-6})+1],
    & r<1 \\
    0, & \mathrm{otherwise},
  \end{cases}
\end{equation}
there are two steps to computing the boundary condition. The first step is a
generic force-balance, which gives the effective ``hard sphere'' minimum
distance. Because of the more complicated nature of the interactions, this new
minimum distance will not just be $r=1$, but will be on the same scale. A new
energy scale (such as $\epsilon$ in eqn \ref{eq:WCAapprox}) will be introduced,
completely decoupled from $k_BT$.

\subsubsection{Force balance.}
As with the hard-sphere case, we consider a trimer of converging interacting
particles with coordinates as given in appendix \ref{app:hardsphereBC}. The
external force on the $i^{th}$ particle is just given by
$-\nabla_i\mathcal{H}$. This force must be matched by particle $i$'s active
force, i.e.
\begin{align}
  0
  &= \lambda\bm{u}_i-\nabla_i\mathcal{H}\nonumber\\
  &= \lambda\bm{u}_i+\sum_{j\neq i}\mathcal{V}'(r_{ij})
    \frac{\bm{r}_{ij}}{r_{ij}}.
\end{align}
For an isolated trimer, using the coordinates of appendix
\ref{app:hardsphereBC} one arrives at the result
\begin{align}\label{eq:genericsoftsphereforcebalance}
  \lambda+\sqrt{3}\mathcal{V}'(r_0)=0.
\end{align}
If you specifically consider the WCA approximation of eqn \ref{eq:WCAapprox},
you end up with
\begin{align}\label{eq:WCAforcebalance}
  \Omega-r_0^{-13}+r_0^{-7}=0.
\end{align}
where $\Omega\equiv{\lambda}/{12\sqrt{3}\epsilon}$.
Now, when $\Omega\ll 1$, the minimum distance is
\begin{align}\label{eq:softsphereforcebalancesmalllambda}
  r_0^{\Omega\ll 1} \approx1+\frac{1}{21}(1-\sqrt{1+7\Omega})
\end{align}
and when $\Omega\gg 1$, the repulsive force dominates and
\begin{align}\label{eq:softsphereforcebalancelargelambda}
  r_0^{\Omega\gg 1}\approx
  &\Omega^{-1/13}\bigg\{1
    +\frac{1}{14}\frac{(13-7\Omega^{-6/13})}{(13-4\Omega^{-6/13})}\nonumber\\
  &\times\bigg(1-\sqrt{1+28\Omega^{-6/13}\frac{(13-4\Omega^{-6/13})}
    {(13-7\Omega^{-6/13})^2}}\bigg)\bigg\}.
\end{align}
These approximate solutions are compared to exact (numerical
\footnote{using secant-method of root finding in SciPy})
solutions of eqn \ref{eq:WCAforcebalance} in Figure \ref{fig:2}.

\begin{figure}[!t]
  \subfloat{
    \includegraphics[width=0.5\textwidth]{Figures/fig2.pdf}
  \label{fig:2a}}
  \subfloat{\label{fig:2b}}
  \caption{\protect\subref{fig:2a} Exact (numerical) distance $r_0$ between
    spheres in a trimer at zero force (from eqn \ref{eq:WCAforcebalance}).
    \protect\subref{fig:2b} relative error between $r_0$ and its approximate
    solutions at small and large $\Omega\equiv\lambda/(12\sqrt{3}\epsilon)$,
    measured as $(r_0^{\Omega\ll1}-r_0)/r_0$ (orange dashed line) and
    $(r_0^{\Omega\gg1}-r_0)/r_0$ (green dash-dotted line), respectively.
    The approximation $r_0^{\Omega\gg1}$ becomes poor around
    $\Omega\approx0.5$, which is why the green dash-dotted line in
    \protect\subref{fig:2b} is cut off at $\Omega = 0.5$. Note that these
    results are independent of $\Pi$.}\label{fig:2}
\end{figure}
  
\subsubsection{Matching conditions for $w(r)$.}
To solve the full inhomogeneous ODE in eqn \ref{eq:WODE_isotropic}, one must
match two solutions at $r=1$. The solution for $r>1$ is just the homogeneous
form from eqn \ref{eq:Whomo}, and will be denoted by $w_{+}(r)$. The solution
for $r<1$ is given through variation of parameters, denoted $w_{-}(r)$.

We require the potential to be continuous and differentiable at $r=1$, finite
at $r\to\infty$, and having zero relative velocity for a trimer in the stable
configuration (the same one as discussed in appendix \ref{app:hardsphereBC}).
This gives four conditions,
\begin{subequations}
  \label{eqs:genericmatchingconditions}
  \begin{align}
    &w_{+}(1)=w_{-}(1),\label{eq:match1}\\
    &w_{+}'(1) = w_{-}'(1),\label{eq:match2}\\
    &w_{+}(r\to\infty)=0,\label{eq:match3}\\
    &w_{-}'(r_0)r_0+w_{-}(r_0)=-\frac{1}{3}\label{eq:match4}
  \end{align}
\end{subequations}

\subsubsection{Variation of parameters to determine $w(r)$.}
The general form of the solution to eqn \ref{eq:WODE_isotropic} for $r<1$ is
\begin{align}
  w_{-}(r) = \frac{I_1(\sqrt{\Pi}r)}{r}\nu_1(r)
  +\frac{K_1(\sqrt{\Pi}r)}{r}\nu_2(r)
\end{align}
where, using eqn \ref{eq:wronskian},
\begin{subequations}
  \label{eqs:ODEcoefficientsVOP}
  \begin{align}
    \nu_1(r)
    &= k_1 - \int_{r_0}^rds\frac{K_1(\sqrt{\Pi}s)}{s}
      \frac{\mathcal{V}'(s)}{2s}W^{-1}(s)\nonumber\\
    &= k_1 + \frac{1}{2}\int_{r_0}^rdsK_1(\sqrt{\Pi}s)
      \mathcal{V}'(s)s,\nonumber\\
    \nu_2(r)
    &= k_2 + \int_{r_0}^rds\frac{I_1(\sqrt{\Pi}s)}{s}
      \frac{\mathcal{V}'(s)}{2s}W^{-1}(s)\nonumber\\
    &= k_2 - \frac{1}{2}\int_{r_0}^rdsI_1(\sqrt{\Pi}s)
      \mathcal{V}'(s)s,
  \end{align}
\end{subequations}
and we have just selected the lower limit of integration as $r_0$ for later
convenience.

Inserting this into the matching conditions allows us to determine $k_1$, and
relates $k_2$ to $c_2$,
\begin{subequations}
  \label{eqs:specificmatching}
  \begin{align}
    0 &= k_1 + \frac{1}{2}\int_{r_0}^1dsK_1(\sqrt{\Pi}s)
        \mathcal{V}'(s)s,\label{eq:specificmatch1}\\
    c_2  &= k_2 - \frac{1}{2}\int_{r_0}^1dsI_1(\sqrt{\Pi}s)
           \mathcal{V}'(s)s\label{eq:specificmatch2}.
  \end{align}
\end{subequations}
Now, inserting eqns \ref{eqs:ODEcoefficientsVOP} into eqn \ref{eq:match4},
\begin{align}
  -\frac{1}{3}=
  &r_0\bigg(-k_1\frac{I_1(\sqrt{\Pi}r_0)}{r_0^2}+k_1\sqrt{\Pi}
    \frac{I_1'(\sqrt{\Pi}r_0)}{r_0}\nonumber\\
  &-k_2\frac{K_1(\sqrt{\Pi}r_0)}{r_0^2}+k_2\sqrt{\Pi}
    \frac{K_1'(\sqrt{\Pi}r_0)}{r_0}\bigg)\nonumber\\
  & +k_1\frac{I_1(\sqrt{\Pi}r_0)}{r_0} + k_2\frac{K_1(\sqrt{\Pi}r_0)}{r_0}
    \nonumber\\
  =&\sqrt{\Pi}[k_1I_1'(\sqrt{\Pi}r_0)+k_2K_1'(\sqrt{\Pi}r_0)],
\end{align}
and so now $k_2$ is determined in terms of $k_1$. Since eqn
\ref{eq:specificmatch2} relates $k_2$ to $c_2$, all of the coefficients are
accounted for. Explicitly,
\begin{align}
  k_1 =& -\frac{1}{2}\int_{r_0}^1dsK_1(\sqrt{\Pi}s)
         \mathcal{V}'(s)s,\label{eq:k1},\\
  k_2 =& \frac{-\frac{1}{3\sqrt{\Pi}}+\frac{1}{2}\int_{r_0}^1ds
         K_1(\sqrt{\Pi}s)\mathcal{V}'(s)sI_1'(\sqrt{\Pi}r_0)}
         {K_1'(\sqrt{\Pi}r_0)}\label{eq:k2},\\
  c_2 =& \frac{-\frac{1}{3\sqrt{\Pi}}+\frac{1}{2}\int_{r_0}^1ds
         K_1(\sqrt{\Pi}s)\mathcal{V}'(s)sI_1'(\sqrt{\Pi}r_0)}
         {K_1'(\sqrt{\Pi}r_0)}\nonumber\\
       &-\frac{1}{2}\int_{r_0}^1dsI_1(\sqrt{\Pi}s)
         \mathcal{V}'(s)s\label{eq:c2}.
\end{align}

The full solution of $w(r)$ is shown in Figure \ref{fig:3} for different
values of $\Omega$ and $\Pi$, the only two free parameters which control
$w(r)$.

\begin{figure}[!t]
  \subfloat{
    \includegraphics[width=0.5\textwidth]{Figures/fig3.pdf}
  \label{fig:3a}}
  \subfloat{\label{fig:3b}}
  \caption{Effective potential $w$ for WCA interactions with
    \protect\subref{fig:3a} $\Pi=1$ held constant, and
    \protect\subref{fig:3b} $\Omega=1$ held constant. The value of $r_0$,
    when an isolated trimer experiences zero force, is shown in black dots
    for each parameter set.
  }\label{fig:3}
\end{figure}

Since the steady-state is given by $P_{ss}\propto\exp(-h)$, with
\begin{align}\label{eq:specifichfunction}
  h(\{\bm{r}_i\},\{\bm{u}_i\}) = \sum_{i=1}^N\sum_{j=1}^{i-1}\big(
  \mathcal{V}(r_{ij})+\lambda w(r_{ij})\bm{r}_{ij}\cdot\bm{u}_{ij}\big),
\end{align}
the maximum in $w(r)$ only corresponds to a minimum probability state if
$\bm{r}_{ij}\cdot\bm{u}_{ij}>0$, i.e. if the particles are moving away from
each other. Therefore, $\mathrm{max}(w)>0$ corresponds to a \textit{maximum}
probability state in our trimer configuration, since the particles are
moving toward each other. We expect that in certain parameter regimes, 
particles on average will tend to have $\bm{r}_{ij}\cdot\bm{u}_{ij}<0$ (as
observed in MIPS), and so the maximum in $w(r)$ will correspond ot a
maximum probability state.

We can see from Figure \ref{fig:3a} that when either the activity, $\lambda$,
is large with respect to the energy scale of the WCA potential, $\epsilon$,
(i.e. small values of $\Omega=\lambda/(12\sqrt{3}\epsilon$), the maximum in
$w(r)$ is pushed to smaller $r$ as the ABPs are able more effectively
penetrate the potential. It seems reasonable that $\Omega$ is a measure of
how ``soft'' the potential is, and not just the potential energy scale
$\epsilon$.

Similarly, larger values of $\Pi\propto D^r/D^t$ in Figure
\ref{fig:3b} move the maximum in $w(r)$ to smaller $r$. This is slightly
more difficult to interpret physically, as increasing rotational diffusion
will in turn change the average behaviour of $\bm{r}_{ij}\cdot\bm{u}_{ij}$.
Naively, one would think that increasing rotational diffusion would decrease
the likelihood of becoming stuck in a potential well, but Figure \ref{fig:3b}
appears to indicate that the potential becomes deeper and less wide as
$Pi$ increases. A deeper look into how the statistical mechanics of the
system, to determine how e.g. $\bm{u}_{ij}$ behaves on average, is required
to understand macroscopic behaviour.

\section{Statistical mechanics of Active Brownian particles.}

The framework of the last section is in theory enough to compute $h$ for any
system of ABPs. In this section, we compute some statistical properties of
ABPs with $h$ having the form of eqn \ref{eq:specifichfunction}.
The ``partition function'' of ABPS is then (see appendix \ref{app:statmech})
\begin{align}\label{eq:partition}
  Z =(2\pi)^N\int\prod_{i=1}^N\big(d^2r_iI_0(\lambda b_i))
  \exp\bigg(-\frac{1}{2}\sum_{i=1}^N\sum_{j\neq i}^N
  \mathcal{V}(r_{ij})\bigg)
\end{align}
where we have defined
\begin{align}\label{eq:b_i}
  b_i = |\bm{b}_i|\equiv\bigg|\sum_{j\neq i}^Nw(r_{ij})
  \bm{r}_{ij}\bigg|,
\end{align}
and the integral
\begin{align}\label{eq:Iintegral}
  I_0(x)\equiv\frac{1}{\pi}\int_0^{\pi}d\phi\exp(x\cos\phi)
\end{align}
is just the order zero modified Bessel function.

From eqn \ref{eq:partition}, we can therefore express the steady-state of
the system for $h$ via integrating out the angular dependence, yielding an
effective potential. This steady state is
\begin{align}
  p(\{\bm{r}_i\})=\frac{1}{\tilde{Z}}\exp[-V_{\mathrm{eff}}
  (\{\bm{r}_i\})]
\end{align}
where
\begin{align}
  V_{\mathrm{eff}}(\{\bm{r}_i\})
  \equiv\frac{1}{2}\sum_{i=1}^N\sum_{j\neq i}^N
  \mathcal{V}(r_{ij})-\sum_{i=1}^N\ln I_0(\lambda b_i).
\end{align}
and $\tilde{Z}$ is 
\begin{align}
  \tilde{Z}=\int\prod_{i=1}^N(d^2r_i)\exp[-V_{\mathrm{eff}}(\{\bm{r}_i\})].
\end{align}
This effective potential can be expanded as
\begin{align}
  V_{\mathrm{eff}}(\{\bm{r}_i\})
  =&\frac{1}{2}\sum_{i=1}^N\sum_{j\neq i}^N\mathcal{V}(r_{ij})\nonumber\\
   &-\frac{1}{4}\sum_{i=1}^N\bigg(\lambda^2b_i^2
     -\frac{\lambda^4}{16}b_i^4+\cdots\bigg)
     \nonumber\\
  =&\frac{1}{2}\sum_{i=1}^N\sum_{j\neq i}^N\mathcal{V}(r_{ij})
     -\frac{1}{4}\lambda^2\sum_{i=1}^Nb_i^2+\mathcal{O}(\lambda^4),
\end{align}
where, consistent with above solutions of $w(r)$, only the lowest order
corrections in $\lambda$ are considered. The sum over $b_i^2$ can be
decomposed into a sum of two-body and three-body interactions,
\begin{align}
  \sum_{i=1}^Nb_i^2
  =&\sum_{i=1}^N\sum_{j\neq i}^Nw^2(r_{ij})r_{ij}^2\nonumber\\
   &+\sum_{i=1}^N\sum_{j\neq i}^N\sum_{\substack{k\neq i \\ k\neq j}}^N
  w(r_{ij})w(r_{ik})\bm{r}_{ij}\cdot\bm{r}_{ik}.
\end{align}
This motivates the definitions of effective two-body and three-body
interactions,
\begin{subequations}
  \label{eqs:2and3body}
  \begin{align}
    W_2(\bm{r}_1,\bm{r}_2)
    &= \mathcal{V}(r_{12})
      -\frac{1}{2}\lambda^2 r_{12}^2w^2(r_{12}),\label{eq:2body}\\
    W_3(\bm{r}_1,\bm{r}_2,\bm{r}_3)
    &= -\frac{1}{4}\lambda^2w(r_{12})w(r_{13})\bm{r}_{12}\cdot\bm{r}_{13},
      \label{eq:3body}
  \end{align}
\end{subequations}
which allows us to write the effective potential as
\begin{align}
  V_{\mathrm{eff}}(\{\bm{r}_i\})
  &=\frac{1}{2}\sum_{i,j;j\neq i}W_2(\bm{r}_i,\bm{r}_j)
    +\sum_{\substack{i,j,k;j\neq i \\k\neq i k\neq j}}
    W_3(\bm{r}_i,\bm{r}_j,\bm{r}_k)
\end{align}
The partition function can be written (to fourth order in $\lambda$) as
\begin{widetext}
  \begin{align}
    \tilde{Z}=\int\prod_{i=1}^N(d^2r_i)
    \exp\bigg[-\frac{1}{2}\sum_{i}\sum_{j\neq i}W_2(\bm{r}_i,\bm{r}_j)
    -\sum_{i}\sum_{j\neq i}\sum_{\substack{k\neq i \\ k\neq j}}
    W_3(\bm{r}_i,\bm{r}_j,\bm{r}_k)\bigg].
  \end{align}
\end{widetext}

%%%%%%%%%%%% begin appendix %%%%%%%%%%%%%


\appendix

\section{Fokker-Planck equations\label{app:fokkerplanck}}

Now, I can define a vector of length $3N$ with components
\begin{align}
  \bm{r}=(\bm{r}^{(1)},\bm{r}^{(2)},\cdots,\bm{r}^{(N)},\theta_1,\theta_2,\cdots,
  \theta_N).
\end{align}

This will then give the differential equation
\begin{align}
  \frac{d\bm{r}}{dt}=\beta\bm{D}\cdot(\bm{F}(\bm{r})+\bm{v}(\bm{r}))
  +\sqrt{2D_{\alpha\beta}}u_{\beta}(t),
\end{align}
where $D_{\alpha\beta}$ is diagonal, and the values of the diagonal are
$D^t$ for the first $2N$ terms, and $D^r$ for the last $N$ terms.

Next, I will define
\begin{align}
  \bm{x}&\equiv(x_1,x_2,\cdots,x_{3N}),\\
  \nabla&\equiv(\partial_{x_1},\partial_{x_2},\cdots,\partial_{x_{3N}}).
\end{align}

as well as the stochastic probability density operator
\begin{align}
  \hat{\rho}(\bm{x},t)=\delta(\bm{x}-\bm{r}(t)).
\end{align}

Taking the partial derivative of this with respect to time gives
\begin{align}
  \frac{\partial\hat{\rho}}{\partial t}
  &=-(\nabla_{\bm{x}}\delta(\bm{x}-\bm{r}(t))
  \cdot\frac{d\bm{r}}{dt}\nonumber\\
  &=-\nabla_{\bm{x}}\cdot\bigg(\frac{d\bm{r}}{dt}
  \delta(\bm{x}-\bm{r}(t)\bigg)\nonumber\\
  &=-\nabla_{\bm{x}}\cdot\hat{\bm{j}}(\bm{x},t),
\end{align}
where I have defined the stochastic current operator
\begin{align}
  \hat{\bm{j}}(\bm{x},t)\equiv
  &\beta\bm{D}\cdot\bm{F}(\bm{x})\delta(\bm{x}-\bm{r}(t))
  +\beta\bm{D}\bm{v}(\bm{x})\delta(\bm{x}-\bm{r}(t))\nonumber\\
  &+\sqrt{2D_{\alpha\beta}}u_{\beta}(t)
  \delta(\bm{x}-\bm{r}(t))\nonumber\\
  =&\beta\bm{D}\cdot\bm{F}(\bm{x})\hat{\rho}(\bm{x},t)
  +\beta\bm{D}\cdot\bm{v}(\bm{x})\hat{\rho}(\bm{x},t)\nonumber\\
  &+\sqrt{2D_{\alpha\beta}}u_{\beta}(t)\hat{\rho}(\bm{x},t).
\end{align}

You can average $\hat{\rho}(\bm{x},t)$ over the fluctuations $\bm{u}(t)$, and
get the resulting density
\begin{align}
  P(\bm{x},t)=<\hat{\rho}(\bm{x},t)>,
\end{align}
where I have used the definition that for some observable $\hat{\mathcal{O}}$,
\begin{align}
  <\hat{\mathcal{O}}(\bm{u}(t))>=\frac{1}{Z}\int\mathcal{D}\bm{u}
  \hat{\mathcal{O}}(\bm{u}(t))
  \exp\bigg(-\frac{1}{2}\int_0^Tdt(\bm{u}(t))^2\bigg).
\end{align}
From this, I can calculate that noise-averaged current as
\begin{align}
  \bm{J}(\bm{x},t)=
  &\beta\bm{D}\cdot(\bm{F}(\bm{x})P(\bm{x},t)
  +\bm{v}(\bm{x})P(\bm{x},t))\nonumber\\
  &+\sqrt{2D_{\alpha\beta}}<u_{\beta}(t)\hat{\rho}(\bm{x},t)>.
\end{align}

In order to write the noise term above in terms of $P(\bm{x},t)$, I will use
the properties of the Dirac-delta function, namely,
\begin{align}
  \delta(\bm{x})=\int\frac{d^dk}{(2\pi)^d}\exp(-ik_jx_j),
\end{align}
to write
\begin{align}
  <u_{\beta}\hat{\rho}(\bm{x},t)>=\int\frac{d^dk}{(2\pi)^d}u_{\beta}(t)
  \exp(-ik_j(x_j-r_j(t))).
\end{align}

Since correlations between the noise
$<u_{\alpha}(t)u_{\beta}(t)>=\delta_{\alpha\beta}\delta(t-t^{\prime})$ only
occur for fluctuations happening at the same time, we can write
\begin{align}
  \bm{r}(t)=
  &\bm{r}(t-\Delta t)
  +\int_{t-\Delta t}^tdt_1[\beta\bm{D}(\bm{F}(\bm{r}(t_1))
    +\bm{v}(\bm{r}(t_1)))\nonumber\\
    &+\sqrt{2D_{j\gamma}}u_{\gamma}(t_1)]
\end{align}
as $\Delta t\to0$. Then, since
\begin{align}
  \exp\bigg[\int_{t-\Delta t}^tdt_1(\bm{F}(\bm{r}(t_1)
  +\bm{v}({\bm{r}(t_1)})\bigg]=1+O(\Delta t),
\end{align}
this can be written as
\begin{widetext}
  \begin{align}
    <u_{\beta}(t)\exp(-ik_jr_j(t))>=
    &\bigg<u_{\beta}(t)\exp\bigg[-ik_jr_j(t-\Delta t)-ik_j
      \int_{t-\Delta t}^tdt_1(F_j(\bm{r}(t_1)+v_j{\bm{r}(t_1)}
      +\sqrt{2D_{j\gamma}}u_{\gamma}(t_1))\bigg]\bigg>\nonumber\\
    =&\bigg<u_{\beta}(t)\exp(-ik_jr_j(t-\Delta t))
    \bigg[1-ik_j\int_{t-\Delta t}^tdt_1
      \sqrt{2D_{j\gamma}}u_{\gamma}(t_1)\bigg]\bigg>+O(\Delta t).
  \end{align}
\end{widetext}
Next, since $t-\Delta t<t_1,t$, there won't be any correlations between the
fluctuations at earlier times and those at $t$, so using the fact that for uncorrelated
functions $<AB>=<A><B>$, I get
\begin{widetext}
\begin{align}
  <u_{\beta}(t)\exp(-ik_jr_j(t))>
  &=\bigg<\exp(-ik_jr_j(t-\Delta t))\bigg>
  \bigg<u_{\beta}(t)\bigg[1-ik_j\int_{t-\Delta t}^tdt_1
    \sqrt{2D_{j\gamma}}u_{\gamma}(t_1)\bigg]\bigg>
  +O(\Delta t)\nonumber\\
  &=\bigg<\exp(-ik_jr_j(t-\Delta t))\bigg>
  \bigg(<u_{\beta}(t)>
  -\bigg<ik_j\int_{t-\Delta t}^tdt_1
  \sqrt{2D_{j\gamma}}u_{\gamma}(t_1)
  u_{\beta}(t)\bigg>\bigg)
  +O(\Delta t)\nonumber\\
  &=-\bigg<ik_j\sqrt{2D_{j\gamma}}\int_{t-\Delta t}^tdt_1
  \delta_{\beta \gamma}\delta(t-t_1)\bigg>
  \bigg<\exp(-ik_jr_j(t-\Delta t))\bigg>
  +O(\Delta t)\nonumber\\
  &=-\bigg<ik_j\sqrt{2D_{j\beta}}\int_{t-\Delta t}^tdt_1
  \delta(t-t_1)\bigg>
  \bigg<\exp(-ik_jr_j(t-\Delta t))\bigg>
  +O(\Delta t)\nonumber\\
  &=-\bigg<ik_j\sqrt{2D_{j\beta}}\int_0^{\Delta t}du
  \delta(u)\bigg>
  \bigg<\exp(-ik_jr_j(t-\Delta t))\bigg>
  +O(\Delta t)\nonumber\\
  &=-\frac{1}{2}ik_j\sqrt{2D_{j\beta}}
  \bigg<\exp(-ik_jr_j(t-\Delta t))\bigg>
  +O(\Delta t)\nonumber\\.
\end{align}
\end{widetext}
Taking the limit as $\Delta t\to0$, this becomes
\begin{align}
  <u_{\beta}(t)&\exp(-ik_jr_j(t)>=\nonumber\\
  &-\frac{1}{2}ik_j\sqrt{2D_{j\beta}}\bigg<\exp(-ik_jr_j(t))\bigg>.
\end{align}
Taking the inverse Fourier transform then gives
\begin{align}
  <u_{\beta}(t)\hat{\rho}(\bm{x},t)>=-\frac{1}{2}\sqrt{2D_{j\beta}}
  \partial_jP(\bm{x},t).
\end{align}
Finally, inserting this into the current I get
\begin{align}
  J_{\alpha}(\bm{x},t)=
  &\beta D_{\alpha\beta}F_{\beta}(\bm{x})P(\bm{x},t)
  +\beta D_{\alpha\beta}v_{\beta}(\bm{x})P(\bm{x},t)\nonumber\\
  &+\sqrt{2D_{\alpha\beta}}\bigg(-\frac{1}{2}\sqrt{2D_{j\beta}}
  \partial_jP(\bm{x},t)\bigg)\nonumber\\
  =&\beta D_{\alpha\beta}F_{\beta}(\bm{x})P(\bm{x},t)
  +\beta D_{\alpha\beta}v_{\beta}(\bm{x})P(\bm{x},t)\nonumber\\
  &-\sqrt{D_{\alpha\beta}D_{\gamma\beta}}
  \partial_{\gamma}P(\bm{x},t))
\end{align}

Since in our case, $D_{\alpha\beta}$ is diagonal, this just gives eqn
\ref{eq:Nparticlecurrent} as desired.

\section{Hard sphere boundary conditions \label{app:hardsphereBC}}

Consider three interacting spheres, with their positions and orientations as
\begin{subequations}
  \label{eqs:3hardspherepositions}
  \begin{align}
    \bm{r}_1 &= \bm{0},\\
    \bm{r}_2 &= r_0\bm{\hat{e}}_x,\\
    \bm{r}_3 &= \frac{r_0}{2}(\bm{\hat{e}}_x+\sqrt{3}\bm{\hat{e}}_y),
  \end{align}
\end{subequations}
and
\begin{subequations}
  \label{eqs:3hardsphereorientations}
  \begin{align}
    \bm{u}_1 &= \frac{1}{2}(\sqrt{3}\bm{\hat{e}}_x+\bm{\hat{e}}_y),\\
    \bm{u}_2 &= \frac{1}{2}(-\sqrt{3}\bm{\hat{e}}_x+\bm{\hat{e}}_y),\\
    \bm{u}_3 &= -\bm{\hat{e}}_y,
  \end{align}
\end{subequations}
respectively. Then, the differences in positions and orientations are
\begin{subequations}
  \label{eqs:3hardsphererelativepositions}
  \begin{align}
    \bm{r}_{12} &= r_0\bm{\hat{e}}_x-\bm{0} = r_0\bm{\hat{e}}_x,\\
    \bm{r}_{13} &= \frac{r_0}{2}(\bm{\hat{e}}_x+\sqrt{3}\bm{\hat{e}}_y)-\bm{0}
    = \frac{r_0}{2}(\bm{\hat{e}}_x+\sqrt{3}\bm{\hat{e}}_y),\\
    \bm{r}_{23} &= \frac{r_0}{2}(\bm{\hat{e}}_x+\sqrt{3}\bm{\hat{e}}_y)
                  -r_0\bm{\hat{e}}_x
    =\frac{r_0}{2}(-\bm{\hat{e}}_x+\sqrt{3}\bm{\hat{e}}_y),
  \end{align}
\end{subequations}
and
\begin{subequations}
  \label{eqs:3hardsphererelativeorientations}
  \begin{align}
    \bm{u}_{12} &= \frac{1}{2}(-\sqrt{3}\bm{\hat{e}}_x+\bm{\hat{e}}_y)
                  -\frac{1}{2}(\sqrt{3}\bm{\hat{e}}_x+\bm{\hat{e}}_y)
    = -\sqrt{3}\bm{\hat{e}}_x,\\
    \bm{u}_{13} &= -\bm{\hat{e}}_y-\frac{1}{2}(\sqrt{3}\bm{\hat{e}}_x
                  +\bm{\hat{e}}_y)
    = -\sqrt{3}\frac{1}{2}(\bm{\hat{e}}_x+\sqrt{3}\bm{\hat{e}}_y),\\
    \bm{u}_{23} &=  \bm{\hat{e}}_y-\frac{1}{2}(-\sqrt{3}\bm{\hat{e}}_x
                  +\bm{\hat{e}}_y)
    = \sqrt{3}\frac{1}{2}(\bm{\hat{e}}_x-\sqrt{3}\bm{\hat{e}}_y),
  \end{align}
\end{subequations}
respectively. Then, from eqns \ref{eqs:3hardsphererelativepositions}
and \ref{eqs:3hardsphererelativeorientations}, one arrives at the generic
relationships
\begin{align}
  \bm{\hat{r}}_{ij} &= -\frac{\bm{u}_{ij}}{\sqrt{3}},\\
  \bm{u}_{ij}\cdot\bm{r}_{ij} &= -\sqrt{3}r_0.
\end{align}
Inserting these into eqn \ref{eq:3hardspherevelocities}
\begin{align}
  \frac{\bm{V}_i}{\lambda}
  &=\bm{\hat{u}}_i-\sum_{j\neq i}^N \bigg[w'(r_{ij})
    (\bm{u}_{ij}\cdot\bm{r}_{ij})\frac{\bm{r}_{ij}}{r_{ij}}
    +w(r_{ij})\bm{u}_{ij}\bigg]\nonumber\\
  &=\bm{\hat{u}}_i-\sum_{j\neq i}^N \bigg[w'(r_0)
    r_0\bm{u}_{ij}+w(r_0)\bm{u}_{ij}\bigg]
\end{align}
and then taking the difference of two (out of three) velocities
\begin{align}
  \frac{\bm{V}_i}{\lambda}-\frac{\bm{V}_j}{\lambda}
  =&-\bm{u}_{ij}-\sum_{k\neq i}^N \bigg[w'(r_0)
     r_0\bm{u}_{ik}+w(r_0)\bm{u}_{ik}\bigg]\nonumber\\
   &+\sum_{k\neq j}^N \bigg[w'(r_0)
     r_0\bm{u}_{jk}+w(r_0)\bm{u}_{jk}\bigg].
\end{align}
Taking $i=1$ and $j=2$ gives
\begin{align}
  \frac{\bm{V}_1}{\lambda}-\frac{\bm{V}_2}{\lambda}
  =&-\bm{u}_{12}-w'(r_0)r_0\bm{u}_{12}-w(r_0)\bm{u}_{12}
  \nonumber\\
  &-w'(r_0)r_0\bm{u}_{13}-w(r_0)\bm{u}_{13}\nonumber\\
  &+w'(r_0)r_0\bm{u}_{21}+w(r_0)\bm{u}_{21}\nonumber\\
  &+w'(r_0)r_0\bm{u}_{23}+w(r_0)\bm{u}_{23}\nonumber\\
  =&-\bm{u}_{12}-2w'(r_0)r_0\bm{u}_{12}-2w(r_0)\bm{u}_{12}
  \nonumber\\
  &-(w'(r_0)r_0+w(r_0))(\bm{u}_{13}-\bm{u}_{23})\nonumber\\
  =&\bm{u}_{12}\big[-1-3w'(r_0)r_0-3w(r_0)\big].
\end{align}
Setting the lefthand side of the above equation then gives
\ref{eq:hardsphereBC}. The same result occurs for any choices of $i\neq j$
for $N=3$ particles in this trimer configuration. Since we are in
dimensionless variables we can set $r_0=1$.

Inserting the homogeneous solution, eqn \ref{eq:Whomo}, into eqn
\ref{eq:hardsphereBC} gives the coeffiecient of $w(r)$ as
\begin{align}
  c_2 = -\frac{1}{3\sqrt{\Pi}K'(\sqrt{\Pi})}>0.
\end{align}
     
\section{Full solution to w(r). \label{app:fullWsolution}}
The full solution to eqn \ref{eq:WODE_isotropic} can be solved using variation
of parameters. To do this requires finding both the homogeneous solution
as found in eqn \ref{eq:Whomo}, and additionally the particular solution.

\subsection{Wronskian of homogeneous solution.}
To use variation of parameters, the Wronskian of eqn \ref{eq:Whomo} must be
calculated. he linearly independent solutions to the homogeneous equation are
\begin{subequations}
  \label{eqs:linindependentWhomo}
  \begin{align}
    w_1(r)&=\frac{I_1(\sqrt{\Pi}r)}{r},\\
    w_2(r)&=\frac{K_1(\sqrt{\Pi}r)}{r}.
  \end{align}
\end{subequations}
The $n^{th}$ order modified Bessel functions satisfy the ODE
\begin{align}
  y'=\frac{(x^2+n^2)y-x^2y''}{x}
\end{align}
and so the Wronskian of the modified Bessel functions satisfy the ODE
\begin{align}
  s_n(x)=
  & I_n(x)K_n'(x)-K_n(x)I_n'(x)\nonumber\\
  =& I_n(x)\frac{(x^2+n^2)K_n(x)-x^2K_n''(x)}{x}\nonumber\\
  & -K_n(x)\frac{(x^2+n^2)I_n(x)-x^2I_n''(x)}{x}\nonumber\\
  =& -x(I_n(x)K_n''(x)-K_n(x)I_n''(x))\nonumber\\
  =& - x s_n'(x),
\end{align}
which has the solution
\begin{align}
  s_n(x)=-\frac{1}{x}
\end{align}
due to the asymptotics of $I_1(x)$ and $K_1(x)$, namely
\begin{align}
  I_n(z)&\sim\frac{e^z}{(2\pi z)^{\frac{1}{2}}},\\
  K_n(z)&\sim\bigg(\frac{\pi}{(2 z)}\bigg)^{\frac{1}{2}}e^{-z},\\
  I_n(z)&\sim\frac{e^z}{(2\pi z)^{\frac{1}{2}}},\\
  K_n(z)&\sim-\bigg(\frac{\pi}{(2 z)}\bigg)^{\frac{1}{2}}e^{-z}.
\end{align}
Therefore, the Wronskian of eqns \ref{eqs:linindependentWhomo} is
\begin{align}\label{eq:wronskian}
  W(r)=
  & \frac{I_1(\sqrt{\Pi}r)}{r}\bigg[-\frac{K_1(\sqrt{\Pi}r)}{r^2}
    +\sqrt{\Pi}\frac{K_1'(\sqrt{\Pi}r)}{r}\bigg]\nonumber\\
  &-\frac{K_1(\sqrt{\Pi}r)}{r}\bigg[-\frac{I_1(\sqrt{\Pi}r)}{r^2}
    +\sqrt{\Pi}\frac{I_1'(\sqrt{\Pi}r)}{r}\bigg]\nonumber\\
  =& \frac{\sqrt{\Pi}}{r^2}\bigg[I_1(\sqrt{\Pi}r)K_1'(\sqrt{\Pi}r)
     - K_1(\sqrt{\Pi}r)I_1'(\sqrt{\Pi}r)\bigg]\nonumber\\
  =& \frac{\sqrt{\Pi}}{r^2}\bigg(-\frac{1}{\sqrt{\Pi}r}\bigg)\nonumber\\
  =& -\frac{1}{r^3}.
\end{align}


\section{Partition function of ABPs. \label{app:statmech}}
The partition function of the steady state can be written as
\begin{widetext}
\begin{align}
  Z =&\int\prod_{i=1}^N\big(d^2r_id\theta_i)
       \times\exp\bigg(-\sum_{i=1}^N\sum_{j=1}^{i-1}\big[
       \mathcal{V}(r_{ij})+\lambda w(r_{ij})\bm{r}_{ij}
       \cdot\bm{u}_{ij}\big]\bigg)\nonumber\\
  =&\int\prod_{i=1}^N\big(d^2r_id\theta_i)
     \times\exp\bigg(-\frac{1}{2}\sum_{i,j;j\neq i}\big[
     \mathcal{V}(r_{ij})+\lambda w(r_{ij})\bm{r}_{ij}
     \cdot\bm{u}_{ij}\big]\bigg)\nonumber\\
  =&\int\prod_{i=1}^N\big(d^2r_id\theta_i)
     \exp\bigg(-\frac{1}{2}\sum_{i=1}^N\sum_{j\neq i}^N
     \mathcal{V}(r_{ij})
     +\lambda\sum_{i=1}^N\sum_{j\neq i}^N\bm{u}_{i}
     \cdot w(r_{ij})\bm{r}_{ij}\bigg).
\end{align}
\end{widetext}
where we have used eqn \ref{eq:u_itou_ij}. With the definition of eqn
\ref{eq:b_i}, this is re-written as
\begin{widetext}
\begin{align}
  Z=&\int\prod_{i=1}^N\big(d^2r_id\theta_i)
     \exp\bigg(-\frac{1}{2}\sum_{i=1}^N\sum_{j\neq i}^N
     \mathcal{V}(r_{ij})+2\lambda\sum_{i=1}^N\bm{u}_{i}
     \cdot \bm{b}_{i}\bigg)\nonumber\\
  =&\int\prod_{i=1}^N\big(d^2r_i)
     \exp\bigg(-\frac{1}{2}\sum_{i=1}^N\sum_{j\neq i}^N
     \mathcal{V}(r_{ij})\bigg)\int\prod_{i=1}^N\bigg[d\theta_i
     \exp(\lambda\bm{u}_{i}\cdot \bm{b}_{i})\bigg].
\end{align}
\end{widetext}
The angular integrals can be re-written as
\begin{widetext}
\begin{align}
  \int_0^{2\pi} d\theta_u\exp(\bm{u}\cdot\lambda\bm{b})
  &=\int_0^{2\pi} d\theta_u\exp(\lambda|\bm{b}|\cos(\theta_b-\theta_u))
    \nonumber\\
  &=\int_{-\theta_b}^{2\pi-\theta_b} d\phi\exp(\lambda|\bm{b}|\cos(\phi))
    \nonumber\\
  &=\bigg(\int_{-\theta_b}^0
    +\int_0^{2\pi}
    -\int_{2\pi-\theta_b}^{2\pi}\bigg)
    d\phi\exp(\lambda|\bm{b}|\cos(\phi))
    \nonumber\\
  &=\bigg(\int_{-\theta_b}^0
    -\int_{2\pi-\theta_b}^{2\pi}\bigg)
    d\phi\exp(\lambda|\bm{b}|\cos(\phi))
    +\int_0^{2\pi}d\phi\exp(|\bm{b}|\cos(\phi))
    \nonumber\\
    &=\int_{-\theta_b}^0d\phi\exp(\lambda|\bm{b}|\cos(\phi))
    -\int_{2\pi-\theta_b}^{2\pi}d\phi'\exp(\lambda|\bm{b}|\cos(\phi'))
    +\int_0^{2\pi}d\phi\exp(\lambda|\bm{b}|\cos(\phi))
    \nonumber\\
  &=\int_{-\theta_b}^0d\phi\exp(\lambda|\bm{b}|\cos(\phi))
    +\int_{\theta_b}^{0}d\phi''\exp(\lambda|\bm{b}|\cos(2\pi-\phi''))
    +\int_0^{2\pi}d\phi\exp(\lambda|\bm{b}|\cos(\phi))
    \nonumber\\
  &=\int_{-\theta_b}^0d\phi\exp(\lambda|\bm{b}|\cos(\phi))
    -\int_{-\theta_b}^{0}d\phi\exp(\lambda|\bm{b}|\cos(\phi))
    +\int_0^{2\pi}d\phi\exp(\lambda|\bm{b}|\cos(\phi))
    \nonumber\\
  &=\int_0^{2\pi}d\phi\exp(\lambda|\bm{b}|\cos(\phi))
\end{align}
\end{widetext}
which is $I_0(\lambda|b|)$ using the definition of eqn \ref{eq:Iintegral}.
Inserting these results back into the partition function gives eqn
\ref{eq:partitionfunction}.
\bibliographystyle{plain}
\bibliography{main}

\end{document}
