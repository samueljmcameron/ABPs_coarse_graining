\documentclass{article}
\usepackage[hmargin=1.25cm,vmargin=2.5cm]{geometry}
\usepackage[utf8]{inputenc}
\usepackage{bm}
\usepackage{amsmath}
\usepackage{amssymb}
\usepackage{comment}
\usepackage{graphicx}

\title{calculation: coarse-graining active brownian particles}
\author{Sam Cameron}
\date{\today}

\begin{document}

\maketitle

\section{Writing the active brownian particles equations of motion into the correct form}

The equations of motion for active brownian particles are:

\begin{align}
  \frac{d\bm{r}_i}{dt}&=\frac{D^t}{k_BT}\big(\bm{F}_i
                        +f^P\bm{P}_i\big)
                        +\sqrt{2D^t}\tilde{\bm{u}}_i^t,\label{eq:micro_pos}\\
    \frac{d\theta_i}{dt}&=\sqrt{2D^r}w_i.\label{eq:micro_theta}
\end{align}
where $\bm{r}_i=(x_i,y_i)$ are the particle positions,
$\bm{P}_i=(\cos\theta_i,\sin\theta_i)$ are the particle orientations,
and $\tilde{\bm{u}}_i=(\tilde{u}_i,v_i)$, $w_i$ are Gaussian white noise
with zero mean and unit variance, i.e. for any two components of the vector
$\bm{s}=(u_i,v_i,w_i)$,
\begin{align}
  \big<s_i(t)\big>=0;\;\;\;\;\; \big<s_i(t)s_j(t^{\prime})\big>=
  \delta_{ij}\delta(t-t^{\prime}).
\end{align}.

Now, I can define a vector of length $3N$ with components
\begin{align}
  \bm{r}=(\bm{r}^{(1)},\bm{r}^{(2)},\cdots,\bm{r}^{(N)},\theta_1,\theta_2,\cdots,
  \theta_N).
\end{align}

This will then give the differential equation
\begin{align}
  \frac{d\bm{r}}{dt}=\bm{F}(\bm{r})+\bm{v}(\bm{r})
  +\sqrt{2\tilde{D}_{\alpha\beta}}u_{\beta}(t),
\end{align}
where $\tilde{D}_{\alpha\beta}$ is diagonal, and the values of the diagonal are
$D^t$ for the first $2N$ terms, and $D^r$ for the last $N$ terms.

Next, I will define
\begin{align}
  \bm{x}&\equiv(x_1,x_2,\cdots,x_{3N}),\\
  \nabla&\equiv(\partial_1,\partial_2,\cdots,\partial_{3N}).
\end{align}

as well as the stochastic probability density operator
\begin{align}
  \hat{\rho}(\bm{x},t)=\delta(\bm{x}-\bm{r}(t)).
\end{align}

Taking the partial derivative of this with respect to time gives
\begin{align}
  \frac{\partial\hat{\rho}}{\partial t}&=-(\nabla_{\bm{x}}\delta(\bm{x}-\bm{r}(t))
                                         \cdot\frac{d\bm{r}}{dt}\nonumber\\
                                       &=-\nabla_{\bm{x}}\cdot\bigg(\frac{d\bm{r}}{dt}
                                         \delta(\bm{x}-\bm{r}(t)\bigg)\nonumber\\
                                       &=-\nabla_{\bm{x}}\cdot\hat{\bm{j}}(\bm{x},t),
\end{align}
where I have defined the stochastic current operator
\begin{align}
  \hat{\bm{j}}(\bm{x},t)&\equiv\bm{F}(\bm{x})\delta(\bm{x}-\bm{r}(t))
                          +\bm{v}(\bm{x})\delta(\bm{x}-\bm{r}(t))
                          +\sqrt{2\tilde{D}_{\alpha\beta}}u_{\beta}(t)
                          \delta(\bm{x}-\bm{r}(t))\nonumber\\
                        &=\bm{F}(\bm{x})\hat{\rho}(\bm{x},t)
                          +\bm{v}(\bm{x})\hat{\rho}(\bm{x},t)
                          +\sqrt{2\tilde{D}_{\alpha\beta}}u_{\beta}(t)\hat{\rho}(\bm{x},t).
\end{align}

You can average $\hat{\rho}(\bm{x},t)$ over the fluctuations $\bm{u}(t)$, and get the
resulting density
\begin{align}
  P(\bm{x},t)=<\hat{\rho}(\bm{x},t)>,
\end{align}
where I have used the definition that for some observable $\hat{\mathcal{O}}$,
\begin{align}
  <\hat{\mathcal{O}}(\bm{u}(t))>=\frac{1}{Z}\int\mathcal{D}\bm{u}
  \hat{\mathcal{O}}(\bm{u}(t))\exp\bigg(-\frac{1}{2}\int_0^Tdt(\bm{u}(t))^2\bigg).
\end{align}
From this, I can calculate that noise-averaged current as
\begin{align}
  \bm{J}(\bm{x},t)=\bm{F}(\bm{x})P(\bm{x},t)+\bm{v}(\bm{x})P(\bm{x},t)
  +\sqrt{2\tilde{D}_{\alpha\beta}}<u_{\beta}(t)\hat{\rho}(\bm{x},t)>.
\end{align}

In order to write the noise term above in terms of $P(\bm{x},t)$, I will use the properties
of the Dirac-delta function, namely,
\begin{align}
  \delta(\bm{x})=\int\frac{d^dk}{(2\pi)^d}\exp(-ik_jx_j),
\end{align}
to write
\begin{align}
  <u_{\beta}\hat{\rho}(\bm{x},t)>=\int\frac{d^dk}{(2\pi)^d}u_{\beta}(t)
  \exp(-ik_j(x_j-r_j(t))).
\end{align}

Since correlations between the noise
$<u_{\alpha}(t)u_{\beta}(t)>=\delta_{\alpha\beta}\delta(t-t^{\prime})$ only occur for
fluctuations happening at the same time, we can write
\begin{align}
  \bm{r}(t)=\bm{r}(t-\Delta t)+\int_{t-\Delta t}^tdt_1[\bm{F}(\bm{r}(t_1))
  +\bm{v}{\bm{r}(t_1)}+\sqrt{2\tilde{D}_{j\gamma}}u_{\gamma}(t_1)]
\end{align}
as $\Delta t\to0$. Then, since
\begin{align}
  \exp\bigg[\int_{t-\Delta t}^tdt_1(\bm{F}(\bm{r}(t_1)
  +\bm{v}({\bm{r}(t_1)})\bigg]=1+O(\Delta t),
\end{align}
this can be written as
\begin{align}
  <u_{\beta}(t)\exp(-ik_jr_j(t))>=&\bigg<u_{\beta}(t)\exp\bigg[-ik_jr_j(t-\Delta t)-ik_j
                                    \int_{t-\Delta t}^tdt_1(F_j(\bm{r}(t_1)
                                    +v_j{\bm{r}(t_1)}\nonumber\\
                                  &+\sqrt{2\tilde{D}_{j\gamma}}u_{\gamma}
                                    (t_1))\bigg]\bigg>\nonumber\\
  =&\bigg<u_{\beta}(t)\exp(-ik_jr_j(t-\Delta t))
     \bigg[1-ik_j\int_{t-\Delta t}^tdt_1
     \sqrt{2\tilde{D}_{j\gamma}}u_{\gamma}(t_1)\bigg]\bigg>+O(\Delta t).
\end{align}

Next, since $t-\Delta t<t_1,t$, there won't be any correlations between the
fluctuations at earlier times and those at $t$, so using the fact that for uncorrelated
functions $<AB>=<A><B>$, I get
\begin{align}
  <u_{\beta}(t)\exp(-ik_jr_j(t))>&=\bigg<\exp(-ik_jr_j(t-\Delta t))\bigg>
                                   \bigg<u_{\beta}(t)\bigg[1-ik_j\int_{t-\Delta t}^tdt_1
                                   \sqrt{2\tilde{D}_{j\gamma}}u_{\gamma}(t_1)\bigg]\bigg>
                                   +O(\Delta t)\nonumber\\
                                 &=\bigg<\exp(-ik_jr_j(t-\Delta t))\bigg>
                                   \bigg(<u_{\beta}(t)>
                                   -\bigg<ik_j\int_{t-\Delta t}^tdt_1
                                   \sqrt{2\tilde{D}_{j\gamma}}u_{\gamma}(t_1)
                                   u_{\beta}(t)\bigg>\bigg)
                                   +O(\Delta t)\nonumber\\
                                 &=-\bigg<ik_j\sqrt{2\tilde{D}_{j\gamma}}\int_{t-\Delta t}^tdt_1
                                   \delta_{\beta \gamma}\delta(t-t_1)\bigg>
                                   \bigg<\exp(-ik_jr_j(t-\Delta t))\bigg>
                                   +O(\Delta t)\nonumber\\
                                 &=-\bigg<ik_j\sqrt{2\tilde{D}_{j\beta}}\int_{t-\Delta t}^tdt_1
                                   \delta(t-t_1)\bigg>
                                   \bigg<\exp(-ik_jr_j(t-\Delta t))\bigg>
                                   +O(\Delta t)\nonumber\\
                                 &=-\bigg<ik_j\sqrt{2\tilde{D}_{j\beta}}\int_0^{\Delta t}du
                                   \delta(u)\bigg>
                                   \bigg<\exp(-ik_jr_j(t-\Delta t))\bigg>
                                   +O(\Delta t)\nonumber\\
                                 &=-\frac{1}{2}ik_j\sqrt{2\tilde{D}_{j\beta}}
                                   \bigg<\exp(-ik_jr_j(t-\Delta t))\bigg>
                                   +O(\Delta t)\nonumber\\.
\end{align}
Taking the limit as $\Delta t\to0$, this becomes
\begin{align}
  <u_{\beta}(t)\exp(-ik_jr_j(t)>=-\frac{1}{2}ik_j\sqrt{2\tilde{D}_{j\beta}}
  \bigg<\exp(-ik_jr_j(t))\bigg>.
\end{align}
Taking the inverse Fourier transform then gives
\begin{align}
  <u_{\beta}(t)\hat{\rho}(\bm{x},t)>=-\frac{1}{2}\sqrt{2\tilde{D}_{j\beta}}
  \partial_jP(\bm{x},t).
\end{align}
Finally, inserting this into the current I get
\begin{align}
  J_{\alpha}(\bm{x},t)&=F_{\alpha}(\bm{x})P(\bm{x},t)+v_{\alpha}(\bm{x})P(\bm{x},t)
                        +\sqrt{2\tilde{D}_{\alpha\beta}}(-\frac{1}{2}\sqrt{2\tilde{D}_{j\beta}}
                        \partial_jP(\bm{x},t))\nonumber\\
                      &=F_{\alpha}(\bm{x})P(\bm{x},t)+v_{\alpha}(\bm{x})P(\bm{x},t)
                        -\sqrt{\tilde{D}_{\alpha\beta}\tilde{D}_{j\beta}}
                        \partial_jP(\bm{x},t))
\end{align}

Since in our case, $\tilde{D}_{\alpha\beta}$ is diagonal, this just becomes
\begin{align}\label{eq:Nparticlecurrent}
  \bm{J}(\bm{x},t)&=\bm{F}(\bm{x})P(\bm{x},t)+\bm{v}(\bm{x})P(\bm{x},t)
                    -\bm{\tilde{D}}\cdot\nabla P(\bm{x},t).
\end{align}

Finally, I get the desired result of the section, which is
\begin{align}\label{eq:NparticleFP}
  \frac{\partial P}{\partial t} + \nabla_{\bm{x}}\cdot\big(\bm{F}(\bm{x})P(\bm{x},t)
  +\bm{v}(\bm{x})P(\bm{x},t)-\bm{\tilde{D}}\cdot\nabla P(\bm{x},t)\big)=0
\end{align}


\subsection{Algebraic conditions on steady states.}

In this section, I am interested in determining the one particle density function, i.e. in
terms of $\bm{r}_1$ (not the angular components). To start with, I will remind myself that
\begin{align}
  P(\bm{x},t)\equiv P(\bm{r}_1,\bm{r}_2,\cdots,\bm{r}_N,\theta_1,\theta_2,\cdots,\theta_N,t)
\end{align}

I will then split the the current of eqn \ref{eq:Nparticlecurrent} into two parts, a part which
cannot be written as a gradient term, and a part that can. The part that cannot be written as
a gradient term is just
\begin{align}\label{eq:dissipative_force}
  \bm{v}(\bm{x})&=\frac{D^t}{k_BT}f^P(\cos\theta_1,\sin\theta_1,\cos\theta_2,
                  \sin\theta_2,\cdots,\cos\theta_N,\sin\theta_N,0,0,\dots,0)^T\nonumber\\
                &=\frac{1}{k_BT}\bm{\tilde{D}}\cdot \bm{u}(\{\theta_i\})
\end{align}

Before writing out the gradient term, I will first note that I am considering just two-body
interactions,
\begin{align}\label{eq:general_conservative_force}
  \bm{F}(\bm{x})&=-\nabla_{\bm{x}}\mathcal{H}(\bm{x})\nonumber\\
                &=-\nabla_{\bm{x}}\sum_{i<j}^N\mathcal{V}(r_{ij}),
\end{align}
where $r_{ij}=|\bm{r}_i-\bm{r}_j|$.

Then, I can re-write eqn \ref{eq:Nparticlecurrent} as
\begin{align}\label{eq:Nparticlewithpotential}
  \bm{J}(\bm{x},t)&=-\bm{\tilde{D}}
                    \cdot\big(\nabla_{\bm{x}}(\frac{1}{k_BT}\mathcal{H}(\{\bm{r}_i\})
                    P(\bm{x},t)
                    +P(\bm{x},t))
                    -\frac{1}{k_BT}\bm{u}(\{\theta_i\})P(\bm{x},t)\big).
\end{align}

I will now redefine
\begin{align}
  \bm{D}=\beta\bm{\tilde{D}},
\end{align}
where $\beta^{-1}=k_BT$.


Finally, I can write the Fokker-Planck equation as
\begin{align}
  \frac{\partial P}{\partial t}=\nabla_{\bm{x}}\cdot\big(\bm{D}
  \big[\beta^{-1}\nabla_{\bm{x}}P(\bm{x},t)+P(\bm{x},t)(\nabla_{\bm{x}}
  \mathcal{H}(\{\bm{r}_i\})-\bm{v}(\{\theta_i\}))\big]\big)
\end{align}

From \cite{liverpool2018nonequilibrium}, I then know that there exists a generalized steady
state of the form
\begin{align}
  P_{ss}=\exp(-h(\bm{x}))
\end{align}
where $h(\bm{x})$ satisfies the equation
\begin{align}\label{eq:ss_condition}
  \sum_{i=1}^{3N}D_iL_i(h)=\bm{0},
\end{align}
where $D_i$ are the (diagonal) components of $\bm{D}$,
\begin{align}
  D_i =
  \begin{cases}
    D^t\beta,&i\leq 2N,\\
    D^r\beta,&i>2N
  \end{cases}
\end{align}
and
\begin{align}\label{eq:L_i}
  L_i(h) = \beta^{-1}(\partial_ih)^2+\partial_i^2\mathcal{H}
  +(v_i-\partial_i\mathcal{H})\partial_ih-\beta^{-1}\partial_i^2h-\partial_iv_i.
\end{align}

I can explicitly calculate the terms not involving $h$ in eqn \ref{eq:L_i}, using eqns
\ref{eq:general_conservative_force} and eqn \ref{eq:dissipative_force}. I do this now:
\begin{subequations}
  \label{eqs:partials_of_L_i}
  \begin{align}
    \nabla_{\alpha}\mathcal{H}
    &=\sum_{i\neq\alpha}^N\mathcal{V}^{\prime}
      (r_{\alpha i})\frac{\bm{r}_{\alpha i}}{r_{\alpha i}},
      \:\:\:\alpha=1,2,...,N\label{eq:partial_H_of_L_i}\\
    \nabla_{\alpha}^2\mathcal{H}
    &=\sum_{i\neq\alpha}^N\bigg[\frac{\mathcal{V}^{\prime}(r_{\alpha i})}{r_{\alpha i}}
      +\mathcal{V}^{\prime\prime}(r_{\alpha i})\bigg],
      \:\:\:\alpha=1,2,...,N\label{eq:partialsq_H_of_L_i}\\
    \partial_i v_i
    &= 0\label{eq:partial_v_of_L_i}
  \end{align}
\end{subequations}

From these results I can see immediately that eqn \ref{eq:ss_condition} simplifies to
\begin{align}\label{eq:steadystatecondition}
  D^t\beta\sum_{i=1}^{2N}L_i(h)+D^r\sum_{i=1}^{N}\big[(\partial_{\theta_i}h)^2
  -\partial_{\theta_i}^2h\big]=0
\end{align}

To actually do any calculations, an interaction form for $\mathcal{H}$ must be chosen. Since we
are interested in capturing motility induced phase separation (MIPS) in the simplest form, I
will choose the (repulsive) Weeks-Chandler-Anderson (WCA) interaction,
\begin{equation}\label{eq:WCAapprox}
  \mathcal{V}_{ij} =
  \begin{cases}
    4\epsilon\bigg[\bigg(\frac{\sigma}{r^{ij}}\bigg)^{12}
    -\bigg(\frac{\sigma}{r^{ij}}\bigg)^6\bigg]+\epsilon, & r<2^{\frac{1}{6}}\sigma \\
    0, & \mathrm{otherwise}.
  \end{cases}
\end{equation}

\newpage
\section{Coarse-graining the steady $h(\bm{x})$ function.}

In this section, I will try several variational guesses to approximate the steady-state $h$
function.

Generically, we expect that the microscopic $h$ function will be invariant under translations,
so
\begin{align}
  h(\{\bm{r}_i+\bm{T}\},\{\theta_i\}) = h(\{\bm{r}_i\},\{\theta_i\}).
\end{align}
Further to this invariance, the system must also be invariant
under full rotations, and so
\begin{align}
  h(\{\bm{Q}(\phi)\bm{r}_i\},\{\theta_i+\phi\})
  =h(\{\bm{r}_i\},\{\theta_i\}).
\end{align}

And finally, interchanging particles should not vary the results, so swapping $\bm{r}_i,\theta_i$
with $\bm{r}_j,\theta_j$ should leave $h$ invariant.

The only way to generically have all of these features is to have
\begin{align}
  h(\{|\bm{r}_i-\bm{r}_j|\},\{|\theta_i-\theta_j|\},
  \{(\bm{r}_i-\bm{r}_j)\cdot(\bm{u}_i-\bm{u}_j)\})
\end{align}
where $\bm{u}_i = (\cos\theta_i,\sin\theta_i)$.

\subsection{Perturbative guess for coupled $h$ function.}

Before doing this calculation, I will write everything in dimensionless form, so
\begin{align}\label{eq:dim_ss}
  \sum_{i=1}^N\big[\mathcal{L}_i^{(0)}(h)+ \frac{f^P\sigma}{\beta^{-1}}
  \mathcal{L}_i^{(1)}(h)\big] = 0
\end{align}
where
\begin{align}
  L_i^{(0)}(h)=(\nabla_i h)^2 + \nabla_i^2\mathcal{H}
  -\nabla_i\mathcal{H}\cdot\nabla_ih-\nabla_i^2h +\frac{D^r\sigma^2}{D^t}
  \big[(\partial_{\theta_i}h)^2-\partial_{\theta_i}^2h\big],
\end{align}
and
\begin{align}
  L_i^{(1)}(h)=\bm{u}_i\cdot\nabla_ih.
\end{align}
Here, $\sigma$ is some microscopic length scale defined the Hamiltonian
$\mathcal{H}$, and everything is dimensionless. When $f^P=0$, the solution
is just $h=\mathcal{H}$ as expected.

I will take a perturbative guess to the solution, with $\epsilon=f^P\sigma\beta$
as my perturbation parameter. So, my guess will be
\begin{align}\label{eq:hperturb}
  h(\epsilon) = \mathcal{H}
  + \epsilon\frac{\partial h}{\partial \epsilon}\bigg|_{\epsilon=0}.
\end{align}

Differentiating eqn \ref{eq:dim_ss} with respect to $\epsilon$ gives
\begin{align}\label{eq:first_perturb_DE}
  0
  &=\sum_{i=1}^N\big\{\partial_{\epsilon}\mathcal{L}_i^{(0)}(h)
    + \mathcal{L}_i^{(1)}(h(0))
    +\epsilon\partial_{\epsilon}\mathcal{L}_i^{(1)}
    (h)\big\}\bigg|_{\epsilon=0}\nonumber\\
  &=\sum_{i=1}^N\big\{2\nabla_i \mathcal{H}\cdot\nabla_i\zeta
    - \nabla_i\mathcal{H}\cdot\nabla_i\zeta
    -\nabla_i^2\zeta+\frac{D^r\sigma^2}{D^t}
    \big[2\partial_{\theta_i}\mathcal{H}\partial_{\theta_i}\zeta
    -\partial_{\theta_i}^2\zeta\big]
    + \bm{u}_i\cdot\nabla_i\mathcal{H}
    +(0)\bm{u}_i\cdot\nabla_i\zeta\big\}\nonumber\\
  &=\sum_{i=1}^N\big[\nabla_i \mathcal{H}\cdot\nabla_i\zeta
    -\nabla_i^2\zeta - \frac{D^r\sigma^2}{D^t}
    \partial_{\theta_i}^2\zeta
    + \bm{u}_i\cdot\nabla_i\mathcal{H}\big]
\end{align}
where $\zeta\equiv\partial_{\epsilon}h|_{\epsilon=0}$.

\subsection{Indistinguishable particles.}

I will take a guess for the solution of the form,
\begin{align}\label{eq:zetaform}
  \zeta &= -\sum_{i=1}^N\sum_{j=1}^{i-1}
      W(r_{ij})\bm{r}_{ij}\cdot\bm{u}_{ij}.
\end{align}

Since, for $i\neq j$,
\begin{align}
  \nabla_i(\bm{r}_{ij}\cdot\bm{u}_{ij}) &= \bm{u}_{ij},\\
  \nabla_j(\bm{r}_{ij}\cdot\bm{u}_{ij}) &= -\bm{u}_{ij},\\
  \nabla_i r_{ij} &=\frac{\bm{r}_{ij}}{r_{ij}},\\
  \nabla_j r_{ij} &=-\frac{\bm{r}_{ij}}{r_{ij}},\\
  \partial_{\theta_i}^2(\bm{u}_{ij}\cdot\bm{r}_{ij})&=-\bm{u}_i\cdot\bm{r}_{ij},\\
  \partial_{\theta_i}^2(\bm{u}_{ji}\cdot\bm{r}_{ji})&=\bm{u}_i\cdot\bm{r}_{ji},
\end{align}
the terms are
\begin{align}
  \nabla_{i}\zeta = -\sum_{j\neq i}^N \bigg[W'(r_{ij})
  (\bm{u}_{ij}\cdot\bm{r}_{ij})\frac{\bm{r}_{ij}}{r_{ij}}
  +W(r_{ij})\bm{u}_{ij}\bigg],
\end{align}

\begin{align}
  \nabla_i^2\zeta=-\sum_{j\neq i}^N\bigg[W''(r_{ij})
  +3\frac{W'(r_{ij})}{r_{ij}}\bigg](\bm{u}_{ij}\cdot\bm{r}_{ij}).
\end{align}

\begin{align}
  \partial_{\theta_i}^2\zeta=\sum_{j\neq i}W(r_{ij})\bm{u}_i\cdot\bm{r}_{ij}
\end{align}

Inserting these into eqn \ref{eq:first_perturb_DE}
\begin{align}
  0=
  &-\sum_{i=1}^N\sum_{j\neq i}^N\sum_{k\neq i}^N
    \mathcal{V}^{\prime}(r_{ij})\frac{\bm{r}_{ij}}{r_{ij}}
    \cdot\bigg[W'(r_{ik})
    (\bm{u}_{ik}\cdot\bm{r}_{ik})\frac{\bm{r}_{ik}}{r_{ik}}
    +W(r_{ik})\bm{u}_{ik}\bigg]
    +\sum_{i=1}^N\sum_{j\neq i}^N\bigg[W''(r_{ij})
    +3\frac{W'(r_{ij})}{r_{ij}}\bigg](\bm{u}_{ij}\cdot\bm{r}_{ij})\nonumber\\
  &-\frac{D^r\sigma^2}{D^t}\sum_{i=1}^N\sum_{j\neq i}^NW(r_{ij})\bm{u}_i\cdot\bm{r}_{ij}
   +\sum_{i=1}^N\sum_{j\neq i}^N 
    \mathcal{V}^{\prime}(r_{ij})\frac{\bm{u}_i\cdot\bm{r}_{ij}}{r_{ij}}
\end{align}

You can factor this equation a bit, using the result that
\begin{align}
  \sum_{i=1}^N\sum_{j\neq i}f(r_{ij}) \bm{u}_i\cdot\bm{r}_{ij}
  =\frac{1}{2}\sum_{i=1}^N\sum_{j\neq i}f(r_{ji})\bm{u}_{ij}\cdot\bm{r}_{ij}
\end{align}
  to get
\begin{align}\label{eq:WODE_exact}
  0&=
     \sum_{i=1}^N\sum_{j\neq i}^N\bm{r}_{ij}\cdot\bigg\{\sum_{k\neq i}^N\bigg[
     -\frac{\mathcal{V}^{\prime}(r_{ik})}{r_{ik}}
     W'(r_{ij})
     \cdot\frac{\bm{r}_{ik}\cdot\bm{r}_{ij}}{r_{ij}}\bm{u}_{ij}
     -\frac{\mathcal{V}^{\prime}(r_{ik})}{r_{ik}}W(r_{ik})\bm{u}_{ik}\bigg]
     +\bigg[W''(r_{ij})
     +3\frac{W'(r_{ij})}{r_{ij}}\bigg]\bm{u}_{ij}\nonumber\\
   &\hphantom{=}-\frac{D^r\sigma^2}{2D^t}W(r_{ij})\bm{u}_{ij}
     +\mathcal{V}^{\prime}(r_{ij})\frac{\bm{u}_{ij}}{2r_{ij}}\bigg\}\nonumber\\
   &=\sum_{i=1}^N\sum_{j\neq i}^N\bm{r}_{ij}\cdot\bm{u}_{ij}
     \bigg\{\sum_{k\neq i}^N\bigg[
     -\frac{\mathcal{V}^{\prime}(r_{ik})}{r_{ik}}
     W'(r_{ij})\frac{\bm{r}_{ik}\cdot\bm{r}_{ij}}{r_{ij}}\bigg]
     +W''(r_{ij})+3\frac{W'(r_{ij})}{r_{ij}}\nonumber\\
   &\hphantom{=}-\frac{D^r\sigma^2}{2D^t}W(r_{ij})
     +\frac{\mathcal{V}^{\prime}(r_{ij})}{2r_{ij}}\bigg\}
     -\sum_{i=1}^N\sum_{j\neq i}^N\bm{r}_{ij}\cdot\sum_{k\neq i}^N
     \bm{u}_{ik}\frac{\mathcal{V}^{\prime}(r_{ik})}{r_{ik}}W(r_{ik})
\end{align}

So far, the approximations we have made in calculating $h$ are that for small activity,
$h$ is perturbatively close to $\mathcal{H}$, and we have taken a lowest order guess in
what this perturbative term is, in terms of some arbitrary function $W(r_{ij})$. Now, to
be able to get some sort of solution for eqn \ref{eq:WODE_exact}, we need to make some
further approximation. The way forward is to assume that the system has translational and
rotational invariance in its pairwise-correlation function.

The standard definition of the pairwise-correlation operator is
\begin{align}
  <n(\bm{x}_1)>\hat{g}_2(\bm{x}_1,\bm{x}_2)<(\bm{x}_2)>
  &=\sum_{i=1}^N\sum_{j\neq i}^N\delta(\bm{x}_1-\bm{r}_i)
    \delta(\bm{x}_2-\bm{r}_j).
\end{align}

If the system is translationally invariant, then $\bm{r}=\bm{x}_1-\bm{x}_2$ and
\begin{align}
  \hat{g}_2(\bm{x}_1,\bm{x}_2)
  &=\hat{g}_2(\bm{r})\nonumber\\
  &=\frac{1}{n^2}\sum_{i=1}^N\sum_{j\neq i}^N\delta(\bm{r}+\bm{x}_2-\bm{r}_i)
    \delta(\bm{x}_2-\bm{r}_j)\nonumber\\
  &=\frac{1}{n^2}\frac{1}{V}\int d^2x_2 \sum_{i=1}^N\sum_{j\neq i}^N
    \delta(\bm{r}+\bm{x}_2-\bm{r}_i)\delta(\bm{x}_2-\bm{r}_j)\nonumber\\
  &=\frac{V}{N^2}\sum_{i=1}^N\sum_{j\neq i}\delta(\bm{r}-\bm{r}_{ij})\nonumber\\
  &=\frac{V}{N}\sum_{j=1,j\neq i}^N\delta(\bm{r}-\bm{r}_{ij})\nonumber\\
\end{align}
where in the last line I have used the fact that the sum over $j$ runs over all possible
values of the difference $\bm{r}_i-\bm{r}_j$, and so each of the $N-1\approx N$ terms are
identical.

Then, any term of the form
\begin{align}
  \sum_{j\neq i} \bm{f}(\bm{r}_{ij})
  &= \sum_{j\neq i}\int \delta(\bm{r}+\bm{r}_{ij})\bm{f}(-\bm{r}) d^2r\nonumber\\
  &=\frac{N}{V}\int\hat{g}_2(\bm{r})\bm{f}(-\bm{r})d^2r.
\end{align}

Now, for $g_2(r)=<\hat{g}_2(\bm{r})>$ is the average pair-correlation function, and is
radially symmetric, any integrals over functions of the form $\bm{\hat{e}}_r f(r)$ will
be zero by symmetry. With this in mind, I can see from both three-sum terms in eqn
\ref{eq:WODE_exact} that both have vector functions which are spherically symmetric, the
first being 
\begin{align}
  \frac{\mathcal{V}^{\prime}(r_{ik})}{r_{ik}}W'(r_{ij})\frac{\bm{r}_{ik},
\end{align}
and the second simply $\bm{r}_{ij}$. 
    

\ref{eq:WODE_exact} is then reducible to
\begin{align}
  W''(r)+3\frac{W'(r)}{r}-\lambda W(r)+\frac{\mathcal{V}'(r)}{2r}=0
\end{align}
where $\lambda=D^r\sigma^2/(2D^t)$.

The second option is to say that, within the sum over the index $k$ in eqn
\ref{eq:WODE_exact}, the terms are essentially averaging out to zero unless $k=j$,
so
\begin{align}
  \bm{r}_{ik}\cdot\bm{u}_{ij}&\approx\delta_{jk}\bm{r}_{ik}\cdot\bm{u}_{ij},\\
  \bm{r}_{ik}\cdot\bm{r}_{ik}&\approx\delta_{jk}\bm{r}_{ik}\cdot\bm{r}_{ij}.
\end{align}
Inserting this approximation into eqn \ref{eq:WODE_exact} then yields
\begin{align}
  W''(r)+W'(r)\bigg(\frac{3}{r}-\mathcal{V}'(r)\bigg)-\bigg(\lambda
  + \frac{\mathcal{V}'(r)}{r}\bigg) W(r)+\frac{\mathcal{V}'(r)}{2r}=0.
\end{align}

\subsection{Initial conditions: Minimum distance of interaction}

To uniquely solve the ODE for $W(r)$, initial conditions are required. One obvious condition
is that $W(r\to\infty)\to0$, but a second condition (near $r=0$) is not so obvious. One
physical set of boundary conditions might be to consider a minimum distance of interaction
between particles.

Consider three particles which are all moving toward a single, central point at the origin.
The (dimensionless) velocity of each particle is
\begin{align}
  \bm{v}_i&=\nabla h -\nabla_i\mathcal{H} + \epsilon \bm{\hat{u}}_i\nonumber\\
          &=\epsilon \bigg(\bm{\hat{u}}_i-\nabla_i\sum_{j=1}^N\sum_{k=1}^{j-1}W(r_{jk})
            \bm{r}_{jk}\cdot\bm{u}_{jk}\bigg)\nonumber\\
          &=\epsilon \bigg(\bm{\hat{u}}_i-\sum_{j\neq i}^N \bigg[W'(r_{ij})
            (\bm{u}_{ij}\cdot\bm{r}_{ij})\frac{\bm{r}_{ij}}{r_{ij}}
            +W(r_{ij})\bm{u}_{ij}\bigg]\bigg),
\end{align}
where I have used eqns \ref{eq:hperturb} and \ref{eq:zetaform}. When all three particles are
hitting each other simultaneously, they should all have zero relative velocity with respect
to each other. Therefore, for this three particle system, the boundary condition on $W(r)$
at $r_0$ is
\begin{align}
  1-3W'(r_0)r_0-3W(r_0)=0,
\end{align}
which can be checked for any of the two particle pairs (e.g. $\bm{v}_1-\bm{v}_2$, see file
''3-hard-spheres-BC.png'' in \textit{ipadCalcs} folder).

%%%%%%%%%%%%%%%%%%%%%%%%%%%%%%%%%%%%%%%%%%%%%%%%%%%%%%%%%%%%%%%%%%%%%%%%%%%%%%%%
%%% Below are old calculations. %%%%%%%%%%%%%%%%%%%%%%%%%%%%%%%%%%%%%%%%%%%%%%%%
%%%%%%%%%%%%%%%%%%%%%%%%%%%%%%%%%%%%%%%%%%%%%%%%%%%%%%%%%%%%%%%%%%%%%%%%%%%%%%%%


\begin{comment}
\subsection{Separable $h$ function.}

To solve eqn \ref{eq:steadystatecondition}, I will start by writing a guess for $h(\bm{x})$, and I
will take it to be of the form
\begin{align}\label{eq:hgeneric}
  h(\bm{x})=R(\{\bm{r}_i\})\Theta(\{\theta_i\})
\end{align}
where $R(\{\bm{r}_i\})$ and $\Theta(\{\theta_i\})$ are functions which I will determine later.

\subsubsection{$R(\{\bm{r}_i\})$ function.}

I will focus on pairwise interactions, so
\begin{align}
  R(\{\bm{r}_k\})= \sum_{i=1}^N\sum_{j=1}^{i-1}b(r_{ij})
\end{align}
A good variational guess to the form of $b(r_{ij})$ might be something like a standard
Lennard-Jones effective potential
\begin{align}
  b(r_{ij})\equiv 4\eta\bigg[\bigg(\frac{\delta}{r_{ij}}\bigg)^{12}-
  \bigg(\frac{\delta}{r_{ij}}\bigg)^6\bigg]
\end{align}
where $\eta$ and $\delta$ are effective parameters for the effective interaction. Note that if
$f_P=0$ (i.e. no activity), then $\eta=\epsilon$ and $\delta=\sigma$.

\subsubsection{$\Theta(\{\theta_i\})$ function.}

A good starting guess for $\Theta$ might be a simple function like
\begin{align}
  \Theta(\{\theta_i\})&=a_0\sum_{i=1}^N\sum_{j=1}^{i-1}(\cos(\theta_i-\theta_j))
\end{align}

\subsubsection{Generic N-particle case for variational form of $h$.}
For the generic, $N$-particle case, the first term in eqn \ref{eq:steadystatecondition} becomes
\begin{align}\label{eq:genericNpartD_t}
  L_i(h) =& \beta^{-1}\Theta(\{\theta_l\})^2\sum_{j\neq i}
            \sum_{k\neq i} b'(r_{ij})b'(r_{ik})
            \frac{\bm{r}_{ij}\cdot\bm{r}_{ik}}{r_{ij}r_{ik}}
            + \sum_{j\neq i}\bigg[\mathcal{V}''(r_{ij})
            +\frac{\mathcal{V}'(r_{ij})}{r_{ij}}\bigg]\nonumber\\
          & - \Theta(\{\theta_l\})\sum_{j\neq i}\sum_{k\neq i}
            b'(r_{ij})\mathcal{V}'(r_{ik})
            \frac{\bm{r}_{ij}\cdot\bm{r}_{ik}}{r_{ij}r_{ik}}
            +\Theta(\{\theta_l\})f^P(\cos\theta_i\hat{\bm{e}}_x+\sin\theta_i\hat{\bm{e}}_y)
            \sum_{j\neq i}b'(r_{ij})
            \frac{\bm{r}_{ij}}{r_{ij}}\nonumber\\
          & - \beta^{-1}\Theta(\{\theta_l\})\sum_{j\neq i}\bigg[b''(r_{ij})
            +\frac{b'(r_{ij})}{r_{ij}}\bigg],\:\:\: i = 1,2,...,N
\end{align}

Here I have defined $\bm{r}_{ij}=(x_i-x_j)\hat{\bm{e}}_x+(y_i-y_j)\hat{\bm{e}}_y$ as the vector
between particles $i$ and $j$. I have also just combined the first $2N$ indices into just $N$
indices by using the unit vectors $\hat{\bm{e}}_x$ and $\hat{\bm{e}}_y$. Further, I have used the
fact that $\partial_i v_i=0$ always, and $v_i$ is only non-zero for the position terms.

Inserting eqn \ref{eq:genericNpartD_t} into eqn \ref{eq:steadystatecondition} will give a method of
determining $b(r)$ and $\Theta(\{\theta_l\})$ in terms of the microscopic parameters $D^r$, $D^t$,
and $f^P$. The resulting equation is
\begin{align}\label{eq:separable_h_algebraic}
  0
  =&\sum_{i}\beta^{-1}\Theta(\{\theta_l\})^2
     \sum_{j\neq i}\sum_{k\neq i} b'(r_{ij})b'(r_{ik})
     \frac{\bm{r}_{ij}\cdot\bm{r}_{ik}}{r_{ij}r_{ik}}
     + \sum_i\sum_{j\neq i}\bigg[\mathcal{V}''(r_{ij})
     +\frac{\mathcal{V}'(r_{ij})}{r_{ij}}\bigg]\nonumber\\
   & - \sum_i\Theta(\{\theta_l\})\sum_{j\neq i}\sum_{k\neq i}
     b'(r_{ij})\mathcal{V}'(r_{ik})
     \frac{\bm{r}_{ij}\cdot\bm{r}_{ik}}{r_{ij}r_{ik}}
     +\sum_i\Theta(\{\theta_l\})
     f^P(\cos\theta_i\hat{\bm{e}}_x+\sin\theta_i\hat{\bm{e}}_y)
     \sum_{j\neq i}b'(r_{ij})
     \frac{\bm{r}_{ij}}{r_{ij}}\nonumber\\
   & - \beta^{-1}\sum_i\Theta(\{\theta_l\})\sum_{j\neq i}\bigg[b''(r_{ij})
     +\frac{b'(r_{ij})}{r_{ij}}\bigg]\nonumber\\
   & + \frac{D^r}{D^t}\sum_{i=1}^{N}\big[\sum_{j\neq i}\sum_{k\neq i} b(r_{ij})b(r_{ik})
     (\partial_{\theta_i}\Theta)^2
     -\sum_{j\neq i}b(r_{ij})\partial_{\theta_i}^2\Theta\big]
\end{align}
  

In order to simplify things now, I will use the fact that
\begin{align}
  f(x_i) = \int dx f(x)\delta(x-x_i),
\end{align}
and so if I define
\begin{align}
  \hat{\rho}(\bm{x},\theta) = \sum_j \delta^2(\bm{x}-\bm{r}_j)\delta(\theta-\theta_j)
\end{align}
I can coarse-grain the system. I will neglect any self-energy issues, as it will be evident
soon that this can be accounted for by introducing a microscopic length scale.
I can re-write sums
\begin{align}
  \sum_i f(\bm{r}_i,\theta_i) = \int d^2rd\theta\hat{\rho}(\bm{r},\theta)f(\bm{r},\theta)
\end{align}

For any distribution function $\rho(\bm{r},\theta)$ which is independent in of $\theta$, the
activity parameter $f^P$ of eqn \ref{eq:separable_h_algebraic} will drop out as the integrals
over $\cos\theta$ and $\sin\theta$ (for the sum with dummy index $i$) will be zero. So a
separable guess to $h$ may not be viable.

\subsection{First variational guess for coupled $h$ function.}

In this section, I will take
\begin{align}
  h = \sum_{i=1}^N\sum_{j=1}^{i-1}b(r_{ij})\cos(\theta_{ij})
\end{align}

Then, the derivatives become
\begin{align}\label{eq:partial_i_h}
  \nabla_{\bm{r}_{\alpha}}h
  &=\sum_{i\neq\alpha}^Nb^{\prime}
    (r_{\alpha i})\frac{\bm{r}_{\alpha i}}{r_{\alpha i}}\cos(\theta_{\alpha i}),
    \:\:\:\alpha=1,2,...,N,
\end{align}
\begin{align}\label{eq:partialsq_i_h}
  \nabla_{\bm{r}_{\alpha}}^2h
  &=\sum_{i\neq\alpha}^N\bigg[\frac{b^{\prime}(r_{\alpha i})}{r_{\alpha i}}
    +b^{\prime\prime}(r_{\alpha i})\bigg]\cos(\theta_{\alpha i}),
    \:\:\:\alpha=1,2,...,N,
\end{align}
\begin{align}
  \partial_{\theta_{\alpha}}h
  &=-\sum_{i\neq\alpha}^Nb(r_{\alpha i})\sin\theta_{\alpha i},
\end{align}
\begin{align}
  \partial_{\theta_{\alpha}}^2h
  &=-\sum_{i\neq\alpha}^Nb(r_{\alpha i})\cos\theta_{\alpha i}.
\end{align}

Inserting these into eqn \ref{eq:steadystatecondition} gives
\begin{align}\label{eq:easiest_non_separable}
  0
  =&\sum_{i}\beta^{-1}\sum_{j\neq i}\sum_{k\neq i} b'(r_{ij})b'(r_{ik})
     \frac{\bm{r}_{ij}\cdot\bm{r}_{ik}}{r_{ij}r_{ik}}\cos\theta_{ij}\cos\theta_{ik}
     + \sum_i\sum_{j\neq i}\bigg[\mathcal{V}''(r_{ij})
     +\frac{\mathcal{V}'(r_{ij})}{r_{ij}}\bigg]\nonumber\\
   & - \sum_i\sum_{j\neq i}\sum_{k\neq i}
     b'(r_{ij})\cos\theta_{ij}\mathcal{V}'(r_{ik})
     \frac{\bm{r}_{ij}\cdot\bm{r}_{ik}}{r_{ij}r_{ik}}
     +\sum_i
     f^P(\cos\theta_i\hat{\bm{e}}_x+\sin\theta_i\hat{\bm{e}}_y)
     \sum_{j\neq i}b'(r_{ij})\cos(\theta_{ij})
     \frac{\bm{r}_{ij}}{r_{ij}}\nonumber\\
   & - \beta^{-1}\sum_i\sum_{j\neq i}\cos\theta_{ij}\bigg[b''(r_{ij})
     +\frac{b'(r_{ij})}{r_{ij}}\bigg]\nonumber\\
   & + \frac{D^r}{D^t}\sum_i\sum_{j\neq i}\sum_{k\neq i} b(r_{ij})b(r_{ik})
     \sin\theta_{ij}\sin\theta_{ik}
     +\frac{D^r}{D^t}\sum_i\sum_{j\neq i}b(r_{ij})\cos\theta_{ij}
\end{align}

However, because of the identity
\begin{align}
  \cos(\theta_{ij})=\cos\theta_i\cos\theta_j+\sin\theta_i\sin\theta_j,
\end{align}
the issue of $f^P$ dropping out still occurs for isotropic distribution functions.

\subsection{Second variational guess for coupled $h$ function.}

This time, I will take
\begin{align}
  h = \sum_{i=1}^N\sum_{j=1}^{i-1}b(r_{ij})
  \big[1-q_0\cos(\theta_{ij}+\zeta)\cos(\theta_{ij}+\zeta)\big].
\end{align}
\end{comment}


\begin{comment}
\subsubsection{Isotropic distribution function.}

I will constrain $\hat{\rho}(\bm{x},\theta)$ to be independent of $\theta$, corresponding to
an isotropic distribution function. Then, I will redefine the distribution as
\begin{align}
  \rho(\bm{x})&\equiv\int d\theta\hat{\rho}(\bm{x},\theta)\nonumber\\
\end{align}

Adding all of these contributions, the sum of $L_i(h)$ over all $N$ is then
\begin{align}
  \sum_i L_i(h)
  &=\beta^{-1}\Theta_0^2\int d^2xd^2yd^2z\rho(\bm{x})\rho(\bm{y})
    \rho(\bm{z})b'(|\bm{z}-\bm{x}|)b'(|\bm{z}-\bm{y}|)
    \frac{(\bm{z}-\bm{x})\cdot(\bm{z}-\bm{y})}{|\bm{z}-\bm{x}||\bm{z}-\bm{y}|}\nonumber\\
  &\phantom{=}-4\pi\beta^{-1}\Theta_0^2\int d^2xd^2y\rho(\bm{x})\rho(\bm{y})
    b'(|\bm{y}-\bm{x}|)b'(0)\bm{q}(\bm{0})\cdot[\bm{y}-\bm{x}]\nonumber\\
  &\phantom{=}+4\pi^2\beta^{-1}\Theta_0^2Nb'(0)^2(\bm{q}(\bm{0}))^2\nonumber\\
  &\phantom{=}+\int d^2xd^2y\rho(\bm{y})\rho(\bm{x})
    \bigg[\mathcal{V}''(|\bm{x}-\bm{y}|)
    +\frac{\mathcal{V}'(|\bm{x}-\bm{y}|)}{|\bm{x}-\bm{y}|}\bigg]
    -2\pi N[\mathcal{V}''(0)+s(0)]\nonumber\\
  &\phantom{=}
    -\Theta_0\int d^2xd^2yd^2z\rho(\bm{x})\rho(\bm{y})
    \rho(\bm{z})b'(|\bm{z}-\bm{x}|)\mathcal{V}'(|\bm{z}-\bm{y}|)
    \frac{(\bm{z}-\bm{x})\cdot(\bm{z}-\bm{y})}{|\bm{z}-\bm{x}||\bm{z}-\bm{y}|}\nonumber\\
  &\phantom{=}+2\pi\Theta_0\int d^2xd^2y\rho(\bm{x})\rho(\bm{y})
    \bigg[b'(|\bm{y}-\bm{x}|)\mathcal{V}'(0)+\mathcal{V}'(|\bm{y}-\bm{x}|)b'(0)\bigg]
    \bm{q}(\bm{0})\cdot[\bm{y}-\bm{x}]\nonumber\\
  &\phantom{=}-4\pi^2\Theta_0Nb'(0)\mathcal{V}'(0)(\bm{q}(\bm{0}))^2\nonumber\\
  &\phantom{=}-\beta^{-1}\Theta_0\int d^2xd^2y\rho(\bm{y})\rho(\bm{x})
    \bigg[b''(|\bm{x}-\bm{y}|)
    +\frac{b'(|\bm{x}-\bm{y}|)}{|\bm{x}-\bm{y}|}\bigg]
    +2\pi\beta^{-1}\Theta_0 N[b''(0)+t(0)].
\end{align}

\subsection{Constant density calculation}
If I take $\rho(\bm{x})=\rho_0$ as the average density, then
\begin{align}
  0
  &=\beta^{-1}\Theta_0^2\rho_0^3\int d^2xd^2yd^2z
    b'(|\bm{z}-\bm{x}|)b'(|\bm{z}-\bm{y}|)
    \frac{(\bm{z}-\bm{x})\cdot(\bm{z}-\bm{y})}{|\bm{z}-\bm{x}||\bm{z}-\bm{y}|}\nonumber\\
  &\phantom{=}-4\pi\beta^{-1}\Theta_0^2\rho_0^2\int d^2xd^2y
    b'(|\bm{y}-\bm{x}|)b'(0)\bm{q}(\bm{0})\cdot[\bm{y}-\bm{x}]\nonumber\\
  &\phantom{=}+4\pi^2\beta^{-1}\Theta_0^2Nb'(0)^2(\bm{q}(\bm{0}))^2\nonumber\\
  &\phantom{=}+\rho_0^2\int d^2xd^2y
    \bigg[\mathcal{V}''(|\bm{x}-\bm{y}|)
    +\frac{\mathcal{V}'(|\bm{x}-\bm{y}|)}{|\bm{x}-\bm{y}|}\bigg]
    -2\pi N[\mathcal{V}''(0)+s(0)]\nonumber\\
  &\phantom{=}
    -\Theta_0\rho_0^3\int d^2xd^2yd^2z
    b'(|\bm{z}-\bm{x}|)\mathcal{V}'(|\bm{z}-\bm{y}|)
    \frac{(\bm{z}-\bm{x})\cdot(\bm{z}-\bm{y})}{|\bm{z}-\bm{x}||\bm{z}-\bm{y}|}\nonumber\\
  &\phantom{=}+2\pi\Theta_0\rho_0^2\int d^2xd^2y
    \bigg[b'(|\bm{y}-\bm{x}|)\mathcal{V}'(0)+\mathcal{V}'(|\bm{y}-\bm{x}|)b'(0)\bigg]
    \bm{q}(\bm{0})\cdot[\bm{y}-\bm{x}]\nonumber\\
  &\phantom{=}-4\pi^2\Theta_0Nb'(0)\mathcal{V}'(0)(\bm{q}(\bm{0}))^2\nonumber\\
  &\phantom{=}-\beta^{-1}\Theta_0\rho_0^2\int d^2xd^2y
    \bigg[b''(|\bm{x}-\bm{y}|)
    +\frac{b'(|\bm{x}-\bm{y}|)}{|\bm{x}-\bm{y}|}\bigg]
    +2\pi\beta^{-1}\Theta_0 N[b''(0)+t(0)]\nonumber\\
  &=\beta^{-1}\Theta_0^2\rho_0^3V\int d^2ud^2v
    b'(|\bm{u}|)b'(|\bm{v}|)
    \frac{\bm{u}\cdot\bm{v}}{|\bm{u}||\bm{v}|}\nonumber\\
  &\phantom{=}-4\pi\beta^{-1}\Theta_0^2\rho_0^2V\int d^2u
    b'(|\bm{u}|)b'(0)\bm{q}(\bm{0})\cdot\bm{u}\nonumber\\
  &\phantom{=}+4\pi^2\beta^{-1}\Theta_0^2Nb'(0)^2(\bm{q}(\bm{0}))^2\nonumber\\
  &\phantom{=}+\rho_0^2V\int d^2u
    \bigg[\mathcal{V}''(|\bm{u}|)
    +\frac{\mathcal{V}'(|\bm{u}|)}{|\bm{u}|}\bigg]
    -2\pi N[\mathcal{V}''(0)+s(0)]\nonumber\\
  &\phantom{=}
    -\Theta_0\rho_0^3V\int d^2ud^2v
    b'(|\bm{u}|)\mathcal{V}'(|\bm{v}|)
    \frac{\bm{u}\cdot\bm{v}}{|\bm{u}||\bm{v}|}\nonumber\\
  &\phantom{=}+2\pi\Theta_0\rho_0^2V\int d^2x
    \bigg[b'(|\bm{u}|)\mathcal{V}'(0)+\mathcal{V}'(|\bm{u}|)b'(0)\bigg]
    \bm{q}(\bm{0})\cdot\bm{u}\nonumber\\
  &\phantom{=}-4\pi^2\Theta_0Nb'(0)\mathcal{V}'(0)(\bm{q}(\bm{0}))^2\nonumber\\
  &\phantom{=}-\beta^{-1}\Theta_0\rho_0^2V\int d^2u
    \bigg[b''(|\bm{u}|)
    +\frac{b'(|\bm{u}|)}{|\bm{u}|}\bigg]
    +2\pi\beta^{-1}\Theta_0 N[b''(0)+t(0)].
\end{align}

If I just re-write the above equations with some molecular length scale cutoff, $a_0$, then
the divergent terms go away (i.e. all terms with a $f(0)$ in them), and so
\begin{align}\label{eq:constantdensity_not_integrated_condition}
  0
  &=\beta^{-1}\rho_0\int_{u>a_0,v>a_0} d^2ud^2v
    b'(|\bm{u}|)b'(|\bm{v}|)
    \frac{\bm{u}\cdot\bm{v}}{|\bm{u}||\bm{v}|}\nonumber\\
  &\phantom{=}+\int_{u>a_0} d^2u
    \bigg[\mathcal{V}''(|\bm{u}|)
    +\frac{\mathcal{V}'(|\bm{u}|)}{|\bm{u}|}\bigg]\nonumber\\
  &\phantom{=}
    -\rho_0\int_{u>a_0,v>a_0} d^2ud^2v
    b'(|\bm{u}|)\mathcal{V}'(|\bm{v}|)
    \frac{\bm{u}\cdot\bm{v}}{|\bm{u}||\bm{v}|}\nonumber\\
  &\phantom{=}-\beta^{-1}\int_{u>a_0} d^2u
    \bigg[b''(|\bm{u}|)
    +\frac{b'(|\bm{u}|)}{|\bm{u}|}\bigg]
\end{align}
where I have divided through by a factor of the density squared times the total
volume $\rho_0^2V$ and set $\Theta_0=1$.

For the WCA approximation of eqn \ref{eq:WCAapprox}, the purely $\mathcal{V}$ integrals are
\begin{equation}
  \int_{u>a_0} d^2u\mathcal{V}''(|\bm{u}|) =
  \begin{cases}
    8\pi\epsilon\bigg\{7\bigg[\frac{1}{2}-\bigg(\frac{\sigma}{a_0}\bigg)^6\bigg]
    -13\bigg[\frac{1}{4}-\bigg(\frac{\sigma}{a_0}\bigg)^{12}\bigg]\bigg\},
    &a_0<2^{\frac{1}{6}}\sigma,\\
    0,&a_0\geq 2^{\frac{1}{6}}\sigma,
  \end{cases}
\end{equation}
\begin{equation}
  \int_{u>a_0} d^2u\frac{\mathcal{V}'(|\bm{u}|)}{|\bm{u}|} =
  \begin{cases}
    8\pi\epsilon\bigg\{\bigg[\frac{1}{4}-\bigg(\frac{\sigma}{a_0}\bigg)^{12}\bigg]
    -\bigg[\frac{1}{2}-\bigg(\frac{\sigma}{a_0}\bigg)^6\bigg]\bigg\},
    &a_0<2^{\frac{1}{6}}\sigma,\\
    0,&a_0\geq 2^{\frac{1}{6}}\sigma,
  \end{cases}
\end{equation}
and so the second term in eqn \ref{eq:constantdensity_not_integrated_condition} is
\begin{align}\label{eq:constantdensity_only_V_integral}
  \int_{u>a_0} d^2u\bigg[\mathcal{V}''(|\bm{u}|)
  + \frac{\mathcal{V}'(|\bm{u}|)}{|\bm{u}|}\bigg]
  = 48\pi\epsilon\bigg[2\bigg(\frac{\sigma}{a_0}\bigg)^{12}
  -\bigg(\frac{\sigma}{a_0}\bigg)^6\bigg]>0.
\end{align}
I can see that eqn \ref{eq:constantdensity_only_V_integral} is greater than zero if I insert
$a_0=2^{1/6}\sigma(1-\chi)$ with $0<\chi<1$. I then get
\begin{align}
  24\pi\epsilon\bigg[\frac{1}{(1-\chi)^{12}}-\frac{1}{(1-\chi)^6}\bigg],
\end{align}
which is always greater than zero for $0<\chi<1$.

\subsubsection{Choice of $b(r)$.}

In order to go any further, I need to choose the form of $b(r)$. I will start with a basic square
well,
\begin{align}
  b(r) =\infty-\infty H(r-a_0)+\alpha[H(r-(a_0+\beta))-H(r-a_0)]
\end{align}
where $H(x)$ is the heaviside function. 
With this definition, the first derivative is
\begin{align}
  b'(r) = -\infty\delta(r-a_0)+\alpha[\delta(r-(a_0+\beta))-\delta(r-a_0)]
\end{align}
Then, I can write the integral
\begin{align}
  \int_{u>a_0} d^2ub''(|\bm{u}|)
  &= 2\pi\int_{u>a_0} udub''(u)\nonumber\\
  &=2\pi \int_{u>a_0} [(ub'(u))'-b'(u)]du\nonumber\\
  &=\lim_{\Delta\to0+}2\pi[ub'(u)-b(u)]_{u=a_0+\Delta}^{u=\infty}\nonumber\\
  &=2\pi\alpha.
\end{align}
Similarly, I can the integral
\begin{align}
  \int_{u>a_0} d^2u\frac{b'(|\bm{u}|)}{|\bm{u}|}
  &=2\pi\int_{u>a_0} dub'(u)\nonumber\\
  &=\lim_{\Delta\to0+}2\pi[b(u)]_{u=a_0+\Delta}^{u=\infty}\nonumber\\
  &=-4\pi\alpha
\end{align}
and so
\begin{align}
  \int_{u>a_0} d^2u\bigg[b''(|\bm{u}|)+\frac{b'(|\bm{u}|)}{|\bm{u}|}\bigg]
  = -2\pi\alpha.
\end{align}

For the remaining two integrals, I will use the fact that I can fix $\bm{u}=u \hat{\bm{e}}_x$
for doing the inner $v$ integral, and then allow $\bm{u}$ to vary freely again for the outer
$u$ integral. Then I can write $\bm{u}\cdot\bm{v}=\cos\phi_v$, and 

\begin{align}
  \int_{u>a_0,v>a_0} d^2ud^2vb'(|\bm{u}|)b'(|\bm{v}|)
  \frac{\bm{u}\cdot\bm{v}}{|\bm{u}||\bm{v}|}
  &=0,
\end{align}
since $\int_0^{2\pi} \cos\phi_v d\phi_v=0$. This means that the only possible option of satisfying
eqn \ref{eq:constantdensity_not_integrated_condition} is to have $b(r)=\mathcal{V}(r)$. This makes
me think we might need to change our assumptions on the form of $h$ being independent of
$\{\theta_i\}$.
\end{comment}
    



%%%%%%%%%%%%%%%%%%%%%%%%%%%%%%%%%%%%%%%%
% Below this, I have an old calculation from before I really understood what to do... %
%%%%%%%%%%%%%%%%%%%%%%%
\begin{comment}
Then, the full interaction potential is
\begin{align}
    \mathcal{H}=\sum_{\alpha=1}^N\sum_{\beta=1}^{\alpha-1}\mathcal{V}^{\alpha\beta},
\end{align}
where I am summing over the $N(N-1)/2$ different pairings of the positions.
Noting that
\begin{equation}\label{eq:dVij}
    \frac{\partial \mathcal{V}^{\alpha\beta}}{\partial r^{\alpha\beta}}=
    \begin{cases}
    -24\epsilon\bigg[\bigg(\sigma^{12}(r^{\alpha\beta})^{-13}-\sigma^6(r^{\alpha\beta})^{-6}\bigg], & r^{\alpha\beta}<2^{\frac{1}{6}}\sigma \\
    0, & \mathrm{otherwise},
    \end{cases}
\end{equation}
and
\begin{align}\label{eq:drij}
    \frac{\partial r^{\alpha\beta}}{\partial x_k}&=\frac{(x_{\alpha}-x_{\beta})(\delta_{k\alpha}-\delta_{k\beta})}{r^{\alpha\beta}},
\end{align}
the gradient of the interaction potential is
\begin{equation}
    \frac{\partial}{\partial x_k}\mathcal{H}=
    \begin{cases}
    \sum\limits_{\substack{\beta=1\\ \beta\neq k}}^N Q_{k\beta}, &1\leq k \leq N,\\
    \sum\limits_{\substack{\beta=N+1\\ \beta\neq k}}^{2N} Q_{k\beta}, &N+1\leq k \leq 2N,\\
    0, & k > 2N
    \end{cases}
\end{equation}
where I have defined $Q_{\alpha\beta}=-Q_{\beta\alpha}$ is
\begin{align}
    Q_{\alpha\beta}=\frac{\partial\mathcal{V}^{\alpha\beta}}{\partial r^{\alpha\beta}}\times\frac{(x_{\alpha}-x_{\beta})}{r^{\alpha\beta}}
\end{align}

\end{comment}

\bibliographystyle{plain}
\bibliography{bib}

\end{document}