\documentclass[twocolumn,amsmath,amssymb,aps]{revtex4-1}%{article}
%\usepackage[hmargin=1.25cm,vmargin=2.5cm]{geometry}
\usepackage[utf8]{inputenc}
\usepackage{bm}
%\usepackage{amsmath}
%\usepackage{amssymb}
\usepackage{comment}
\usepackage{graphicx}
\usepackage{dcolumn}
\usepackage[caption=false]{subfig}
\usepackage{subfiles}

\begin{document}
 
\title{calculation: coarse-graining active brownian particles}
\author{Sam Cameron}
\affiliation{%
  University of Bristol}%
\date{\today}

\begin{abstract}
  In equilibrium systems, collective properties are calculated
  through knowledge of the Boltzmann probability density,
  $e^{-\beta \mathcal{H}(\{\bm{x}_i)\}}$ and the corresponding partition
  function. In contrast, non-equilibrium/active systems do not obey
  Boltzmann statistics, and so the functional form of the steady-state
  probability density (if it exists) is not known a priori. In this letter,
  we present a systematic way of calculating the steady-state
  probability density of an active systems. As a proof of concept, we use
  this method to calculate the steady-state probability density of
  interacting active brownian particles. Finally, we compare our calculation
  to Molecular Dynamics simulations.
\end{abstract}


\maketitle

\section{Writing the active brownian particles equations of motion into
  the correct form}

The equations of motion for active brownian particles are:
\begin{subequations}
  \label{eqs:basicABPsODEs}
  \begin{align}
    \frac{d\bm{r}_i}{dt}&=\beta D^t\big(\bm{F}_i
    +f^P\bm{P}_i\big)
    +\sqrt{2D^t}\tilde{\bm{u}}_i,\label{eq:micro_pos}\\
    \frac{d\theta_i}{dt}&=\sqrt{2D^r}w_i.\label{eq:micro_theta}
  \end{align}
\end{subequations}
where $\bm{r}_i=(x_i,y_i)$ are the particle positions,
$\bm{P}_i=(\cos\theta_i,\sin\theta_i)$ are the particle orientations,
and $\tilde{\bm{u}}_i=(\tilde{u}_i,v_i)$, $w_i$ are Gaussian white noise
with zero mean and unit variance, i.e. for any two components of the vector
$\bm{s}=(u_i,v_i,w_i)$,
\begin{align}
  \big<s_i(t)\big>=0;\;\;\;\;\; \big<s_i(t)s_j(t^{\prime})\big>=
  \delta_{ij}\delta(t-t^{\prime}).
\end{align}.

This is equivalent to a Fokker-Planck equation for the $N$-body Fokker-Planck
equation (see appendix \ref{app:fokkerplanck})
\begin{align}\label{eq:NparticleFP}
  \frac{\partial P}{\partial t} +
  \nabla_{\bm{x}}\cdot\bm{J}=0.
\end{align}
The current $\bm{J}$ is
\begin{align}\label{eq:Nparticlecurrent}
  \bm{J}(\bm{x},t)=
  &\bm{D}\cdot(\beta\bm{F}(\bm{x})P(\bm{x},t)
  +\beta\bm{v}(\bm{x})P(\bm{x},t)\nonumber\\
  &-\nabla P(\bm{x},t)).
\end{align}
where $\bm{D}$ is a diagonal, and equal to $D^t$ for the first $2N$ components,
and equal to $D^r$ for the last $N$.
The vector $\bm{F}$ is for simplicity just two-body interactions, and so is a
concatenation of the two-component vectors $\bm{F}_i$ from eqn
\ref{eq:micro_pos} along with a final $N$ components all equal to zero (so the
total number of components of $\bm{F}$ is $3N$, but only the first $2N$ are
non-zero). The two body interactions will be written as
\begin{align}\label{eq:general_conservative_force}
  \bm{F}_i&=-\nabla_{i}\mathcal{H}(\bm{x})\nonumber\\
          &=-\nabla_{i}\sum_{j=1}^N\sum_{k=1}^{j-1}\mathcal{V}(r_{jk}),
\end{align}
where $\bm{r}_{ij}=\bm{r}_j-\bm{r}_i$.
The vector $\bm{v}$ cannot be represented
as the gradient of a scalar, and is explicitly given by a concatenation of two-
component vectors $\bm{v}_i$, and then a final $N$ zeros added on. The
$\bm{v}_i$ take the form
\begin{align}\label{eq:dissipative_force}
  \bm{v}_i =f^P\bm{u}(\theta_i)
\end{align}
where $\bm{u}(\theta)=(\cos\theta,\sin\theta)$.

It can be shown \cite{liverpool2018nonequilibrium} that there exists a
generalized steady state of the form
\begin{align}
  P_{ss}=\exp(-h(\bm{x})).
\end{align}
For ABPs, $h(\bm{x})$ satisfies the equation
\begin{align}\label{eq:steadystatecondition}
  D^t\beta\sum_{i=1}^{N}L_i(h)
  +D^r\sum_{i=1}^{N}\big[(\partial_{\theta_i}h)^2
  -\partial_{\theta_i}^2h\big]=0
\end{align}
where
\begin{align}\label{eq:L_i}
  L_i(h) =& \beta^{-1}(\nabla_ih)^2+\nabla_i^2\mathcal{H}
  +(\bm{v}_i-\nabla_i\mathcal{H})\cdot\nabla_ih\nonumber\\
  &-\beta^{-1}\nabla_i^2h-\nabla_i\cdot\bm{v}_i.
\end{align}

I can explicitly calculate the terms not involving $h$ in eqn \ref{eq:L_i}
using eqn \ref{eq:general_conservative_force}, which gives the relevant
derivatives of $\mathcal{H}$ and $\bm{v}_i$,
\begin{subequations}
  \label{eqs:partials_of_L_i}
  \begin{align}
    \nabla_{\alpha}\mathcal{H}
    &=-\sum_{i\neq\alpha}^N\mathcal{V}^{\prime}
    (r_{\alpha i})\frac{\bm{r}_{\alpha i}}{r_{\alpha i}},
    \label{eq:partial_H_of_L_i}\\
    \nabla_{\alpha}^2\mathcal{H}
    &=\sum_{i\neq\alpha}^N\bigg[\frac{\mathcal{V}^{\prime}
        (r_{\alpha i})}{r_{\alpha i}}
      +\mathcal{V}^{\prime\prime}(r_{\alpha i})\bigg],
    \label{eq:partialsq_H_of_L_i}\\
    \nabla_i\cdot \bm{v}_i
    &= 0.\label{eq:partial_v_of_L_i}
  \end{align}
\end{subequations}

\section{Computing the steady $h(\bm{x})$ function.}

Generically, we expect that the microscopic $h$ function will be invariant
under translations, so
\begin{align}
  h(\{\bm{r}_i+\bm{T}\},\{\theta_i\}) = h(\{\bm{r}_i\},\{\theta_i\}).
\end{align}
Further to this invariance, the system must also be invariant
under arbitrary rotations, and so
\begin{align}
  h(\{\bm{Q}(\phi)\bm{r}_i\},\{\theta_i+\phi\})
  =h(\{\bm{r}_i\},\{\theta_i\}).
\end{align}
The system should also have periodic rotational invariance, so
\begin{align}
  h(\{\bm{r}_i\},\{\theta_i+2\pi\})
    =h(\{\bm{r}_i\},\{\theta_i\}).
\end{align}
And finally, interchanging particles should not vary the results, so swapping
$\bm{r}_i,\theta_i$ with $\bm{r}_j,\theta_j$ should leave $h$ invariant.

To generically have all of these features, we require
\begin{align}
  h(\{r_{ij}\},\{\bm{\hat{u}}_{ij}\cdot\bm{\hat{u}}_{kl}\},
  \{\bm{r}_{ij}\cdot\bm{\hat{u}}_{kl}\})
\end{align}
where $\bm{\hat{u}}_i = (\cos\theta_i,\sin\theta_i)$.

\subsection{Non-dimensionalisation.}

Before doing this calculation, I will write everything in dimensionless form.
If $\tilde{x}$ are the dimensional variables, then
\begin{align}
  \tilde{\mathcal{H}} &= \beta^{-1}\mathcal{H},\\
  \tilde{h} &= h,\\
  \tilde{\bm{r}} &= \sigma\bm{r},\\
  \tilde{\nabla} &= \sigma^{-1}\nabla.
\end{align}
Here, $\sigma$ is some microscopic length scale defined in the Hamiltonian
$\mathcal{H}$. I will also define
\begin{align}
  \Pi &= D^r\sigma^2/(2D^t),\\
  \lambda &= f^P\sigma\beta.
\end{align}
From this point forward, all variables will be held in their dimensionless
forms.


Then, I can re-write eqn \ref{eq:steadystatecondition} as a sum of terms
linear and non-linear in $\mathrm{h}$,
\begin{align}\label{eq:dim_ss}
  \sum_{i=1}^N\big[\mathcal{L}_i^{(0)}(h)
    +\lambda\mathcal{L}_i^{(1)}(h)\big] = 0
\end{align}
where
\begin{align}
  \mathcal{L}_i^{(0)}(h)=&(\nabla_i h)^2 + \nabla_i^2\mathcal{H}
  -\nabla_i\mathcal{H}\cdot\nabla_ih-\nabla_i^2h\nonumber\\
  &+2\Pi\big[(\partial_{\theta_i}h)^2-\partial_{\theta_i}^2h\big],
\end{align}
and
\begin{align}
  \mathcal{L}_i^{(1)}(h)=\bm{u}_i\cdot\nabla_ih.
\end{align}
When $\lambda = 0$ ($f^P=0$), the solution is just $h=\mathcal{H}$ as expected.

\subsection{Perturbative guess for coupled $h$ function.}

I will take a perturbative guess to the solution, with $\lambda=f^P\sigma\beta$
as my perturbation parameter. So, my guess will be
\begin{align}\label{eq:hperturb}
  h(\lambda) = \mathcal{H}
  + \lambda\frac{\partial h}{\partial \lambda}\bigg|_{\lambda=0}.
\end{align}

Differentiating eqn \ref{eq:dim_ss} with respect to $\lambda$ gives
\begin{align}\label{eq:first_perturb_DE}
  0
  =&\sum_{i=1}^N\big\{\partial_{\lambda}\mathcal{L}_i^{(0)}(h)
    + \mathcal{L}_i^{(1)}(h(0))
    +\lambda\partial_{\lambda}\mathcal{L}_i^{(1)}
    (h)\big\}\bigg|_{\lambda=0}\nonumber\\
  =&\sum_{i=1}^N\big\{2\nabla_i \mathcal{H}\cdot\nabla_i\zeta
    - \nabla_i\mathcal{H}\cdot\nabla_i\zeta
    -\nabla_i^2\zeta\nonumber\\
    &+2\Pi\big[2\partial_{\theta_i}\mathcal{H}\partial_{\theta_i}\zeta
    -\partial_{\theta_i}^2\zeta\big]
    + \bm{u}_i\cdot\nabla_i\mathcal{H}
    +(0)\bm{u}_i\cdot\nabla_i\zeta\big\}\nonumber\\
  =&\sum_{i=1}^N\big[\nabla_i \mathcal{H}\cdot\nabla_i\zeta
    -\nabla_i^2\zeta - 2\Pi
    \partial_{\theta_i}^2\zeta
    + \bm{u}_i\cdot\nabla_i\mathcal{H}\big]
\end{align}
where $\zeta\equiv\partial_{\lambda}h|_{\lambda=0}$.

Assuming $h$ is analytic in its variables, I will Taylor expand it to
first order as
\begin{align}\label{eq:zetaform}
  \zeta &= \sum_{i=1}^N\sum_{j=1}^{i-1}
      w(r_{ij})\bm{r}_{ij}\cdot\bm{u}_{ij}.
\end{align}
Here, I am neglecting terms of the form $\bm{u}_{ij}\cdot\bm{u}_{ij}$ as
they will only be (on average) non-zero when a non-zero polarity is present
in the system (and MIPS is not a polar phase).


Since, for $i\neq j$,
\begin{align}
  \nabla_i(\bm{r}_{jk}\cdot\bm{u}_{jk})
  &= \bm{u}_{jk}(\delta_{ik}-\delta_{ij}),\\
  \nabla_i r_{jk}
  &=\frac{\bm{r}_{jk}}{r_{jk}}(\delta_{ik}-\delta_{ij}),\\
  \partial_{\theta_i}^2(\bm{u}_{ij}\cdot\bm{r}_{ij})
  &=\bm{u}_i\cdot\bm{r}_{ij},\\
  \partial_{\theta_j}^2(\bm{u}_{ij}\cdot\bm{r}_{ij})
  &=-\bm{u}_j\cdot\bm{r}_{ij},
\end{align}
the terms are
\begin{align}
  \nabla_{i}\zeta = -\sum_{j\neq i}^N \bigg[w'(r_{ij})
  (\bm{u}_{ij}\cdot\bm{r}_{ij})\frac{\bm{r}_{ij}}{r_{ij}}
  +w(r_{ij})\bm{u}_{ij}\bigg],
\end{align}

\begin{align}
  \nabla_i^2\zeta=\sum_{j\neq i}^N\bigg[w''(r_{ij})
  +3\frac{w'(r_{ij})}{r_{ij}}\bigg](\bm{u}_{ij}\cdot\bm{r}_{ij}).
\end{align}

\begin{align}
  \partial_{\theta_i}^2\zeta=\sum_{j\neq i}w(r_{ij})\bm{u}_i\cdot\bm{r}_{ij}
\end{align}

Inserting these into eqn \ref{eq:first_perturb_DE}
\begin{widetext}
\begin{align}
  0=
  &\sum_{i=1}^N\sum_{j\neq i}^N\sum_{k\neq i}^N
  \mathcal{V}^{\prime}(r_{ij})\frac{\bm{r}_{ij}}{r_{ij}}
  \cdot\bigg[w'(r_{ik})
    (\bm{u}_{ik}\cdot\bm{r}_{ik})\frac{\bm{r}_{ik}}{r_{ik}}
    +w(r_{ik})\bm{u}_{ik}\bigg]
  -\sum_{i=1}^N\sum_{j\neq i}^N\bigg[w''(r_{ij})
    +3\frac{w'(r_{ij})}{r_{ij}}\bigg](\bm{u}_{ij}\cdot\bm{r}_{ij})\nonumber\\
  &-2\Pi\sum_{i=1}^N\sum_{j\neq i}^Nw(r_{ij})\bm{u}_i\cdot\bm{r}_{ij}
  -\sum_{i=1}^N\sum_{j\neq i}^N 
  \mathcal{V}^{\prime}(r_{ij})\frac{\bm{u}_i\cdot\bm{r}_{ij}}{r_{ij}}
\end{align}
\end{widetext}

You can factor this equation a bit, using the result that
\begin{align}\label{eq:u_itou_ij}
  \sum_{i=1}^N\sum_{j\neq i}f(r_{ij}) \bm{u}_i\cdot\bm{r}_{ij}
  =-\frac{1}{2}\sum_{i=1}^N\sum_{j\neq i}f(r_{ji})\bm{u}_{ij}\cdot\bm{r}_{ij}
\end{align}
to get
\begin{widetext}
\begin{align}\label{eq:WODE_exact}
  0=
  &\sum_{i=1}^N\sum_{j\neq i}^N\bm{r}_{ij}\cdot\bigg\{\sum_{k\neq i}^N\bigg[
    \frac{\mathcal{V}^{\prime}(r_{ik})}{r_{ik}}
    w'(r_{ij})
    \cdot\frac{\bm{r}_{ik}\cdot\bm{r}_{ij}}{r_{ij}}\bm{u}_{ij}
    +\frac{\mathcal{V}^{\prime}(r_{ik})}{r_{ik}}w(r_{ik})\bm{u}_{ik}\bigg]
  -\bigg[w''(r_{ij})
    +3\frac{w'(r_{ij})}{r_{ij}}\bigg]\bm{u}_{ij}\nonumber\\
  &+\Pi w(r_{ij})\bm{u}_{ij}
  +\mathcal{V}^{\prime}(r_{ij})\frac{\bm{u}_{ij}}{2r_{ij}}\bigg\}\nonumber\\
  =&\sum_{i=1}^N\sum_{j\neq i}^N\bm{r}_{ij}\cdot\bm{u}_{ij}
  \bigg\{\sum_{k\neq i}^N\bigg[
    \frac{\mathcal{V}^{\prime}(r_{ik})}{r_{ik}}
    w'(r_{ij})\frac{\bm{r}_{ik}\cdot\bm{r}_{ij}}{r_{ij}}\bigg]
  -w''(r_{ij})-3\frac{w'(r_{ij})}{r_{ij}}\nonumber\\
  &+\Pi w(r_{ij})
  +\frac{\mathcal{V}^{\prime}(r_{ij})}{2r_{ij}}\bigg\}
  +\sum_{i=1}^N\sum_{j\neq i}^N\bm{r}_{ij}\cdot\sum_{k\neq i}^N
  \bm{u}_{ik}\frac{\mathcal{V}^{\prime}(r_{ik})}{r_{ik}}w(r_{ik})
\end{align}
\end{widetext}
So far, the approximations we have made in calculating $h$ are that for
small activity, $h$ is perturbatively close to $\mathcal{H}$, and we have
taken a lowest order guess in what this perturbative term is, in terms of some
arbitrary function $w(r_{ij})$. Now, to be able to get some sort of solution
for eqn \ref{eq:WODE_exact}, we need to make some further approximation. The
way forward is to assume that the system has translational and rotational
invariance in its pairwise-correlation function.

The standard definition of the pairwise-correlation operator is
\begin{align}
  <n(\bm{x}_1)>\hat{g}_2(\bm{x}_1,\bm{x}_2)<(\bm{x}_2)>
  &=\sum_{i=1}^N\sum_{j\neq i}^N\delta(\bm{x}_1-\bm{r}_i)
    \delta(\bm{x}_2-\bm{r}_j).
\end{align}

If the system is translationally invariant, then $\bm{r}=\bm{x}_1-\bm{x}_2$
and
\begin{align}
  \hat{g}_2(\bm{x}_1,\bm{x}_2)
  &=\hat{g}_2(\bm{r})\nonumber\\
  &=\frac{1}{n^2}\sum_{i=1}^N\sum_{j\neq i}^N\delta(\bm{r}+\bm{x}_2-\bm{r}_i)
  \delta(\bm{x}_2-\bm{r}_j)\nonumber\\
  &=\frac{1}{n^2}\frac{1}{V}\int d^2x_2 \sum_{i=1}^N\sum_{j\neq i}^N
  \delta(\bm{r}+\bm{x}_2-\bm{r}_i)\delta(\bm{x}_2-\bm{r}_j)\nonumber\\
  &=\frac{V}{N^2}\sum_{i=1}^N\sum_{j\neq i}
  \delta(\bm{r}+\bm{r}_{ij})\nonumber\\
  &=\frac{V}{N}\sum_{j=1,j\neq i}^N\delta(\bm{r}-\bm{r}_{ij})\nonumber\\
\end{align}
where in the last line I have used the fact that the sum over $j$ runs over
all possible values of the difference $\bm{r}_i-\bm{r}_j$, and so each of the
$N-1\approx N$ terms are identical.

Then, any term of the form
\begin{align}
  \sum_{j\neq i} \bm{f}(\bm{r}_{ij})
  &= \sum_{j\neq i}\int \delta(\bm{r}+\bm{r}_{ij})\bm{f}(-\bm{r})
  d^2r\nonumber\\
  &=\frac{N}{V}\int\hat{g}_2(\bm{r})\bm{f}(-\bm{r})d^2r.
\end{align}

Now, for $g_2(r)=<\hat{g}_2(\bm{r})>$ is the average pair-correlation function,
and is radially symmetric, any integrals over functions of the form
$\bm{\hat{e}}_r f(r)$ will be zero by symmetry. With this in mind, I can see
from both three-sum terms in eqn \ref{eq:WODE_exact} that both have vector
functions which are spherically symmetric, the first being 
\begin{align}
  \frac{\mathcal{V}^{\prime}(r_{ik})}{r_{ik}}w'(r_{ij})\bm{r}_{ik},
\end{align}
and the second simply $\bm{r}_{ij}$. 

Eqn \ref{eq:WODE_exact} is then reducible to
\begin{align}\label{eq:WODE_isotropic}
  w''(r)+3\frac{w'(r)}{r}-\Pi w(r)-\frac{\mathcal{V}'(r)}{2r}=0.
\end{align}
\subsection{Hard-sphere result.}
For the case of a hard sphere potential, the homogeneous solution to eqn
\ref{eq:WODE_isotropic} (plus boundary conditions) gives the final solution.
The homogeneous solution can be found analytically, and is
\begin{align}
  w_h(r)=a_1\frac{J_1(i\sqrt{\Pi}r)}{r}-a_2\frac{Y_1(-i\sqrt{\Pi}r)}{r},
\end{align}
where
\begin{align}
  J_n(z) =& \frac{1}{\pi}\int_0^{\pi}\cos(z\sin\theta-n\theta)d\theta,\\
  Y_n(z) =& -\frac{2\bigg(\frac{1}{2}z\bigg)^{-n}}{\sqrt{\pi}
    \Gamma\bigg(\frac{1}{2}-n\bigg)}
  \int_1^{\infty}\frac{\cos(zt)}{(t^2-1)^{n+1/2}}dt.
\end{align}
From the latter equation, I can see that $Y(-i\sqrt{\Pi}r)=-Y(i\sqrt{\Pi}r)$,
and so
\begin{align}
  w_h(r)=a_1\frac{J_1(i\sqrt{\Pi}r)}{r}+a_2\frac{Y_1(i\sqrt{\Pi}r)}{r}.
\end{align}

I can re-write this in terms of the modified bessel functions of the first and
second kinds, using
\begin{align}
  J_1(ix) &= i I_1(x),\\
  Y_1(ix) &= \frac{2i}{\pi}K_1(x)-I_1(x).
\end{align}
Therefore, the homogeneous solution of eqn \ref{eq:WODE_isotropic} can be
written as
\begin{align}
  w_h(r) = c_1\frac{I_1(\sqrt{\Pi}r)}{r} + c_2\frac{K_1(\sqrt{\Pi}r)}{r}.
\end{align}
Now, I want $w_h(r\to\infty)\to0$, so $c_1=0$. This leaves
\begin{align}\label{eq:Whomo}
  w_h(r) = c_2\frac{K_1(\sqrt{\Pi}r)}{r}.
\end{align}

To uniquely solve the ODE for $w(r)$, a second boundary conditions is required.
One physically motivated option is considering the minimum distance of
interaction between particles.

Consider three particles which are all moving toward a single, central point at
the origin.
The (dimensionless) velocity of each particle is
\begin{align}\label{eq:3hardspherevelocities}
  \bm{V}_i
  &=\nabla h -\nabla_i\mathcal{H} + \lambda \bm{\hat{u}}_i\nonumber\\
  &=\lambda \bigg(\bm{\hat{u}}_i+\nabla_i\sum_{j=1}^N\sum_{k=1}^{j-1}w(r_{jk})
  \bm{r}_{jk}\cdot\bm{u}_{jk}\bigg)\nonumber\\
  &=\lambda \bigg(\bm{\hat{u}}_i-\sum_{j\neq i}^N \bigg[w'(r_{ij})
    (\bm{u}_{ij}\cdot\bm{r}_{ij})\frac{\bm{r}_{ij}}{r_{ij}}
    +w(r_{ij})\bm{u}_{ij}\bigg]\bigg),
\end{align}
where I have used eqns \ref{eq:hperturb} and \ref{eq:zetaform}. When all three
particles are hitting each other simultaneously, they should all have zero
relative velocity with respect to each other. Therefore, for this three
particle system, the boundary condition on $w(r)$ at $r_0=1$ (see appendix
\ref{app:hardsphereBC}) is
\begin{align}\label{eq:hardsphereBC}
  -1-3w'(1)-3w(1)=0.
\end{align}


Thus, inserting eqn \ref{eq:Whomo} into the boundary condition, $c_2$ is
uniquely determined and the resulting solution is
\begin{align}
  w_h(r) &= -\frac{1}{3\sqrt{\Pi}K_1'(\sqrt{\Pi})}\frac{K_1(\sqrt{\Pi}r)}{r}.
\end{align}
Since $K_1(x)>0$, $K_1'(x)<0$ for all $x>0$, this implies that $w_h(r)>0$
for all $r>0$.

Analytical and numerical solutions of eqn \ref{eq:Whomo} subject to eqn
\ref{eq:hardsphereBC} are shown in Figure \ref{fig:1}.

\begin{figure}[!t]
  \subfloat{
    \includegraphics[width=0.5\textwidth]{Figures/fig1.pdf}
  \label{fig:1a}}
  \subfloat{\label{fig:1b}}
  \caption{\protect\subref{fig:1a} Effective potential $w_h$ and
    \protect\subref{fig:1b} its gradient for hard-sphere interactions between
    active brownian particles, with $\Pi=0.1$. The numerical result has not
    been fine-tuned as it only approximately solves the boundary conditions
    (i.e. it was found by eye using a shooting method).}\label{fig:1}
\end{figure}

\subsection{Soft potential.}
For a different, soft, two-body potential such as the Weeks-Chandler-Anderson
(WCA) interaction in dimensionless form\footnote{With $\epsilon$ re-scaled
  by temperature and length in conventional LJ units, $\sigma$.}
\begin{equation}\label{eq:WCAapprox}
  \mathcal{V}^{WCA}_{ij} =
  \begin{cases}
    4\epsilon[(r_{ij}^{-12}
      -r_{ij}^{-6})]+\epsilon,
    & r<2^{1/6} \\
    0, & \mathrm{otherwise},
  \end{cases}
\end{equation}
there are two steps to computing the boundary condition. The first step is a
generic force-balance, which gives the effective ``hard sphere'' minimum
distance. Because of the more complicated nature of the interactions, this new
minimum distance will not just be $r=2^{1/6}$, but will be on the same scale.
A new energy scale (such as $\epsilon$ in eqn \ref{eq:WCAapprox}) will
be introduced, completely decoupled from $k_BT$.

\subsubsection{Force balance.}
As with the hard-sphere case, we consider a trimer of converging interacting
particles with coordinates as given in appendix \ref{app:hardsphereBC}. The
external force on the $i^{th}$ particle is just given by
$-\nabla_i\mathcal{H}$. This force must be matched by particle $i$'s active
force, i.e.
\begin{align}
  0
  &= \lambda\bm{u}_i-\nabla_i\mathcal{H}\nonumber\\
  &= \lambda\bm{u}_i+\sum_{j\neq i}\mathcal{V}'(r_{ij})
    \frac{\bm{r}_{ij}}{r_{ij}}.
\end{align}
For an isolated trimer, using the coordinates of appendix
\ref{app:hardsphereBC} one arrives at the result
\begin{align}\label{eq:genericsoftsphereforcebalance}
  \lambda+\sqrt{3}\mathcal{V}'(r_0)=0.
\end{align}
If you specifically consider the WCA approximation of eqn \ref{eq:WCAapprox},
you end up with
\begin{align}\label{eq:WCAforcebalance}
  \Omega-2r_0^{-13}+r_0^{-7}=0.
\end{align}
where $\Omega\equiv{\lambda}/{24\sqrt{3}\epsilon}$.
Now, when $\Omega\ll 1$, the minimum distance is
\begin{align}\label{eq:softsphereforcebalancesmalllambda}
  \frac{r_0^{\Omega\ll 1}}{2^{1/6}}
  \approx1+\frac{1}{21}(1-\sqrt{1+14(2)^{1/6}\Omega})
\end{align}
and when $\Omega\gg 1$, the repulsive force dominates and
\begin{align}\label{eq:softsphereforcebalancelargelambda}
  \frac{r_0^{\Omega\gg 1}}{2^{1/6}}\approx
  &(2^{7/6}\Omega)^{-1/13}\bigg\{1
  +\frac{1}{14}
  \frac{(13-7(2^{7/6}\Omega)^{-6/13})}{(13-4(2^{7/6}\Omega)^{-6/13})}
  \nonumber\\
  &\times\bigg(1-\sqrt{1+28(2^{7/6}\Omega)^{-6/13}
    \frac{(13-4(2^{7/6}\Omega)^{-6/13})}
    {(13-7(2^{7/6}\Omega)^{-6/13})^2}}\bigg)\bigg\}.
\end{align}
These approximate solutions are compared to exact (numerical
\footnote{using secant-method of root finding in SciPy})
solutions of eqn \ref{eq:WCAforcebalance} in Figure \ref{fig:2}.

\begin{figure}[!t]
  \subfloat{
    \includegraphics[width=0.5\textwidth]{Figures/fig2.pdf}
  \label{fig:2a}}
  \subfloat{\label{fig:2b}}
  \caption{\protect\subref{fig:2a} Exact (numerical) distance $r_0$ between
    spheres in a trimer at zero force (from eqn \ref{eq:WCAforcebalance}).
    \protect\subref{fig:2b} relative error between $r_0$ and its approximate
    solutions at small and large $\Omega\equiv\lambda/(24\sqrt{3}\epsilon)$,
    measured as $(r_0^{\Omega\ll1}-r_0)/r_0$ (orange dashed line) and
    $(r_0^{\Omega\gg1}-r_0)/r_0$ (green dash-dotted line), respectively.
    The approximation $r_0^{\Omega\gg1}$ becomes poor around
    $\Omega\approx0.5$, which is why the green dash-dotted line in
    \protect\subref{fig:2b} is cut off at $\Omega = 0.5$. Note that these
    results are independent of $\Pi$.}\label{fig:2}
\end{figure}
  
\subsubsection{Matching conditions for $w(r)$.}
To solve the full inhomogeneous ODE in eqn \ref{eq:WODE_isotropic}, one must
match two solutions at $r=2^{1/6}$.
The solution for $r>2^{1/6}$ is just the homogeneous
form from eqn \ref{eq:Whomo}, and will be denoted by $w_{+}(r)$. The solution
for $r<2^{1/6}$ is given through variation of parameters, denoted $w_{-}(r)$.

We require the potential to be continuous and differentiable at $r=2^{1/6}$,
finite at $r\to\infty$, and having zero relative velocity for a trimer in
the stable configuration (the same one as discussed in appendix
\ref{app:hardsphereBC}). This gives four conditions,
\begin{subequations}
  \label{eqs:genericmatchingconditions}
  \begin{align}
    &w_{+}(2^{1/6})=w_{-}(2^{1/6}),\label{eq:match1}\\
    &w_{+}'(2^{1/6}) = w_{-}'(2^{1/6}),\label{eq:match2}\\
    &w_{+}(r\to\infty)=0,\label{eq:match3}\\
    &w_{-}'(r_0)r_0+w_{-}(r_0)=-\frac{1}{3}\label{eq:match4}
  \end{align}
\end{subequations}

\subsubsection{Variation of parameters to determine $w(r)$.}
The general form of the solution to eqn \ref{eq:WODE_isotropic} for
$r<2^{1/6}$ is
\begin{align}
  w_{-}(r) = \frac{I_1(\sqrt{\Pi}r)}{r}\nu_1(r)
  +\frac{K_1(\sqrt{\Pi}r)}{r}\nu_2(r)
\end{align}
where, using eqn \ref{eq:wronskian},
\begin{subequations}
  \label{eqs:ODEcoefficientsVOP}
  \begin{align}
    \nu_1(r)
    &= k_1 - \int_{r_0}^rds\frac{K_1(\sqrt{\Pi}s)}{s}
      \frac{\mathcal{V}'(s)}{2s}W^{-1}(s)\nonumber\\
    &= k_1 + \frac{1}{2}\int_{r_0}^rdsK_1(\sqrt{\Pi}s)
      \mathcal{V}'(s)s,\nonumber\\
    \nu_2(r)
    &= k_2 + \int_{r_0}^rds\frac{I_1(\sqrt{\Pi}s)}{s}
      \frac{\mathcal{V}'(s)}{2s}W^{-1}(s)\nonumber\\
    &= k_2 - \frac{1}{2}\int_{r_0}^rdsI_1(\sqrt{\Pi}s)
      \mathcal{V}'(s)s,
  \end{align}
\end{subequations}
and we have just selected the lower limit of integration as $r_0$ for later
convenience.

Inserting this into the matching conditions allows us to determine $k_1$, and
relates $k_2$ to $c_2$,
\begin{subequations}
  \label{eqs:specificmatching}
  \begin{align}
    0 &= k_1 + \frac{1}{2}\int_{r_0}^{2^{1/6}}dsK_1(\sqrt{\Pi}s)
        \mathcal{V}'(s)s,\label{eq:specificmatch1}\\
    c_2  &= k_2 - \frac{1}{2}\int_{r_0}^{2^{1/6}}dsI_1(\sqrt{\Pi}s)
           \mathcal{V}'(s)s\label{eq:specificmatch2}.
  \end{align}
\end{subequations}
Now, inserting eqns \ref{eqs:ODEcoefficientsVOP} into eqn \ref{eq:match4},
\begin{align}
  -\frac{1}{3}=
  &r_0\bigg(-k_1\frac{I_1(\sqrt{\Pi}r_0)}{r_0^2}+k_1\sqrt{\Pi}
    \frac{I_1'(\sqrt{\Pi}r_0)}{r_0}\nonumber\\
  &-k_2\frac{K_1(\sqrt{\Pi}r_0)}{r_0^2}+k_2\sqrt{\Pi}
    \frac{K_1'(\sqrt{\Pi}r_0)}{r_0}\bigg)\nonumber\\
  & +k_1\frac{I_1(\sqrt{\Pi}r_0)}{r_0} + k_2\frac{K_1(\sqrt{\Pi}r_0)}{r_0}
    \nonumber\\
  =&\sqrt{\Pi}[k_1I_1'(\sqrt{\Pi}r_0)+k_2K_1'(\sqrt{\Pi}r_0)],
\end{align}
and so now $k_2$ is determined in terms of $k_1$. Since eqn
\ref{eq:specificmatch2} relates $k_2$ to $c_2$, all of the coefficients are
accounted for. Explicitly,
\begin{align}
  k_1 =& -\frac{1}{2}\int_{r_0}^{2^{1/6}}dsK_1(\sqrt{\Pi}s)
         \mathcal{V}'(s)s,\label{eq:k1},\\
  k_2 =& \frac{-\frac{1}{3\sqrt{\Pi}}+\frac{1}{2}\int_{r_0}^{2^{1/6}}ds
         K_1(\sqrt{\Pi}s)\mathcal{V}'(s)sI_1'(\sqrt{\Pi}r_0)}
         {K_1'(\sqrt{\Pi}r_0)}\label{eq:k2},\\
  c_2 =& \frac{-\frac{1}{3\sqrt{\Pi}}+\frac{1}{2}\int_{r_0}^{2^{1/6}}ds
         K_1(\sqrt{\Pi}s)\mathcal{V}'(s)sI_1'(\sqrt{\Pi}r_0)}
         {K_1'(\sqrt{\Pi}r_0)}\nonumber\\
       &-\frac{1}{2}\int_{r_0}^{2^{1/6}}dsI_1(\sqrt{\Pi}s)
         \mathcal{V}'(s)s\label{eq:c2}.
\end{align}

The full solution of $w(r)$ is shown in Figure \ref{fig:3} for different
values of $\Omega$ and $\Pi$, the only two free parameters which control
$w(r)$.

\begin{figure}[!t]
  \subfloat{
    \includegraphics[width=0.5\textwidth]{Figures/fig3.pdf}
  \label{fig:3a}}
  \subfloat{\label{fig:3b}}
  \caption{Effective potential $w$ for WCA interactions with
    \protect\subref{fig:3a} $\Pi=1$ held constant, and
    \protect\subref{fig:3b} $\Omega=1$ held constant. The value of $r_0$,
    when an isolated trimer experiences zero force, is shown in black dots
    for each parameter set.
  }\label{fig:3}
\end{figure}

Since the steady-state is given by $P_{ss}\propto\exp(-h)$, with
\begin{align}\label{eq:specifichfunction}
  h(\{\bm{r}_i\},\{\bm{u}_i\}) = \sum_{i=1}^N\sum_{j=1}^{i-1}\big(
  \mathcal{V}(r_{ij})+\lambda w(r_{ij})\bm{r}_{ij}\cdot\bm{u}_{ij}\big),
\end{align}
the maximum in $w(r)$ only corresponds to a minimum probability state if
$\bm{r}_{ij}\cdot\bm{u}_{ij}>0$, i.e. if the particles are moving away from
each other. Therefore, $\mathrm{max}(w)>0$ corresponds to a \textit{maximum}
probability state in our trimer configuration, since the particles are
moving toward each other. We expect that in certain parameter regimes, 
particles on average will tend to have $\bm{r}_{ij}\cdot\bm{u}_{ij}<0$ (as
observed in MIPS), and so the maximum in $w(r)$ will correspond ot a
maximum probability state.

We can see from Figure \ref{fig:3a} that when either the activity, $\lambda$,
is large with respect to the energy scale of the WCA potential, $\epsilon$,
(i.e. small values of $\Omega=\lambda/(12\sqrt{3}\epsilon$), the maximum in
$w(r)$ is pushed to smaller $r$ as the ABPs are able more effectively
penetrate the potential. It seems reasonable that $\Omega$ is a measure of
how ``soft'' the potential is, and not just the potential energy scale
$\epsilon$.

Similarly, larger values of $\Pi\propto D^r/D^t$ in Figure
\ref{fig:3b} move the maximum in $w(r)$ to smaller $r$. This is slightly
more difficult to interpret physically, as increasing rotational diffusion
will in turn change the average behaviour of $\bm{r}_{ij}\cdot\bm{u}_{ij}$.
Naively, one would think that increasing rotational diffusion would decrease
the likelihood of becoming stuck in a potential well, but Figure \ref{fig:3b}
appears to indicate that the potential becomes deeper and less wide as
$Pi$ increases. A deeper look into how the statistical mechanics of the
system, to determine how e.g. $\bm{u}_{ij}$ behaves on average, is required
to understand macroscopic behaviour.

\section{Statistical mechanics of Active Brownian particles.}

The framework of the last section is in theory enough to compute $h$ for any
system of ABPs. In this section, we compute some statistical properties of
ABPs with $h$ having the form of eqn \ref{eq:specifichfunction}.
The ``characteristic function'' of ABPs is then (see appendix
\ref{app:statmech})
\begin{widetext}
  \begin{align}\label{eq:partition}
    Z(J) =(2\pi)^N\int\prod_{i=1}^N\big(d^2r_iI_0(\lambda \tilde{b}_i(J)))
    \exp\bigg(-\frac{1}{2}\sum_{i=1}^N\sum_{j\neq i}^N
    \mathcal{V}(r_{ij})\bigg)
  \end{align}
  where we have defined
  \begin{align}\label{eq:b_i2}
    \tilde{b}_i^2(J)
    =&\sum_{i=1}^N\sum_{j\neq i}^N(w(r_{ij})+J)^2r_{ij}^2
    +\sum_{i=1}^N\sum_{j\neq i}^N\sum_{\substack{k\neq i \\ k\neq j}}^N
    (w(r_{ij})+J)(w(r_{ik})+J)\bm{r}_{ij}\cdot\bm{r}_{ik},
  \end{align}
\end{widetext}
and the integral
\begin{align}\label{eq:Iintegral}
  I_0(x)\equiv\frac{1}{\pi}\int_0^{\pi}d\phi\exp(x\cos\phi)
\end{align}
is just the order zero modified Bessel function. The partition function
is just $Z(J=0)$, and so we will take $J=0$ for the remainder of this
section, defining $b_i\equiv \tilde{b}_i(J=0)$.

From eqn \ref{eq:partition} with $J=0$, we can therefore express the
steady-state of the system for $h$ via integrating out the angular
dependence, yielding an effective potential. This steady state is
\begin{align}
  p(\{\bm{r}_i\})=\frac{1}{\tilde{Z}}\exp[-V_{\mathrm{eff}}
  (\{\bm{r}_i\})]
\end{align}
where
\begin{align}
  V_{\mathrm{eff}}(\{\bm{r}_i\})
  \equiv\frac{1}{2}\sum_{i=1}^N\sum_{j\neq i}^N
  \mathcal{V}(r_{ij})-\sum_{i=1}^N\ln I_0(\lambda b_i).
\end{align}
and $\tilde{Z}$ is 
\begin{align}
  \tilde{Z}=\int\prod_{i=1}^N(d^2r_i)\exp[-V_{\mathrm{eff}}(\{\bm{r}_i\})].
\end{align}
\subsection{Small $\lambda$ approximation to the effective potential.}
This effective potential can be expanded for small $\lambda$ as
\begin{align}
  V_{\mathrm{eff}}(\{\bm{r}_i\})
  =&\frac{1}{2}\sum_{i=1}^N\sum_{j\neq i}^N\mathcal{V}(r_{ij})%\nonumber\\
  -\sum_{i=1}^N\ln(I_0(\lambda b_i))
  \nonumber\\
  =&\frac{1}{2}\sum_{i=1}^N\sum_{j\neq i}^N\mathcal{V}(r_{ij})%\nonumber\\
  -\frac{1}{4}\sum_{i=1}^N\bigg(\lambda^2b_i^2
  -\frac{\lambda^4}{16}b_i^4+\cdots\bigg)
  \nonumber\\
  =&\frac{1}{2}\sum_{i=1}^N\sum_{j\neq i}^N\mathcal{V}(r_{ij})
  -\frac{1}{4}\lambda^2\sum_{i=1}^Nb_i^2+\mathcal{O}(\lambda^4),
\end{align}
where, consistent with above solutions of $w(r)$, only the lowest order
corrections in $\lambda$ are considered. Inserting the expression for
$b_i^2$ into this expansion motivates the definitions of effective
two-body and three-body interactions,
\begin{subequations}
  \label{eqs:2and3body}
  \begin{align}
    W_2(\bm{r}_1,\bm{r}_2;\lambda)
    &= \mathcal{V}(r_{12})
      -\frac{1}{2}\lambda^2 r_{12}^2w^2(r_{12}),\label{eq:2body}\\
    W_3(\bm{r}_1,\bm{r}_2,\bm{r}_3;\lambda)
    &= -\frac{3!}{4}\lambda^2w(r_{12})w(r_{13})\bm{r}_{12}\cdot\bm{r}_{13},
      \label{eq:3body}
  \end{align}
\end{subequations}
which allow us to write the effective potential as
\begin{align}\label{eq:Vthreebody}
  V_{\mathrm{eff}}(\{\bm{r}_i\};\lambda)
  &=\frac{1}{2}\sum_{i,j;j\neq i}W_2(\bm{r}_i,\bm{r}_j)
    +\frac{1}{3!}\sum_{\substack{i,j,k;j\neq i \\k\neq i k\neq j}}
    W_3(\bm{r}_i,\bm{r}_j,\bm{r}_k)\nonumber\\
  &=\sum_{j<i}W_2(\bm{r}_i,\bm{r}_j)
    +\sum_{k<j<i}
    W_3(\bm{r}_i,\bm{r}_j,\bm{r}_k)
\end{align}

\subsection{Two-body approximation to potential.}
If one instead assumes that only two-body interactions are important for
measuring thermodynamic quantities, i.e. only the pair correlation function
\footnote{also known as the radial distribution function} contributes, then
the effective potential can instead by written as (see appendix
\ref{app:onlytwobody})
\begin{align}
  V_{\mathrm{eff}}(\{\bm{r}_i\})
  =&\frac{1}{2}\sum_{i=1}^N\sum_{j\neq i}^N\mathcal{V}(r_{ij})
  -\sum_{i=1}^N\sum_{j\neq i}^N\ln(I_0(\lambda w(r_{ij})r_{ij}))
\end{align}
which now motivates a second effective two body interaction,
\begin{align}\label{eq:Vfulltwobody}
  \chi_2(r_{12};\lambda)
  &= \mathcal{V}(r_{12})-2\ln(I_0(\lambda w(r_{12})r_{12})).
\end{align}

\subsection{Pressure calculations.}

\subsubsection{Two-body pressure.}
In order to check whether the two-body corrections are accurate, one can
compute the pressure $P$ (see appendix \ref{app:pressure}) using the pair
correlation function (obtained via simulation), 
\begin{align}
  \beta P
  =& \frac{N}{A}
  -\frac{\pi N^2}{2A^2}
  \int_0^{\infty} du V_{2body}'(u)u^2 g^{(2)}(u),
\end{align}
where $V_{2body}$ is either the two body interaction $W_2$ defined in
eqn \ref{eq:2body} or $\chi_2$ defined in eqn \ref{eq:Vfulltwobody}.

\subsection{Correlation functions.}

\subsubsection{Global correlation functions.}

We can compute the global correlation function
\begin{align}
  \bigg<\sum_{i,j;j\neq i} \bm{u}_{ij}\cdot\bm{r}_{ij}
  \bigg>,
\end{align}
by first noting that the characteristic function defined in eqn
\ref{eq:partition} generates the moments of
$\bm{u}_{ij}\cdot\bm{r}_{ij}$ (see appendix \ref{app:statmech})

%%%%%%%%%%%% begin appendix %%%%%%%%%%%%%


\appendix

\section{Fokker-Planck equations\label{app:fokkerplanck}}

\subfile{appendices/fokker_planck_equations.tex}
\section{Hard sphere boundary conditions \label{app:hardsphereBC}}

\subfile{appendices/hard_sphere_BCs.tex}
     
\section{Full solution to w(r). \label{app:fullWsolution}}

\subfile{appendices/full_solution_w.tex}

\section{Partition function of ABPs. \label{app:statmech}}

\subfile{appendices/partition_function_details.tex}

\section{Equilibrium pressure in terms of correlation
 functions.\label{app:pressure}}

\subfile{appendices/correlation_functions_and_pressure.tex}

\section{Non-equilibrium pressure for ABPs.}

\subfile{appendices/non_eq_pressure.tex}

\section{$b_i^n$}
\subfile{appendices/b_i_expansion.tex}

\section{Pressure calculation from virial expansion}

\subfile{appendices/virial_expansion.tex}


\bibliographystyle{plain}
\bibliography{main}

\end{document}
