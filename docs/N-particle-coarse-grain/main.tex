\documentclass{article}
\usepackage[hmargin=1.25cm,vmargin=2.5cm]{geometry}
\usepackage[utf8]{inputenc}
\usepackage{bm}
\usepackage{amsmath}
\usepackage{amssymb}
\usepackage{comment}
\usepackage{graphicx}

\title{calculation: coarse-graining active brownian particles}
\author{Sam Cameron}
\date{October 2019}

\begin{document}

\maketitle

\section{Writing the active brownian particles equations of motion into the correct form}



The equations of motion for active brownian particles are:

\begin{align}
  \frac{d\bm{r}_i}{dt}&=\frac{D^t}{k_BT}\big(\bm{F}_i
                        +f^P\bm{P}_i\big)
                        +\sqrt{2D^t}\tilde{\bm{u}}_i^t,\label{eq:micro_pos}\\
    \frac{d\theta_i}{dt}&=\sqrt{2D^r}w_i.\label{eq:micro_theta}
\end{align}
where $\bm{r}_i=(x_i,y_i)$ are the particle positions,
$\bm{P}_i=(\cos\theta_i,\sin\theta_i)$ are the particle orientations,
and $\tilde{\bm{u}}_i=(\tilde{u}_i,v_i)$, $w_i$ are Gaussian white noise
with zero mean and unit variance, i.e. for any two components of the vector
$\bm{s}=(u_i,v_i,w_i)$,
\begin{align}
  \big<s_i(t)\big>=0;\;\;\;\;\; \big<s_i(t)s_j(t^{\prime})\big>=
  \delta_{ij}\delta(t-t^{\prime}).
\end{align}.

Now, I can define a vector of length $3N$ with components
\begin{align}
  \bm{r}=(\bm{r}^{(1)},\bm{r}^{(2)},\cdots,\bm{r}^{(N)},\theta_1,\theta_2,\cdots,
  \theta_N).
\end{align}

This will then give the differential equation
\begin{align}
  \frac{d\bm{r}}{dt}=\bm{F}(\bm{r})+\bm{v}(\bm{r})
  +\sqrt{2\tilde{D}_{\alpha\beta}}u_{\beta}(t),
\end{align}
where $\tilde{D}_{\alpha\beta}$ is diagonal, and the values of the diagonal are
$D^t$ for the first $2N$ terms, and $D^r$ for the last $N$ terms.

Next, I will define
\begin{align}
  \bm{x}&\equiv(x_1,x_2,\cdots,x_{3N}),\\
  \nabla&\equiv(\partial_1,\partial_2,\cdots,\partial_{3N}).
\end{align}

as well as the stochastic probability density operator
\begin{align}
  \hat{\rho}(\bm{x},t)=\delta(\bm{x}-\bm{r}(t)).
\end{align}

Taking the partial derivative of this with respect to time gives
\begin{align}
  \frac{\partial\hat{\rho}}{\partial t}&=-(\nabla_{\bm{x}}\delta(\bm{x}-\bm{r}(t))
                                         \cdot\frac{d\bm{r}}{dt}\nonumber\\
                                       &=-\nabla_{\bm{x}}\cdot\bigg(\frac{d\bm{r}}{dt}
                                         \delta(\bm{x}-\bm{r}(t)\bigg)\nonumber\\
                                       &=-\nabla_{\bm{x}}\cdot\hat{\bm{j}}(\bm{x},t),
\end{align}
where I have defined the stochastic current operator
\begin{align}
  \hat{\bm{j}}(\bm{x},t)&\equiv\bm{F}(\bm{x})\delta(\bm{x}-\bm{r}(t))
                          +\bm{v}(\bm{x})\delta(\bm{x}-\bm{r}(t))
                          +\sqrt{2\tilde{D}_{\alpha\beta}}u_{\beta}(t)
                          \delta(\bm{x}-\bm{r}(t))\nonumber\\
                        &=\bm{F}(\bm{x})\hat{\rho}(\bm{x},t)
                          +\bm{v}(\bm{x})\hat{\rho}(\bm{x},t)
                          +\sqrt{2\tilde{D}_{\alpha\beta}}u_{\beta}(t)\hat{\rho}(\bm{x},t).
\end{align}

You can average $\hat{\rho}(\bm{x},t)$ over the fluctuations $\bm{u}(t)$, and get the
resulting density
\begin{align}
  P(\bm{x},t)=<\hat{\rho}(\bm{x},t)>,
\end{align}
where I have used the definition that for some observable $\hat{\mathcal{O}}$,
\begin{align}
  <\hat{\mathcal{O}}(\bm{u}(t))>=\frac{1}{Z}\int\mathcal{D}\bm{u}
  \hat{\mathcal{O}}(\bm{u}(t))\exp\bigg(-\frac{1}{2}\int_0^Tdt(\bm{u}(t))^2\bigg).
\end{align}
From this, I can calculate that noise-averaged current as
\begin{align}
  \bm{J}(\bm{x},t)=\bm{F}(\bm{x})P(\bm{x},t)+\bm{v}(\bm{x})P(\bm{x},t)
  +\sqrt{2\tilde{D}_{\alpha\beta}}<u_{\beta}(t)\hat{\rho}(\bm{x},t)>.
\end{align}

In order to write the noise term above in terms of $P(\bm{x},t)$, I will use the properties
of the Dirac-delta function, namely,
\begin{align}
  \delta(\bm{x})=\int\frac{d^dk}{(2\pi)^d}\exp(-ik_jx_j),
\end{align}
to write
\begin{align}
  <u_{\beta}\hat{\rho}(\bm{x},t)>=\int\frac{d^dk}{(2\pi)^d}u_{\beta}(t)
  \exp(-ik_j(x_j-r_j(t))).
\end{align}

Since correlations between the noise
$<u_{\alpha}(t)u_{\beta}(t)>=\delta_{\alpha\beta}\delta(t-t^{\prime})$ only occur for
fluctuations happening at the same time, we can write
\begin{align}
  \bm{r}(t)=\bm{r}(t-\Delta t)+\int_{t-\Delta t}^tdt_1[\bm{F}(\bm{r}(t_1))
  +\bm{v}{\bm{r}(t_1)}+\sqrt{2\tilde{D}_{j\gamma}}u_{\gamma}(t_1)]
\end{align}
as $\Delta t\to0$. Then, since
\begin{align}
  \exp\bigg[\int_{t-\Delta t}^tdt_1(\bm{F}(\bm{r}(t_1)
  +\bm{v}({\bm{r}(t_1)})\bigg]=1+O(\Delta t),
\end{align}
this can be written as
\begin{align}
  <u_{\beta}(t)\exp(-ik_jr_j(t)>=&\bigg<u_{\beta}(t)\exp\bigg[-ik_jr_j(t-\Delta t)-ik_j
                                   \int_{t-\Delta t}^tdt_1(F_j(\bm{r}(t_1)
                                   +v_j{\bm{r}(t_1)}\nonumber\\
                                 &+\sqrt{2\tilde{D}_{j\gamma}}u_{\gamma}
                                   (t_1))\bigg]\bigg>\nonumber\\
  =&\bigg<u_{\beta}(t)\exp(-ik_jr_j(t-\Delta t))
     \bigg[1-ik_j\int_{t-\Delta t}^tdt_1
     \sqrt{2\tilde{D}_{j\gamma}}u_{\gamma}(t_1)\bigg]\bigg>+O(\Delta t).
\end{align}

Next, since $t-\Delta t<t_1,t$, there won't be any correlations between the
fluctuations at earlier times and those at $t$, so using the fact that for uncorrelated
functions $<AB>=<A><B>$, I get
\begin{align}
  <u_{\beta}(t)\exp(-ik_jr_j(t)>&=\bigg<\exp(-ik_jr_j(t-\Delta t))\bigg>
                                  \bigg<u_{\beta}(t)\bigg[1-ik_j\int_{t-\Delta t}^tdt_1
                                  \sqrt{2\tilde{D}_{j\gamma}}u_{\gamma}(t_1)\bigg]\bigg>
                                  +O(\Delta t)\nonumber\\
                                &=\bigg<\exp(-ik_jr_j(t-\Delta t))\bigg>
                                  \bigg(<u_{\beta}(t)>
                                  -\bigg<ik_j\int_{t-\Delta t}^tdt_1
                                  \sqrt{2\tilde{D}_{j\gamma}}u_{\gamma}(t_1)
                                  u_{\beta}(t)\bigg]\bigg>\bigg)
                                  +O(\Delta t)\nonumber\\
                                &=-\bigg<ik_j\sqrt{2\tilde{D}_{j\gamma}}\int_{t-\Delta t}^tdt_1
                                  \delta_{\beta \gamma}\delta(t-t_1)\bigg]\bigg>
                                  \bigg<\exp(-ik_jr_j(t-\Delta t))\bigg>
                                  +O(\Delta t)\nonumber\\
                                &=-\bigg<ik_j\sqrt{2\tilde{D}_{j\beta}}\int_{t-\Delta t}^tdt_1
                                  \delta(t-t_1)\bigg>
                                  \bigg<\exp(-ik_jr_j(t-\Delta t))\bigg>
                                  +O(\Delta t)\nonumber\\
                                &=-\bigg<ik_j\sqrt{2\tilde{D}_{j\beta}}\int_0^{\Delta t}du
                                  \delta(u)\bigg>
                                  \bigg<\exp(-ik_jr_j(t-\Delta t))\bigg>
                                  +O(\Delta t)\nonumber\\
                                &=-\frac{1}{2}ik_j\sqrt{2\tilde{D}_{j\beta}}
                                  \bigg<\exp(-ik_jr_j(t-\Delta t))\bigg>
                                  +O(\Delta t)\nonumber\\.
\end{align}
Taking the limit as $\Delta t\to0$, this becomes
\begin{align}
  <u_{\beta}(t)\exp(-ik_jr_j(t)>=-\frac{1}{2}ik_j\sqrt{2\tilde{D}_{j\beta}}
  \bigg<\exp(-ik_jr_j(t))\bigg>.
\end{align}
Taking the inverse Fourier transform then gives
\begin{align}
  <u_{\beta}(t)\hat{\rho}(\bm{x},t)>=-\frac{1}{2}\sqrt{2\tilde{D}_{j\beta}}
  \partial_jP(\bm{x},t).
\end{align}
Finally, inserting this into the current I get
\begin{align}
  J_{\alpha}(\bm{x},t)&=F_{\alpha}(\bm{x})P(\bm{x},t)+v_{\alpha}(\bm{x})P(\bm{x},t)
                        +\sqrt{2\tilde{D}_{\alpha\beta}}(-\frac{1}{2}\sqrt{2\tilde{D}_{j\beta}}
                        \partial_jP(\bm{x},t))\nonumber\\
                      &=F_{\alpha}(\bm{x})P(\bm{x},t)+v_{\alpha}(\bm{x})P(\bm{x},t)
                        -\sqrt{\tilde{D}_{\alpha\beta}\tilde{D}_{j\beta}}
                        \partial_jP(\bm{x},t))
\end{align}

Since in our case, $\tilde{D}_{\alpha\beta}$ is diagonal, this just becomes
\begin{align}\label{eq:Nparticlecurrent}
  \bm{J}(\bm{x},t)&=\bm{F}(\bm{x})P(\bm{x},t)+\bm{v}(\bm{x})P(\bm{x},t)
                    -\bm{\tilde{D}}\cdot\nabla P(\bm{x},t).
\end{align}

Finally, I get the desired result of the section, which is
\begin{align}\label{eq:NparticleFP}
  \frac{\partial P}{\partial t} + \nabla_{\bm{x}}\cdot\big(\bm{F}(\bm{x})P(\bm{x},t)
  +\bm{v}(\bm{x})P(\bm{x},t)-\bm{\tilde{D}}\cdot\nabla P(\bm{x},t)\big)=0
\end{align}


\section{Deriving a one-particle density function.}

In this section, I am interested in determining the one particle density function, i.e. in
terms of $\bm{r}_1$ (not the angular components). To start with, I will remind myself that
\begin{align}
  P(\bm{x},t)\equiv P(\bm{r}_1,\bm{r}_2,\cdots,\bm{r}_N,\theta_1,\theta_2,\cdots,\theta_N,t)
\end{align}

I will then split the the current of eqn \ref{eq:Nparticlecurrent} into two parts, a part which
cannot be written as a gradient term, and a part that can. The part that cannot be written as
a gradient term is just
\begin{align}\label{eq:dissipative_force}
  \bm{v}(\bm{x})&=\frac{D^t}{k_BT}f^P(\cos\theta_1,\sin\theta_1,\cos\theta_2,
                  \sin\theta_2,\cdots,\cos\theta_N,\sin\theta_N,0,0,\dots,0)^T\nonumber\\
                &=\frac{1}{k_BT}\bm{\tilde{D}}\cdot \bm{u}(\{\theta_i\})
\end{align}

Before writing out the gradient term, I will first note that I am considering just two-body
interactions,
\begin{align}\label{eq:general_conservative_force}
  \bm{F}(\bm{x})&=-\nabla_{\bm{x}}\mathcal{H}(\bm{x})\nonumber\\
                &=-\nabla_{\bm{x}}\sum_{i<j}^N\mathcal{V}(r_{ij}),
\end{align}
where $r_{ij}=|\bm{r}_i-\bm{r}_j|$.

Then, I can re-write eqn \ref{eq:Nparticlecurrent} as
\begin{align}\label{eq:Nparticlewithpotential}
  \bm{J}(\bm{x},t)&=-\bm{\tilde{D}}
                    \cdot\big(\nabla_{\bm{x}}(\frac{1}{k_BT}\mathcal{H}(\{\bm{r}_i\})
                    P(\bm{x},t)
                    +P(\bm{x},t))
                    -\frac{1}{k_BT}\bm{u}(\{\theta_i\})P(\bm{x},t)\big).
\end{align}

I will now redefine
\begin{align}
  \bm{D}=\beta\bm{\tilde{D}},
\end{align}
where $\beta^{-1}=k_BT$.


Finally, I can write the Fokker-Planck equation as
\begin{align}
  \frac{\partial P}{\partial t}=\nabla_{\bm{x}}\cdot\big(\bm{D}
  \big[\beta^{-1}\nabla_{\bm{x}}P(\bm{x},t)+P(\bm{x},t)(\nabla_{\bm{x}}
  \mathcal{H}(\{\bm{r}_i\})-\bm{v}(\{\theta_i\}))\big]\big)
\end{align}

\subsection{Steady state calculation.}

From \cite{liverpool2018nonequilibrium}, I then know that there exists a generalized steady
state of the form
\begin{align}
  P_{ss}=\exp(-h(\bm{x}))
\end{align}
where $h(\bm{x})$ satisfies the equation
\begin{align}\label{eq:ss_condition}
  \sum_{i=1}^{3N}D_iL_i(h)=\bm{0},
\end{align}
where $D_i$ are the (diagonal) components of $\bm{D}$,
\begin{align}
  D_i =
  \begin{cases}
    D^t\beta,&i\leq 2N,\\
    D^r\beta,&i>2N
  \end{cases}
\end{align}
and
\begin{align}\label{eq:L_i}
  L_i(h) = \beta^{-1}(\partial_ih)^2+\partial_i^2\mathcal{H}
  +(v_i-\partial_i\mathcal{H})\partial_ih-\beta^{-1}\partial_i^2h-\partial_iv_i.
\end{align}

I can explicitly calculate the terms not involving $h$ in eqn \ref{eq:L_i}, using eqns
\ref{eq:general_conservative_force} and eqn \ref{eq:dissipative_force}. I do this now:
\begin{subequations}
  \label{eqs:partials_of_L_i}
  \begin{align}
    \partial_{\alpha}\mathcal{H} &=
                                   \begin{cases}
                                     \sum_{i\neq\alpha}^N\mathcal{V}^{\prime}
                                     (r_{\alpha i})\frac{x_{\alpha}-x_i}{r_{\alpha i}},
                                     & \alpha\: \mathrm{odd},\:\alpha\leq2N,\\
                                     \sum_{i\neq\alpha}^N\mathcal{V}^{\prime}
                                     (r_{\alpha i})\frac{y_{\alpha}-y_i}{r_{\alpha i}},
                                     & \alpha\: \mathrm{even},\:\alpha\leq2N,\\
                                     0 & \alpha>2N
                                   \end{cases}\label{eq:partial_H_of_L_i}\\
    \partial_{\alpha}^2\mathcal{H} &=
                                     \begin{cases}
                                       \sum_{i\neq\alpha}^N\bigg[\frac{\mathcal{V}^{\prime}
                                         (r_{\alpha i})}{r_{\alpha i}}
                                       +\frac{(x_{\alpha}-x_i)^2}{r_{\alpha i}^2}
                                       \bigg(-\frac{\mathcal{V}^{\prime}(r_{\alpha i})}
                                       {r_{\alpha i}}+\mathcal{V}^{\prime\prime}
                                       (r_{\alpha i})\bigg)\bigg],
                                       & \alpha\: \mathrm{odd},\:\alpha\leq2N,\\
                                       \sum_{i\neq\alpha}^N\bigg[\frac{\mathcal{V}^{\prime}
                                         (r_{\alpha i})}{r_{\alpha i}}
                                       +\frac{(y_{\alpha}-y_i)^2}{r_{\alpha i}^2}
                                       \bigg(-\frac{\mathcal{V}^{\prime}(r_{\alpha i})}
                                       {r_{\alpha i}}+\mathcal{V}^{\prime\prime}
                                       (r_{\alpha i})\bigg)\bigg],
                                       & \alpha\: \mathrm{even},\:\alpha\leq2N,\\
                                       0 & \alpha>2N
                                     \end{cases}\label{eq:partialsq_H_of_L_i}\\
    \partial_i v_i &= 0\label{eq:partial_v_of_L_i}
  \end{align}
\end{subequations}

From these results I can see immediately that eqn \ref{eq:ss_condition} simplifies to
\begin{align}\label{eq:steadystatecondition}
  D^t\beta\sum_{i=1}^{2N}L_i(h)+D^r\sum_{i=1}^{N}\big[(\partial_{\theta_i}h)^2
  -\partial_{\theta_i}^2h\big]=0
\end{align}

To actually do any calculations, an interaction form for $\mathcal{H}$ must be chosen. Since we
are interested in capturing motility induced phase separation (MIPS) in the simplest form, I
will choose the (repulsive) Weeks-Chandler-Anderson (WCA) interaction,
\begin{equation}
  \mathcal{V}_{ij} =
  \begin{cases}
    4\epsilon\bigg[\bigg(\frac{\sigma}{r^{ij}}\bigg)^{12}
    -\bigg(\frac{\sigma}{r^{ij}}\bigg)^6\bigg]+\epsilon, & r<2^{\frac{1}{6}}\sigma \\
    0, & \mathrm{otherwise}.
  \end{cases}
\end{equation}


\subsection{Coarse-graining the steady $h(\bm{x})$ function.}

To solve eqn \ref{eq:steadystatecondition}, I will start by writing a guess for $h(\bm{x})$, and I
will take it to be of the form
\begin{align}\label{eq:hgeneric}
  h(\bm{x})=R(\{\bm{r}_i\})\Theta(\{\theta_i\})
\end{align}
where $R(\{\bm{r}_i\})$ and $\Theta(\{\theta_i\})$ are functions which I will determine later.

Now, I will assume that the system has translational symmetry, so
\begin{align}\label{eq:Rtranssym}
  R(\{\bm{r}_i+\bm{T}\}=R(\{\bm{r}_i\})
\end{align}
for any arbitrary translational vector $\bm{T}$. This justifies writing $h(\bm{x})$ as a separable
function as in eqn \ref{eq:hgeneric}. I will also impose full rotational symmetry, so that
\begin{align}\label{eq:hrotsymm}
  R(\{\bm{Q}(\phi)\bm{r}_i\})\Theta(\{\theta_i+\phi\})=R(\{\bm{r}_i\})\Theta(\{\theta_i\}),
\end{align}
where $Q(\phi)$ is a tensor which rotates a vector by some arbitrary angle, $\phi$.

\subsubsection{$R(\{\bm{r}_i\})$ function.}

I will focus on $R(\{\bm{r}_i\})$ to start with. By eqn \ref{eq:Rtranssym}, I know that
$a(\{\bm{r}_i\})=\mathrm{constant}$. From eqns \ref{eq:Rtranssym} and \ref{eq:hrotsymm}, I know
that $b(\bm{r}_i,\bm{r}_j)=b(r_{ij})$. I can derive similar results for higher order
terms, but will focus only on second order, and so
\begin{align}
  R(\{\bm{r}_k\})= \sum_{i=1}^N\sum_{j=1}^{i-1}b(r_{ij})
\end{align}

I will take a variational guess to the form of $b(r_{ij})$, and just keep at as a standard
Lennard-Jones effective potential
\begin{align}
  b(r_{ij})\equiv 4\eta\bigg[\bigg(\frac{\delta}{r_{ij}}\bigg)^{12}-
  \bigg(\frac{\delta}{r_{ij}}\bigg)^6\bigg]
\end{align}
where $\eta$ and $\delta$ are effective parameters for the effective interaction. Note that if
$f_P=0$ (i.e. no activity), then $\eta=\epsilon$ and $\delta=\sigma$.

\subsubsection{$\Theta(\{\theta_i\})$ function.}

The only requirement of $\Theta(\{\theta_i\})$ is that it is invariant under arbitrary rotations
of all angles by some constant $\phi$. So, to lowest order this would be
\begin{align}
  \Theta(\{\theta_k\})=\sum_{i=1}^N\sum_{j=1}^{i-1}c(\theta_i-\theta_j).
\end{align}

\subsubsection{Starting with the two particle case.}

If I just take $N=2$, then eqn \ref{eq:steadystatecondition} becomes
\begin{align}
  &D^t\bigg\{c(\theta_1-\theta_2)^2\bigg[\bigg(b'(r_{12})\frac{(x_1-x_2)}{r_{12}}\bigg)^2
    +\bigg(b'(r_{12})\frac{(y_1-y_2)}{r_{12}}\bigg)^2
    +\bigg(b'(r_{21})\frac{(x_2-x_1)}{r_{21}}\bigg)^2
    +\bigg(b'(r_{21})\frac{(y_2-y_1)}{r_{21}}\bigg)^2\bigg]\nonumber\\
  &+\beta\bigg[\frac{\mathcal{V}'(r_{12})}{r_{12}}
    +\frac{(x_1-x_2)^2}{r_{12}^2}\bigg(\mathcal{V}''(r_{12})
    -\frac{\mathcal{V}'(r_{12})}{r_{12}}\bigg)
    +\frac{\mathcal{V}'(r_{12})}{r_{12}}
    +\frac{(y_1-y_2)^2}{r_{12}^2}\bigg(\mathcal{V}''(r_{12})
    -\frac{\mathcal{V}'(r_{12})}{r_{12}}\bigg)\nonumber\\
  &+\frac{\mathcal{V}'(r_{21})}{r_{21}}
    +\frac{(x_2-x_1)^2}{r_{21}^2}\bigg(\mathcal{V}''(r_{12})
    -\frac{\mathcal{V}'(r_{12})}{r_{12}}\bigg)
    +\frac{\mathcal{V}'(r_{12})}{r_{12}}
    +\frac{(y_1-y_2)^2}{r_{12}^2}\bigg(\mathcal{V}''(r_{12})
    -\frac{\mathcal{V}'(r_{12})}{r_{12}}\bigg)\bigg]\nonumber\\
  &+\beta c(\theta_1-\theta_2)\bigg[
    \bigg(f^P\cos\theta_1-\mathcal{V}'(r_{12})\frac{(x_1-x_2)}{r_{12}}\bigg)
    b'(r_{12})\frac{(x_1-x_2)}{r_{12}}
    +\bigg(f^P\sin\theta_1-\mathcal{V}'(r_{12})\frac{(y_1-y_2)}{r_{12}}\bigg)
    b'(r_{12})\frac{(y_1-y_2)}{r_{12}}\nonumber\\
  &+\bigg(f^P\cos\theta_2-\mathcal{V}'(r_{21})\frac{(x_2-x_1)}{r_{21}}\bigg)
    b'(r_{21})\frac{(x_2-x_1)}{r_{21}}
    +\bigg(f^P\sin\theta_2-\mathcal{V}'(r_{21})\frac{(y_2-y_1)}{r_{21}}\bigg)
    b'(r_{21})\frac{(y_2-y_1)}{r_{21}}\bigg]\nonumber\\
  &-c(\theta_1-\theta_2)\bigg[\frac{b'(r_{12})}{r_{12}}
    +\frac{(x_1-x_2)^2}{r_{12}^2}\bigg(b''(r_{12})
    -\frac{b'(r_{12})}{r_{12}}\bigg)
    +\frac{b'(r_{12})}{r_{12}}
    +\frac{(y_1-y_2)^2}{r_{12}^2}\bigg(b''(r_{12})
    -\frac{b'(r_{12})}{r_{12}}\bigg)\nonumber\\
  &+\frac{b'(r_{21})}{r_{21}}
    +\frac{(x_2-x_1)^2}{r_{21}^2}\bigg(b''(r_{12})
    -\frac{b'(r_{12})}{r_{12}}\bigg)
    +\frac{b'(r_{12})}{r_{12}}
    +\frac{(y_1-y_2)^2}{r_{12}^2}\bigg(b''(r_{12})
    -\frac{b'(r_{12})}{r_{12}}\bigg)\bigg]\bigg\}\nonumber\\
  &+D^r\bigg\{\big(b(r_{12})\big)^2\bigg[\bigg(\frac{\partial c}{\partial \theta_1}\bigg)^2
    +\bigg(\frac{\partial c}{\partial \theta_2}\bigg)^2\bigg]
    -b(r_{12})\bigg[\frac{\partial^2 c}{\partial \theta_1^2}
    +\frac{\partial^2 c}{\partial \theta_2^2}\bigg]\bigg\}=0.
\end{align}
which can be written after some simplification as
\begin{align}
  &D^t\bigg\{2(b'(r_{12}))^2c(\theta_1-\theta_2)^2
    +2\beta\bigg(\frac{\mathcal{V}'(r_{12})}{r_{12}}
    +\mathcal{V}''(r_{12})\bigg)
    -2\beta c(\theta_1-\theta_2)b'(r_{12})\mathcal{V}'(r_{12})
    -2c(\theta_1-\theta_2)\bigg(\frac{b'(r_{12})}{r_{12}}
    +b''(r_{12})\bigg)\bigg\}\nonumber\\
  &+D^t\beta c(\theta_1-\theta_2)b'(r_{12})f^P
    \bigg\{\big(\cos\theta_1-\cos\theta_2)\frac{x_1-x_2}{r_{12}}
    +\big(\sin\theta_1-\sin\theta_2)\frac{y_1-y_2}{r_{12}}\bigg\}\nonumber\\
  &+D^r\bigg\{\big(b(r_{12})\big)^2\bigg[\bigg(\frac{\partial c}{\partial \theta_1}\bigg)^2
    +\bigg(\frac{\partial c}{\partial \theta_2}\bigg)^2\bigg]
    -b(r_{12})\bigg[\frac{\partial^2 c}{\partial \theta_1^2}
    +\frac{\partial^2 c}{\partial \theta_2^2}\bigg]\bigg\}=0.
\end{align}

From this calculation, it appears that even for a two-particle, interacting system with activity
(i.e. $f^P\neq0$), you have to consider a non-constant angular distribution function
$c(\theta_1-\theta_2)$ in order to do some sort of matching between the theta terms. Also, this
equation is making me second guess whether rotation of particle pairs should not change the energy...




%%%%%%%%%%%%%%%%%%%%%%%%%%%%%%%%%%%%%%%%
% Below this, I have an old calculation from before I really understood what to do... %
%%%%%%%%%%%%%%%%%%%%%%%
\begin{comment}
Then, the full interaction potential is
\begin{align}
    \mathcal{H}=\sum_{\alpha=1}^N\sum_{\beta=1}^{\alpha-1}\mathcal{V}^{\alpha\beta},
\end{align}
where I am summing over the $N(N-1)/2$ different pairings of the positions.
Noting that
\begin{equation}\label{eq:dVij}
    \frac{\partial \mathcal{V}^{\alpha\beta}}{\partial r^{\alpha\beta}}=
    \begin{cases}
    -24\epsilon\bigg[\bigg(\sigma^{12}(r^{\alpha\beta})^{-13}-\sigma^6(r^{\alpha\beta})^{-6}\bigg], & r^{\alpha\beta}<2^{\frac{1}{6}}\sigma \\
    0, & \mathrm{otherwise},
    \end{cases}
\end{equation}
and
\begin{align}\label{eq:drij}
    \frac{\partial r^{\alpha\beta}}{\partial x_k}&=\frac{(x_{\alpha}-x_{\beta})(\delta_{k\alpha}-\delta_{k\beta})}{r^{\alpha\beta}},
\end{align}
the gradient of the interaction potential is
\begin{equation}
    \frac{\partial}{\partial x_k}\mathcal{H}=
    \begin{cases}
    \sum\limits_{\substack{\beta=1\\ \beta\neq k}}^N Q_{k\beta}, &1\leq k \leq N,\\
    \sum\limits_{\substack{\beta=N+1\\ \beta\neq k}}^{2N} Q_{k\beta}, &N+1\leq k \leq 2N,\\
    0, & k > 2N
    \end{cases}
\end{equation}
where I have defined $Q_{\alpha\beta}=-Q_{\beta\alpha}$ is
\begin{align}
    Q_{\alpha\beta}=\frac{\partial\mathcal{V}^{\alpha\beta}}{\partial r^{\alpha\beta}}\times\frac{(x_{\alpha}-x_{\beta})}{r^{\alpha\beta}}
\end{align}

\end{comment}

\bibliographystyle{plain}
\bibliography{bib}

\end{document}