\documentclass[twocolumn,amsmath,amssymb,aps]{revtex4-1}%{article}
%\usepackage[hmargin=1.25cm,vmargin=2.5cm]{geometry}
\usepackage[utf8]{inputenc}
\usepackage{bm}
%\usepackage{amsmath}
%\usepackage{amssymb}
\usepackage{comment}
\usepackage{graphicx}
\usepackage{dcolumn}
\usepackage[caption=false]{subfig}
\usepackage{subfiles}

\begin{document}
 
\title{calculation: coarse-graining active brownian particles}
\author{Sam Cameron}
\affiliation{%
  University of Bristol}%
\date{\today}

\begin{abstract}
  In equilibrium systems, collective properties are calculated
  through knowledge of the Boltzmann probability density,
  $e^{-\beta \mathcal{H}(\{\bm{x}_i)\}}$ and the corresponding partition
  function. In contrast, non-equilibrium/active systems do not obey
  Boltzmann statistics, and so the functional form of the steady-state
  probability density (if it exists) is not known a priori. In this letter,
  we present a systematic way of calculating the steady-state
  probability density of an active systems. As a proof of concept, we use
  this method to calculate the steady-state probability density of
  interacting active brownian particles. Finally, we compare our calculation
  to Molecular Dynamics simulations.
\end{abstract}


\maketitle

\section{Equilibrium pressure in terms of correlation
 functions.\label{app:pressure}}

\subfile{appendices/correlation_functions_and_pressure.tex}

\section{Non-equilibrium pressure for ABPs.}

\subfile{appendices/non_eq_pressure.tex}

\bibliographystyle{plain}
\bibliography{main}

\end{document}
