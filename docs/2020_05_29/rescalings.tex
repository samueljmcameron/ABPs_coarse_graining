\documentclass[twocolumn,amsmath,amssymb,aps]{revtex4-1}%{article}
%\usepackage[hmargin=1.25cm,vmargin=2.5cm]{geometry}
\usepackage[utf8]{inputenc}
\usepackage{bm}
%\usepackage{amsmath}
%\usepackage{amssymb}
\usepackage{comment}
\usepackage{graphicx}
\usepackage{dcolumn}
\usepackage[caption=false]{subfig}
\usepackage{braket}
\usepackage{subfiles}

\begin{document}
 
\title{May 29, 2020: LJ units vs WCA units.}
\author{Sam Cameron}
\affiliation{%
  University of Bristol}%
\date{\today}

\begin{abstract}
  Writing the re-scalings out between measuring things in LJ units
  vs in WCA units. The only difference is the definition of the
  microscopic length scale.
\end{abstract}


\maketitle

\section{LJ units vs WCA units.}

The WCA potential is written as
\begin{align*}\label{eq:WCAapprox}
  \mathcal{V}^{WCA}_{ij}
  &=
  \begin{cases}
    4\epsilon\bigg[\bigg(\frac{\sigma}{r_{ij}}\bigg)^{-12}
      -\bigg(\frac{\sigma}{r_{ij}}\bigg)^{-6})\bigg]+\epsilon,
    & r<2^{1/6}\sigma, \\
    0, & \mathrm{otherwise},
  \end{cases}\nonumber\\
  &=
  \begin{cases}
    \epsilon\bigg[\bigg(\frac{r_m}{r_{ij}}\bigg)^{-12}
      -2\bigg(\frac{r_m}{r_{ij}}\bigg)^{-6})\bigg]+\epsilon,
    & r<r_m, \\
    0, & \mathrm{otherwise}.
  \end{cases}
\end{align*}

$\sigma$ is the distance between two particles where the potential
is zero for the full LJ case (i.e. if the above equation did not have
the additive constant $\epsilon$ appended), and $r_m$ is the distance between
two particles where the potential is minimum.

\section{Notation.}
When I discuss ``LJ units'', I am saying that length is measured in units
of $\sigma$. When I discuss ``WCA units'', I am saying that length is
measured in units of $r_m$.

Quantities measured in LJ units will be denoted with an asterisk.
Quantities measured in WCA units will be denoted with a tilde.
Quantities measured in real units will not have anything added to them.

\section{Fundamental units.}
So, for example, inter-particle distance $r$ can be written as
\begin{align}
  r^* = \frac{r}{\sigma}
\end{align}
or
\begin{align}
  \tilde{r} = \frac{r}{r_m}
\end{align}
for LJ units and WCA units, respectively.


Regardless of length scale, energy is measured in units of
$\beta^{-1}$, so
\begin{align}
  \epsilon^* = \beta\epsilon,
\end{align}
or
\begin{align}
  \tilde{\epsilon} = \beta\epsilon.
\end{align}


Time is given by
\begin{align}
  t^* = t\frac{D^t}{\sigma^2}.
\end{align}
or
\begin{align}
  \tilde{t} = t\frac{D^t}{r_m^2}.
\end{align}

\subsection{Dimensionless numbers.}

The Peclet number is
\begin{align}
  \mathrm{Pe}^* = \beta f^P \sigma
\end{align}
or
\begin{align}
  \tilde{\mathrm{Pe}} = \beta f^P r_m.
\end{align}

The other dimensionless number is
\begin{align}
  \Pi^* = \frac{D^r\sigma^2}{2D^t}
\end{align}
or
\begin{align}
  \tilde{\Pi} = \frac{D^rr_m^2}{2D^t}.
\end{align}

A third dimensionless number which is less important is
\begin{align}
  \Omega^*
  = \frac{\mathrm{Pe}^*}{24\sqrt{3}\epsilon^*}
  = \frac{f^P \sigma}{24\sqrt{3}\epsilon}
\end{align}
or
\begin{align}
  \tilde{\Omega}
  = \frac{\tilde{\mathrm{Pe}}}{12\sqrt{3}\tilde{\epsilon}}
  = \frac{f^P r_m}{12\sqrt{3}\epsilon}
\end{align}

\section{Relationships between LJ and WCA units.}

We see that
\begin{align}
  r^* = \frac{r}{\sigma} = \frac{r_m}{\sigma}\frac{r}{r_m}
  = 2^{1/6} \tilde{r},
\end{align}
and
\begin{align}
  t^* = t\frac{D^t}{\sigma^2} = \frac{r_m^2}{\sigma^2}t\frac{D^t}{r_m^2}
  = 2^{1/3} \tilde{t},
\end{align}
and
\begin{align}
  \mathrm{Pe}^* = \beta f^P \sigma = \frac{\sigma}{r_m}\beta f^P r_m
  = 2^{-1/6}\tilde{\mathrm{Pe}},
\end{align}
and
\begin{align}
  \Pi^* = \frac{D^r\sigma^2}{2D^t}
  = \frac{\sigma^2}{r_m^2}\frac{D^rr_m^2}{2D^t}
  = 2^{-1/3}\tilde{\Pi},
\end{align}
and
\begin{align}
  \Omega^*  = \frac{f^P \sigma}{24\sqrt{3}\epsilon}
  =\frac{\sigma}{2r_m}\frac{f^P r_m}{12\sqrt{3}\epsilon}
  =2^{-7/6}\tilde{\Omega}.
\end{align}
\end{document}
