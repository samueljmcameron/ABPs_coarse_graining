\documentclass[twocolumn,amsmath,amssymb,aps]{revtex4-1}%{article}
%\usepackage[hmargin=1.25cm,vmargin=2.5cm]{geometry}
\usepackage[utf8]{inputenc}
\usepackage{bm}
%\usepackage{amsmath}
%\usepackage{amssymb}
\usepackage{comment}
\usepackage{graphicx}
\usepackage{dcolumn}
\usepackage[caption=false]{subfig}
\usepackage{braket}
\usepackage{subfiles}

\begin{document}
 
\title{May 27, 2020: Summary of pressure calculations.}
\author{Sam Cameron}
\affiliation{%
  University of Bristol}%
\date{\today}

\begin{abstract}
  In this report, I discuss the two different methods that pressure
  can be measured in active brownian particle (ABP) systems. The
  first method to measure the pressure follows the work of Winkler
  et al \cite{C5SM01412C}, and is a direct measurement of the
  pressure using molecular dynamics (LAMMPS) software.
  The second method utilises a non-equilibrium, steady-state
  probability distribution \cite{liverpool2018nonequilibrium}
  to measure the pressure via correlation functions.
\end{abstract}


\maketitle

\section{Equations of motion of active brownian particles and
  dimensionless units.}

The equations of motion of the ABP system in $2D$ is
\begin{subequations}
  \label{eqs:ABPsEOMs}
  \begin{align}
    \dot{\bm{r}}_i
    &= -\nabla_i U+\mathrm{Pe}\:\bm{e}_i + \sqrt{2}\bm{\xi}_{i}^r,
    \label{eq:pos_EOM}\\
    \dot{\theta}_i &= \sqrt{4\Pi}\xi_i^{\theta}\label{eq:theta_EOM},
  \end{align}
\end{subequations}
where $\Pi=\sigma^2D^r/(2D^t)$ and $\mathrm{Pe} = f^P\beta\sigma$ are the
(dimensionless) rotational diffusion constant and Peclet number,
respectively ($f^P$ is the self-propulsion force of a single particle).
We measure length in units of a microscopic length scale $\sigma$
(specified by the potential $ U$), time in units of $\sigma^2/D^t$,
and energy in units of $\beta^{-1}= k_B T$, where $T$ is the temperature
of the passive solvent.
$\bm{e}_i = \cos\theta_i\bm{e}_x + \sin\theta_i\bm{e}_y$ is the direction
of self-propulsion for particle $i$.

One further assumption we make is that the interaction potential $U$
is two-bodied, and so can be written as
\begin{align}
  U(\{\bm{r}_i\})
  = \sum_{i=1}^N\sum_{j=1}^{i-1}\mathcal{V}(r_{ij}).
\end{align}

\section{Direct measurement of pressure via simulation.}

From eqns \ref{eqs:ABPsEOMs}, a straight-forward generalisation of the
derivation in ref. \cite{C5SM01412C} to $d=2$ gives the ABP pressure
for a system of $N$ particles in periodic boundary conditions with
density $N/A=\rho$
\begin{align}\label{eq:fullP_winkler}
  P_L=
  P_{\mathrm{id},p} + P_{\mathrm{id},s}
  P_{\mathrm{int},p} + P_{\mathrm{int},s}
\end{align}
I have broken up the pressure into four different terms for later
convenience. The first term on the right of eqn \ref{eq:fullP_winkler} is just
the standard ``ideal gas'' term,
\begin{align}\label{eq:P_id_p}
  P_{\mathrm{id},p}
  = \rho,
\end{align}
which is present in all gases, whether they are passive or active,
interacting or not.
The second term on the right of eqn \ref{eq:fullP_winkler} is the
``ideal swim pressure''
\cite{C5SM01412C,PhysRevLett.113.028103}
\begin{align}\label{eq:P_id_s}
  P_{\mathrm{id},s}
  = \rho\frac{\mathrm{Pe}^2}{4\Pi},
\end{align}
which is present only in active gases, but is there regardless of
interactions.
The third term on the right of eqn \ref{eq:fullP_winkler} is the
``interacting passive'' term,
\begin{align}\label{eq:P_int_p}
  P_{\mathrm{int},p}
  =\frac{\rho}{4N}\sum_{i=1}^N\sum_{j=1;j\neq i}^N\sum_{\bm{n}}
  \Braket{\bm{F}_{ij}^{\bm{n}}\cdot(\bm{r}_i-\bm{r}_j-\bm{R}_{\bm{n}})},
\end{align}
which is present in both passive and active gases, for pairwise
interactions $\bm{F}_{ij}=-\nabla{i}\mathcal{V}(r_{ij})$. The
bold $\bm{n}$ term corresponds to summing interactions over
all groups of particles which are partially within the the
unit cell (known as the \textit{minimum image-convention}
when only two-body interactions are present
\cite{doi:10.1063/1.3245303}).
The final term in eqn \ref{eq:fullP_winkler} is the
``interacting swim pressure'',
\begin{align}\label{eq:P_int_s}
  P_{\mathrm{int},s}
  = \frac{\rho\mathrm{Pe}}{8\Pi N}
  \sum_{i=1}^N\sum_{j=1;j\neq i}^N\sum_{\bm{n}}
  \Braket{\bm{F}_{ij}^{\bm{n}}\cdot(\bm{e}_i-\bm{e}_j)}
\end{align}
which is present only for active, interacting gases. One further
thing to note is that eqn \ref{eq:P_int_s} is only valid in
the steady-state (see eqn 31 in ref. \cite{C5SM01412C}).

Eqns \ref{eq:P_int_p} and \ref{eq:P_int_s} are measured in
LAMMPS by assuming that the system is ergodic and in steady-state.
We wait for $t > 10$ to start measuring pressure, and take
measurements of $P_{\mathrm{int},p} + P_{\mathrm{int},s}$
in intervals of $\delta t = 10$ to reduce correlations between
measurements. The relaxation time of the system is $\approx 1$
since simulation time is rescaled in terms of the translational
diffusion. Eqns \ref{eq:P_id_p} and \ref{eq:P_id_s} are just
added on after the calculation is completed, since they are
just constants.

\section{Indirect measurement of pressure via correlation functions.}

\subsection{Standard definition of pressure in equilibrium systems.
  \label{sub:eq}}
In order to obtain the pressure using correlation functions, it is
assumed that there exists a steady-state distribution
$\{\bm{r}_i\}\sim \frac{1}{Z}\exp(-\tilde{\beta} f(\{\bm{r}_i\}))$,
where $Z$ is the normalising factor (i.e. the partition function),
\begin{align}\label{eq:Zstandard}
  Z = \int\prod_{i=1}^N(d^2r_i)\exp[-\tilde{\beta} f(\{\bm{r}_i\})].
\end{align}
We have written temporarily $\tilde{\beta}$ (dimensional inverse
temperature) instead of setting it to $1$ as is done in the rest of
this report, just to highlight that there is a temperature dependence
of the pressure defined below.
Through the usual methods, one can write
\begin{align}\label{eq:betaP}
  \tilde{\beta} P
  &= \frac{\partial \ln Z}{\partial A}\nonumber\\
  &= \frac{N}{A}
  -\frac{\tilde{\beta}}{2A}\Braket{\sum_{j=1}\bm{r}_j\cdot
  \frac{\partial f}{\partial\bm{r}_j}},
\end{align}
where
\begin{align}
  \Braket{\cdots} = \frac{1}{Z}
  \int\prod_{k=1}^N(d^2r_k)
  \cdots\exp[-\beta f(\{\bm{r}_i\})],
\end{align}
For us to use this definition, it would require that the
temperature of the ABPs is
the same as the temperature of the solvent molecules, which is
$\tilde{\beta}^{-1}=1$ in our case.
However, it does not seem likely that the temperature of the
ABPs, if it is well defined, would be the same as the solvent
temperature. Instead, it seems more likely that the effective
temperature of the ABPs would be different than that of the solvent
(switching back now to dimensionless units),
\begin{align}
  \beta_{\mathrm{eff}}^{-1}
  = 1 + \frac{\mathrm{Pe}^2}{4\Pi},
\end{align}
or in dimensional units with  $\tilde{\beta}_{\mathrm{eff}}$ being
the dimensional ABP temperature,
$\tilde{\beta}_{\mathrm{eff}}^{-1}
=\tilde{\beta}^{-1}[1+(f^P\tilde{\beta})^2D^t/(2D^r)]$.

Therefore, we might instead consider the pressure to be
defined as
\begin{align}\label{eq:beta_effP}
  \beta_{\mathrm{eff}} P
  &= \frac{\partial \ln Z}{\partial A}\nonumber\\
  &= \frac{N}{A}
  -\frac{\beta_{\mathrm{eff}}}{2A}\Braket{\sum_{j=1}\bm{r}_j\cdot
  \frac{\partial \mathcal{V}_{\mathrm{eff}}}{\partial\bm{r}_j}},
\end{align}
where
\begin{align}\label{eq:Veff_def}
  \beta_{\mathrm{eff}} \mathcal{V}_{\mathrm{eff}}
  = \beta f
\end{align}
by definition. This further implies that eqn \label{eq:Zstandard}
can be written as
\begin{align}\label{eq:Zeff}
  Z = \int\prod_{i=1}^N(d^2r_i)\exp[-\beta_{\mathrm{eff}}
    \mathcal{V}_{\mathrm{eff}}(\{\bm{r}_i\})].
\end{align}

In this case, the ideal terms $P_{\mathrm{id},p}$ and
$P_{\mathrm{id},s}$ (eqns \ref{eq:P_id_p} and \ref{eq:P_id_s},
respectively), are automatically included since
\begin{align}
  \beta_{\mathrm{eff}}^{-1}\frac{N}{A}
  = P_{\mathrm{id},p} + P_{\mathrm{id},s}.
\end{align}

\subsection{Indirect pressure of ABPs.}


\begin{figure}[t!]
  \includegraphics[width=0.5\textwidth]{{Figures/2020_05_27_threebody_correctionno_eff_temp_P_vs_fp_rho_is_0.05_and_0.5}.pdf}
  \caption{Steady-pressure of the ABP system for $\rho=0.05$ (top) and
    $\rho=0.5$ (bottom), with the effective temperature of the ABPs
    assumed to be the same as the solvent.
    Orange circles denote the simulation pressure
    as measured in LAMMPS, $P_L$, less the ideal gas term
    $P_{\mathrm{id}}=\rho$. The green
    squares denote the pressure calculated using only the two-body
    distribution function, $S^{(2)}$ (eqn \ref{eq:S2} in the text),
    whereas the red triangles denote the pressure calculated using both
    $S^{(2)}$ and the three-body distribution function, $S^{(3)}$
    (eqn \ref{eq:S3} in the text). For the theory to be a good predictor
    of the simulation, we would expect that the red curves (and/or green
    curves)
    would overlap with the orange curves.
  }\label{fig:1}
\end{figure}

\begin{figure}[t!]
  \includegraphics[width=0.5\textwidth]{{Figures/2020_05_27_threebody_correctionwith_eff_temp_P_vs_fp_rho_is_0.05_and_0.5}.pdf}
  \caption{Steady-pressure of the ABP system for $\rho=0.05$ (top) and
    $\rho=0.5$ (bottom), with the effective temperature of the ABPs
    assumed to be larger than the solvent by a multiplicative factor of
    $1 + \frac{\mathrm{Pe}^2}{4\Pi}$.
    Orange circles denote the simulation pressure
    as measured in LAMMPS, $P_L$, less the ideal gas term
    $P_{\mathrm{id},p}=\rho$ and the ideal swim pressure term,
    $P_{\mathrm{id},s}=\rho\mathrm{Pe}^2/(4\Pi)$. The green
    squares denote the pressure calculated using only the two-body
    distribution function, $S^{(2)}$ (eqn \ref{eq:S2} in the text),
    whereas the red triangles denote the pressure calculated using both
    $S^{(2)}$ and the three-body distribution function, $S^{(3)}$
    (eqn \ref{eq:S3} in the text). For the theory to be a good predictor
    of the simulation, we would expect that the red curves (and/or green
    curves)
    would overlap with the orange curves.
  }\label{fig:2}
\end{figure}

We have shown in a separate calculation that such a steady-state
function does exist for ABPs. Up to $O(\mathrm{Pe}^4)$, the form
of the function $\beta f$ with partition function given by
eqn \ref{eq:Zstandard}, or equivalently due to eqn \ref{eq:Veff_def},
$\beta_{\mathrm{eff}}$ with partition function given by eqn
\ref{eq:Zeff},
\begin{align}
  \beta f(\{\bm{r}_i\})
  &= \beta_{\mathrm{eff}} \mathcal{V}_{\mathrm{eff}}\nonumber\\
  &=\sum_{j<i}W_2(r_{ij})
  +\sum_{k<j<i}W_3(r_{ij},r_{ik},\theta_{ij,ik}),
\end{align}
where $\bm{r}_{ij} = \bm{r}_j-\bm{r}_i$. Therefore,
\begin{align}\label{eq:P_basic}
  P
  &= \beta_{\mathrm{eff}}^{-1}\frac{N}{A}
  +\beta_{\mathrm{eff}}^{-1}S^{(2)}
  +\beta_{\mathrm{eff}}^{-1}S^{(3)}
\end{align}
where
\begin{subequations}
  \label{eqs:S2andS3}
  \begin{align}
    S^{(2)}
    &\equiv-\frac{\pi N^2}{2A^2}\int du W_2'(u)u^2 g^{(2)}(u),
    \label{eq:S2}\\
    S^{(3)}
    &\equiv-\frac{2\pi N^3}{3A^3}\int dudvd\theta u^2v
    \frac{\partial W_3(u,v,\theta)}{\partial u}
    g^{(3)}(u,v,\theta),\label{eq:S3}
  \end{align}
\end{subequations}
are the contributions to the pressure due to two-body and
three-body correlations.

If we set $\beta_{\mathrm{eff}}=1$, we are saying that the ABP particle
effective temperature is the same as the solvent, which is unlikely.
However, changing $\beta_{\mathrm{eff}}$ in eqn \ref{eq:P_basic} to
any other number just results in changing the pressure by a
multiplicative factor, namely $\beta_{\mathrm{eff}}$.

\subsubsection{Setting $\beta_{\mathrm{eff}}=1$.}


If we choose the effective temperature of the ABPs to be
$\beta_{\mathrm{eff}}=1$, then eqn \ref{eq:P_basic} becomes
\begin{align}\label{eq:beta1}
  P_{\beta_{\mathrm{eff}}=1}
  &= P_{\mathrm{id},p}+S^{(2)}+S^{(3)}.
\end{align}

In Figure \ref{fig:1}, we compare this interpretation of the
pressure to the pressure measured from LAMMPs (defined in
eqn \ref{eq:fullP_winkler}), less the ideal, passive component,
$P_L - P_{\mathrm{id},p}$. We expect that if this were the
correct interpretation, then $P_{\beta_{\mathrm{eff}}=1} = P_L$.
From eqn \ref{eq:beta1}, this would imply that
$P_L - P_{\mathrm{id},p}$ would
be approximately equal to $S^{(2)}+S^{(3)}$.

As can be seen in Figure \ref{fig:1}, we do not match the behaviour
well between the theory (green and red curves) and the simulation
(orange curves) in either the low density case of $\rho=0.05$ or
the high density case $\rho = 0.5$. In both cases, we are greatly
under-estimating the simulation pressure.

\subsubsection{Setting $\beta_{\mathrm{eff}}
  = 1 + \frac{\mathrm{Pe}^2}{4\Pi}$.}

If we instead choose the effective temperature of the ABPs to
be $\beta_{\mathrm{eff}} = 1 + \frac{\mathrm{Pe}^2}{4\Pi}$, which
is what it would be in the limit of no interactions, then
eqn \ref{eq:P_basic} becomes
\begin{align}\label{eq:betaeff}
  P_{\beta_{\mathrm{eff}}}
  &= P_{\mathrm{id},p}+P_{\mathrm{id},s}
  +\beta_{\mathrm{eff}}(S^{(2)}+S^{(3)}).
\end{align}

In Figure \ref{fig:2}, we compare this interpretation of the
pressure to the pressure measured from LAMMPS $P_L$, except this time
we exclude both the active and passive ideal components,
$P_L-P_{\mathrm{id},p}-P_{\mathrm{id},s}$. We expect that if this
were the correct interpretation, then
$P_{\beta_{\mathrm{eff}}}=P_L$. From eqn \ref{eq:betaeff},
this would imply that
$P_L-P_{\mathrm{id},p}-P_{\mathrm{id},s}$ would be approximately
equal to $\beta_{\mathrm{eff}}(S^{(2)}+S^{(3)})$.

As can be seen in Figure \ref{fig:2}, we match the behaviour well
between theory (green and red curves) and the simulation
(orange curves) only when the Peclet number is very small.

\subsubsection{Hacking the curve to get it to fit!}

The case that seems to work the best is if we instead define
the pressure to be
\begin{align}\label{eq:Prand}
  P
  &= \beta_{\mathrm{eff}}^{-1}\frac{N}{A} + S^{(2)} + S^{(3)}
  \nonumber\\
  &= P_{\mathrm{id},p} + P_{\mathrm{id},s} + S^{(2)} + S^{(3)}.
\end{align}

As can be seen from eqn \ref{eq:P_basic}, this form of the pressure
does not follow from our derivation given in section \ref{sub:eq}. But,
it seems to work slightly better when matching it to simulations.

In Figure \ref{fig:3}, we plot $P_L-P_{\mathrm{id},p}-P_{\mathrm{id},s}$,
$S^{(2)}$, and $S^{(2)}+S^{(3)}$. If we accept \ref{eq:Prand} as true
(which again, isn't clear that it should be), then we would expect
these curves to overlap.


\begin{figure}[t!]
  \includegraphics[width=0.5\textwidth]{{Figures/2020_05_27_threebody_correctionhack_P_vs_fp_rho_is_0.05_and_0.5}.pdf}
  \caption{Steady-pressure of the ABP system for $\rho=0.05$ (top) and
    $\rho=0.5$ (bottom), with the effective temperature of the ABPs
    assumed to be larger than the solvent by a multiplicative factor of
    $1 + \frac{\mathrm{Pe}^2}{4\Pi}$.
    Orange circles denote the simulation pressure
    as measured in LAMMPS, $P_L$, less the ideal gas term
    $P_{\mathrm{id},p}=\rho$ and the ideal swim pressure term,
    $P_{\mathrm{id},s}=\rho\mathrm{Pe}^2/(4\Pi)$. The green
    squares denote the pressure calculated using only the two-body
    distribution function, $S^{(2)}$ (eqn \ref{eq:S2} in the text),
    whereas the red triangles denote the pressure calculated using both
    $S^{(2)}$ and the three-body distribution function, $S^{(3)}$
    (eqn \ref{eq:S3} in the text). For the theory to be a good predictor
    of the simulation, we would expect that the red curves (and/or green
    curves)
    would overlap with the orange curves.
  }\label{fig:3}
\end{figure}

\bibliographystyle{plain}
\bibliography{summary_of_pressure_calculations}

\end{document}
