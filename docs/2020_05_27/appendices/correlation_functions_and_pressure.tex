\documentclass[../main.tex]{subfiles}

\begin{document}

In an equilibrium system, the pressure $P$ is defined as
\begin{align}
  \beta_{\mathrm{eff}} P = \frac{1}{Z}\frac{\partial Z}{\partial A}
\end{align}
where $Z$ is the (canonical) partition function. Consider a generic
partition function
\begin{align}
  Z=\int\prod_{k=1}^N(d^2r_k)\exp[-\beta_{\mathrm{eff}} U(\{\bm{r}_i\})].
\end{align}
I will first explicitly introduce the area dependence in the integrand by
writing $\bm{r}_i = A^{1/2}\bm{t}_i$, and so
\begin{align}
  Z = A^N\int\prod_{k=1}^N(d^2t_k)\exp[-\beta_{\mathrm{eff}} U(\{A^{1/2}\bm{t}_i\})].
\end{align}
Next, I take the area derivative of the partition function,
\begin{align}
  \frac{\partial Z}{\partial A}
  =&NA^{N-1}\int\prod_{k=1}^N(d^2t_k)\exp[-\beta_{\mathrm{eff}} U(\{A^{1/2}\bm{t}_i\})]
  \nonumber\\
  &+A^N\int\frac{\partial}{\partial A}\prod_{k=1}^N(d^2t_k)
  \exp[-\beta_{\mathrm{eff}} U(\{A^{1/2}\bm{t}_i\})]\nonumber\\
  =&\frac{N}{A}Z
  -A^N\int\frac{\partial}{\partial A}\prod_{k=1}^N(d^2t_k)
  \exp[-\beta_{\mathrm{eff}} U(\{A^{1/2}\bm{t}_i\})]\nonumber\\
  =&\frac{N}{A}Z
  -\frac{\beta_{\mathrm{eff}}}{2A}A^N\int\prod_{k=1}^N(d^2t_k)
  \frac{\partial U}{\partial\bm{r}_j}\cdot
  \bm{r}_j\exp[-\beta_{\mathrm{eff}} U(\{A^{1/2}\bm{t}_i\})]\nonumber\\
  =&\frac{N}{A}Z
  -\frac{\beta_{\mathrm{eff}}}{2A}\int\prod_{k=1}^N(d^2r_k)
  \sum_{j}\frac{\partial U}{\partial\bm{r}_j}\cdot
  \bm{r}_j\exp[-\beta_{\mathrm{eff}} U(\{\bm{r}_i\})].
\end{align}
This then gives the pressure as
\begin{align}\label{eq:P_with_U}
  \beta_{\mathrm{eff}} P = \frac{N}{A}
  -\frac{\beta_{\mathrm{eff}}}{2A}\bigg<\sum_{j=1}\bm{r}_j\cdot
  \frac{\partial U}{\partial\bm{r}_j}\bigg>
\end{align}
where
\begin{align}
  <\cdots> = \frac{1}{Z}
  \int\prod_{k=1}^N(d^2r_k)
  \cdots\exp[-\beta_{\mathrm{eff}} U(\{\bm{r}_i\})].
\end{align}

\subsection{Pairwise and three-body interactions.\label{app:onlytwobody}}
Since we are looking at a steady-state probability distribution, with

\begin{align}
  <\cdots> = \frac{1}{Z}
  \int\prod_{k=1}^N(d^2r_k)
  \cdots\exp[-h(\{\bm{r}_i\})],
\end{align}
eqn \ref{eq:P_with_U} above should instead be written with replacing
$\beta_{\mathrm{eff}} U$ with $h$, e.g.
\begin{align}\label{eq:P_with_U}
  \beta_{\mathrm{eff}} P = \frac{N}{A}
  -\frac{1}{2A}\bigg<\sum_{j=1}\bm{r}_j\cdot
  \frac{\partial h}{\partial\bm{r}_j}\bigg>.
\end{align}
Next, we assume that
\begin{align}
  h = \sum_{i}\sum_{j<i}W_2+\sum_{i}\sum_{j<i}\sum_{k<j}W_3.
\end{align}
The partial derivative of this interaction potential with respect
to position is then
\begin{widetext}
  \begin{align}
    \frac{\partial h}{\partial \bm{r}_l}
    =& \sum_{i}\sum_{j<i}
    \frac{\partial W_2(\bm{r}_i,\bm{r}_j)}{\partial \bm{r}_l}
    +\sum_{i}\sum_{j<i}\sum_{k<j}
    \frac{\partial W_3(\bm{r}_i,\bm{r}_j,\bm{r}_k)}{\partial \bm{r}_l}
    \nonumber\\
    =& \frac{1}{2}\bigg(\sum_{j\neq l}
    \frac{\partial W_2(\bm{r}_l,\bm{r}_j)}{\partial \bm{r}_l}
    +\sum_{i\neq l}
    \frac{\partial W_2(\bm{r}_i,\bm{r}_l)}{\partial \bm{r}_l}\bigg)
    +\frac{1}{6}\bigg(\sum_{j\neq l}\sum_{k\neq l,k\neq j}
    \frac{\partial W_3(\bm{r}_l,\bm{r}_j,\bm{r}_k)}{\partial \bm{r}_l}
    \nonumber\\
    &+\sum_{i\neq l}\sum_{k\neq l, k\neq i}
    \frac{\partial W_3(\bm{r}_i,\bm{r}_l,\bm{r}_k)}{\partial \bm{r}_l}
    +\sum_{i\neq l}\sum_{j\neq l,j\neq i}
    \frac{\partial W_3(\bm{r}_i,\bm{r}_j,\bm{r}_l)}{\partial \bm{r}_l}
    \bigg).
  \end{align}
  Therefore,
  \begin{align}
    \sum_l\bm{r}_l\cdot\frac{\partial h}{\partial \bm{r}_l}
    =&\frac{1}{2}\bigg(\sum_l\sum_{j\neq l}\bm{r}_l\cdot
    \frac{\partial W_2(\bm{r}_l,\bm{r}_j)}{\partial \bm{r}_l}
    +\sum_l\sum_{i\neq l}\bm{r}_l\cdot
    \frac{\partial W_2(\bm{r}_i,\bm{r}_l)}{\partial \bm{r}_l}\bigg)
    +\frac{1}{6}\bigg(\sum_l\sum_{j\neq l}\sum_{k\neq l,k\neq j}
    \bm{r}_l\cdot
    \frac{\partial W_3(\bm{r}_l,\bm{r}_j,\bm{r}_k)}{\partial \bm{r}_l}
    \nonumber\\
    &+\sum_l\sum_{i\neq l}\sum_{k\neq l, k\neq i}\bm{r}_l\cdot
    \frac{\partial W_3(\bm{r}_i,\bm{r}_l,\bm{r}_k)}{\partial \bm{r}_l}
    +\sum_l\sum_{i\neq l}\sum_{j\neq l,j\neq i}\bm{r}_l\cdot
    \frac{\partial W_3(\bm{r}_i,\bm{r}_j,\bm{r}_l)}{\partial \bm{r}_l}
    \bigg).
  \end{align}
\end{widetext}
Now, when integrating over the above expression, there are $N$ ways to
integrate over the first particle, $N-1$ ways to integrate over the
second particle, and $N-2$ ways to integrate over the third particle,
so that the pressure becomes
\begin{widetext}
  \begin{align}
    \beta_{\mathrm{eff}} P
    =& \frac{N}{A}
    -\frac{1}{Z}\frac{N(N-1)}{4A}\int d^2r_1d^2r_2\bigg(\bm{r}_1\cdot
    \frac{\partial W_2(\bm{r}_1,\bm{r}_2)}{\partial \bm{r}_1}
    +\bm{r}_2\cdot
    \frac{\partial W_2(\bm{r}_1,\bm{r}_2)}{\partial \bm{r}_2}\bigg)
    \int\prod_{k=3}^N(d^2r_k)
    \exp[-h(\{\bm{r}_i\})]\nonumber\\
    &-\frac{1}{Z}\frac{N(N-1)(N-2)}{12A}\int d^2r_1d^2r_2d^2r_3\bigg(
    \bm{r}_1\cdot
    \frac{\partial W_3(\bm{r}_1,\bm{r}_2,\bm{r}_3)}{\partial \bm{r}_1}
    +\bm{r}_2\cdot
    \frac{\partial W_3(\bm{r}_1,\bm{r}_2,\bm{r}_3)}{\partial \bm{r}_2}
    \nonumber\\
    &+\bm{r}_3\cdot
    \frac{\partial W_3(\bm{r}_1,\bm{r}_2,\bm{r}_3)}{\partial \bm{r}_3}
    \bigg)
    \int\prod_{k=4}^N(d^2r_k)
    \exp[-h(\{\bm{r}_i\})].
  \end{align}

  Defining the correlation function of order $n$ as
  \begin{align}
    g^{(n)}(\bm{r}_1,\cdots,\bm{r}_n)=\frac{A^nN!}{N^n(N-n)!}
    \frac{1}{Z}\int\prod_{k=n+1}^N(d^2r_k)
    \exp[-h(\{\bm{r}_i\})],
  \end{align}
  this simplifies to
  \begin{align}\label{appeq:p_2_and_3}
    \beta_{\mathrm{eff}} P
    =& \frac{N}{A}
    -\frac{N^2}{4A^3}\int d^2r_1d^2r_2\bigg(\bm{r}_1\cdot
    \frac{\partial W_2(\bm{r}_1,\bm{r}_2)}{\partial \bm{r}_1}
    +\bm{r}_2\cdot
    \frac{\partial W_2(\bm{r}_1,\bm{r}_2)}{\partial \bm{r}_2}\bigg)
    g^{(2)}(\bm{r}_1,\bm{r}_2)\nonumber\\
    &-\frac{N^3}{12A^4}\int d^2r_1d^2r_2d^2r_3\bigg(
    \bm{r}_1\cdot
    \frac{\partial W_3(\bm{r}_1,\bm{r}_2,\bm{r}_3)}{\partial \bm{r}_1}
    +\bm{r}_2\cdot
    \frac{\partial W_3(\bm{r}_1,\bm{r}_2,\bm{r}_3)}{\partial \bm{r}_2}
    +\bm{r}_3\cdot
    \frac{\partial W_3(\bm{r}_1,\bm{r}_2,\bm{r}_3)}{\partial \bm{r}_3}
    \bigg)g^{(3)}(\bm{r}_1,\bm{r}_2,\bm{r}_3).
  \end{align}
\end{widetext}

If we assume that $W_2(\bm{r}_1,\bm{r}_2)=W_2(r_{12})$ is translationally
invariant and $g^{(2)}(\bm{r}_1,\bm{r}_2)=g^{(2)}(r_{12})$, then the term
with the second order correlation function can be simplified further.
Using the change of coordinates
\begin{align}
  \bm{r}_1&=\frac{\bm{u}+\bm{v}}{\sqrt{2}},\\
  \bm{r}_2&=\frac{\bm{v}-\bm{u}}{\sqrt{2}}
\end{align}
or
\begin{align}
  \bm{u}&=\frac{\bm{r}_2-\bm{r}_1}{\sqrt{2}},\\
  \bm{v}&=\frac{\bm{r}_1+\bm{r}_2}{\sqrt{2}}.
\end{align}
then
\begin{align}
  &\int d^2r_1d^2r_2\bigg(\bm{r}_1\cdot
  \frac{\partial W_2(r_{12})}{\partial \bm{r}_1}
  +\bm{r}_2\cdot
  \frac{\partial W_2(r_{12})}{\partial \bm{r}_2}\bigg)
  g^{(2)}(r_{12})\nonumber\\
  &=\int d^2r_1d^2r_2
  W_2'(r_{12})(\bm{r}_2-\bm{r}_1)\cdot
  \frac{(\bm{r}_2-\bm{r}_1)}{r_{12}}
  g^{(2)}(r_{12})\nonumber\\
  &=\int d^2u W_2'(u)u g^{(2)}(u)\nonumber\\
  &=A\int d^2u W_2'(u)u g^{(2)}(u)\nonumber\\
  &=2\pi A\int_0^{\infty} du W_2'(u)u^2 g^{(2)}(u).
\end{align}

Next, if we assume that $W_3(r_{12},r_{13},\cos(\theta_{12}-\theta_{13}))$
(consistent with eqn \ref{eq:3body}), where
$\bm{r}_{12}\cdot\bm{r}_{13}=r_{12}r_{13}\cos(\theta_{12}-\theta_{13})$,
the third order correlation function term in eqn \ref{appeq:p_2_and_3}
can also be simplified further. The partial derivatives in the integrand
of the third order term are
\begin{widetext}
  \begin{align}
    \frac{\partial W_3}{\partial \bm{r}_1}
    &= -\frac{\bm{r}_{12}}{r_{12}}\frac{\partial W_3}{\partial r_{12}}
    - \frac{\bm{r}_{13}}{r_{13}}\frac{\partial W_3}{\partial r_{13}}
    +\bigg[\frac{\bm{r}_{12}}{r_{12}^3r_{13}}\bm{r}_{12}\cdot\bm{r}_{13}
      +\frac{\bm{r}_{13}}{r_{12}r_{13}^3}\bm{r}_{12}\cdot\bm{r}_{13}
      -\frac{(\bm{r}_{12}+\bm{r}_{13})}{r_{12}r_{13}}\bigg]
    \frac{\partial W_3}{\partial\cos(\theta_{12}-\theta_{13})},\\
    \frac{\partial W_3}{\partial \bm{r}_2}
    &= \frac{\bm{r}_{12}}{r_{12}}\frac{\partial W_3}{\partial r_{12}}
    +\frac{1}{r_{12}r_{13}}\bigg[\bm{r}_{13}-\frac{\bm{r}_{12}}{r_{12}^2}
      \bm{r}_{12}\cdot\bm{r}_{13}\bigg]
    \frac{\partial W_3}{\partial\cos(\theta_{12}-\theta_{13})},\\
    \frac{\partial W_3}{\partial \bm{r}_3}
    &= \frac{\bm{r}_{13}}{r_{13}}\frac{\partial W_3}{\partial r_{13}}
    +\frac{1}{r_{12}r_{13}}\bigg[\bm{r}_{12}-\frac{\bm{r}_{13}}{r_{13}^2}
      \bm{r}_{12}\cdot\bm{r}_{13}\bigg]
    \frac{\partial W_3}{\partial\cos(\theta_{12}-\theta_{13})}.
  \end{align}
\end{widetext}
The integrand itself is then
\begin{align}
  &\bm{r}_1\cdot\frac{\partial W_3}{\partial \bm{r}_1}+
  \bm{r}_2\cdot\frac{\partial W_3}{\partial \bm{r}_2}+
  \bm{r}_3\cdot\frac{\partial W_3}{\partial \bm{r}_3}\nonumber\\
  =&\frac{\bm{r}_{12}\cdot\bm{r}_{12}}{r_{12}}
  \frac{\partial W_3}{\partial r_{12}}+
  \frac{\bm{r}_{13}\cdot\bm{r}_{13}}{r_{13}}
  \frac{\partial W_3}{\partial r_{13}}\nonumber\\
  &+\frac{\bm{r}_{12}\cdot\bm{r}_{13}}{r_{12}r_{13}}\bigg[
    2-\frac{\bm{r}_{12}\cdot\bm{r}_{12}}{r_{12}^2}
    -\frac{\bm{r}_{13}\cdot\bm{r}_{13}}{r_{13}^2}\bigg]
  \frac{\partial W_3}{\partial \cos(\theta_{12}-\theta_{13})}\nonumber\\
  =&r_{12}\frac{\partial W_3}{\partial r_{12}}+
  r_{13}\frac{\partial W_3}{\partial r_{13}}.
\end{align}


Next, we assume that the three body correlation function is rotationally
invariant, so
\begin{align}
  g^{(3)}(\bm{r}_1,\bm{r}_2,\bm{r}_3)
  = g^{(3)}(\bm{r}_{12},\bm{r}_{13},\bm{r}_{23}).
\end{align}

Introducing the change of variables
\begin{equation}
  \bm{x}=
  \begin{bmatrix}
    -1 &  0 &  1 &  0 &  0 &  0 \\
    0  & -1 &  0 &  1 &  0 &  0 \\
    -1 &  0 &  0 &  0 &  1 &  0 \\
    0  & -1 &  0 &  0 &  0 &  1 \\
    0  &  0 &  0 &  0 &  1 &  0 \\
    0  &  0 &  0 &  0 &  0 &  1
  \end{bmatrix}
  \begin{bmatrix}
    x_1 \\
    y_1 \\
    x_2 \\
    y_2 \\
    x_3 \\
    y_3
  \end{bmatrix},
\end{equation}
and defining $\bm{u} = \bm{r}_{12}$, $\bm{v} = \bm{r}_{13}$,
the three body correlation term in eqn \ref{appeq:p_2_and_3}
becomes (since the determinant of the above matrix is $1$)
\begin{widetext}
  \begin{align}\label{eq:trans_inv_g3}
    &\frac{N^3}{12A^4}\int d^2r_1d^2r_2d^2r_3\bigg(
    \bm{r}_1\cdot
    \frac{\partial W_3(\bm{r}_1,\bm{r}_2,\bm{r}_3)}{\partial \bm{r}_1}
    +\bm{r}_2\cdot
    \frac{\partial W_3(\bm{r}_1,\bm{r}_2,\bm{r}_3)}{\partial \bm{r}_2}
    +\bm{r}_3\cdot
    \frac{\partial W_3(\bm{r}_1,\bm{r}_2,\bm{r}_3)}{\partial \bm{r}_3}
    \bigg)g^{(3)}(\bm{r}_1,\bm{r}_2,\bm{r}_3)\nonumber\\
    &=\frac{N^3}{12A^4}\int d^2r_1d^2r_2d^2r_3\bigg(
    r_{12}\frac{\partial W_3}{\partial r_{12}}+
    r_{13}\frac{\partial W_3}{\partial r_{13}}
    \bigg)g^{(3)}(r_{12},r_{13},|\bm{r}_{12}-\bm{r}_{13}|)\nonumber\\
    &=\frac{N^3}{12A^3}\int d^2ud^2v\bigg(
    u\frac{\partial W_3}{\partial u}+
    v\frac{\partial W_3}{\partial v}
    \bigg)g^{(3)}(u,v,|\bm{u}-\bm{v}|)\nonumber\\
    &=\frac{N^3}{12A^3}\int udud\theta_{12}vdvd\theta_{13}\bigg(
    u\frac{\partial W_3}{\partial u}+
    v\frac{\partial W_3}{\partial v}
    \bigg)g^{(3)}(u,v,\cos(\theta_{12}-\theta_{13})).
  \end{align}
\end{widetext}
In the last line, we have re-written $g^{(3)}$ in terms of the 
difference in angle, $\theta_{12}-\theta_{13}$, between $\bm{u}$ and
$\bm{v}$.

Eqn \ref{eq:trans_inv_g3} can be simplified further by making the change
of variables
$\theta=\theta_{12}-\theta_{13}$, $\phi=\theta_{12}+\theta_{13}$,
so that for some function $f(\theta_{12}-\theta_{13})$,
\begin{align}
  \int_0^{2\pi}d\theta_{12}\int_0^{2\pi}d\theta_{13}
  f(\theta_{12}-\theta_{13})
  =\frac{1}{2}\int_{\mathcal{R}}d\theta d\phi f(\theta)
\end{align}
where the region $\mathcal{R}$ is a square in the $\phi>0$ plane,
with diagonals along the $\theta=0$ axis and the axis $\phi=2\pi$. To
evaluate things consistently along the integration region, we integrate
over four regions, and find that it simplifies greatly,
\begin{widetext}
  \begin{align}
    \frac{1}{2}\int\int_{\mathcal{R}}d\theta d\phi f(\theta)
    =&\frac{1}{2}\int_{-2\pi}^0d\theta
    \int_{-\theta}^{2\pi}d\phi f(\theta)
    +\frac{1}{2}\int_0^{2\pi}d\theta
    \int_{\theta}^{2\pi}d\phi f(\theta)\nonumber\\
    &+\frac{1}{2}\int_{-2\pi}^0d\theta
    \int_{2\pi}^{4\pi+\theta}d\phi f(\theta)
    +\frac{1}{2}\int_0^{2\pi}d\theta
    \int_{2\pi}^{4\pi-\theta}d\phi f(\theta)\nonumber\\
    =&\frac{1}{2}\int_{-2\pi}^0d\theta
    (2\pi+\theta) f(\theta)
    +\frac{1}{2}\int_0^{2\pi}d\theta
    (2\pi-\theta) f(\theta)\nonumber\\
    &+\frac{1}{2}\int_{-2\pi}^0d\theta
    (2\pi+\theta) f(\theta)
    +\frac{1}{2}\int_0^{2\pi}d\theta
    (2\pi-\theta) f(\theta)\nonumber\\
    =&\int_{-2\pi}^0d\theta
    (2\pi+\theta) f(\theta)
    +\int_0^{2\pi}d\theta
    (2\pi-\theta) f(\theta)\nonumber\\
    =&\int_0^{2\pi}d\theta
    (2\pi-\theta) (f(\theta)+f(-\theta)).
  \end{align}
\end{widetext}
We know explicitly that $W_3\propto\cos\theta$, and from physical
considerations (since $g^{(3)}$ is a function only of the distances
between three particles) that
$g^{(3)}(\theta)=g^{(3)}(-\theta)=g^{(3)}(2\pi-\theta)$, so the
$f(\theta)$ in the above equation should also satisfy these conditions.
Therefore the above integral simplifies further (using the substitution
$\theta'=2\pi-\theta$ where appropriate) as
\begin{align}
  &\int_0^{2\pi}d\theta
  (2\pi-\theta) (f(\theta)+f(-\theta))\nonumber\\
  =&2\int_0^{2\pi}d\theta(2\pi-\theta)f(\theta)\nonumber\\
  =&2\int_0^{\pi}d\theta(2\pi-\theta)f(\theta)+
  2\int_{\pi}^{2\pi}d\theta(2\pi-\theta)f(\theta)\nonumber\\
  =&2\int_0^{\pi}d\theta(2\pi-\theta)f(\theta)+
  2\int_{\pi}^{0}-d\theta'\theta'f(2\pi-\theta')\nonumber\\
  =&2\int_0^{\pi}d\theta(2\pi-\theta)f(\theta)+
  2\int_0^{\pi}d\theta'\theta'f(\theta')\nonumber\\
  =&4\pi\int_0^{\pi}d\theta f(\theta)
\end{align}
This maybe should have been expected, as it corresponds to essentially
integrating out one of the two original coordinates (e.g. $d\theta_{12}$)
to get a factor of $2\pi$, and then knowing that the integral from $0$ to
$\pi$ should be the same as the integral from $\pi$ to $2\pi$ to get another
factor of $2$.
  
Inserting this result into eqn \ref{eq:trans_inv_g3} with its integrand
\footnote{I have been a bit sloppy in changing the arguments of the function
  freely without changing the function name, but it should be clear from
  context what is meant.}
instead of $f(\theta)$,
\begin{widetext}
  \begin{align}
    &\frac{N^3}{12A^3}\int udud\theta_{12}vdvd\theta_{13}\bigg(
    u\frac{\partial W_3}{\partial u}+
    v\frac{\partial W_3}{\partial v}
    \bigg)g^{(3)}(u,v,\cos(\theta_{12}-\theta_{13}))\nonumber\\
    =&\frac{\pi N^3}{3A^3}\int_{0}^{\infty} udu\int_{0}^{\infty}vdv
    \int_{0}^{\pi}d\theta\bigg(
    u\frac{\partial W_3(u,v,\theta)}{\partial u}+
    v\frac{\partial W_3(u,v,\theta)}{\partial v}
    \bigg)g^{(3)}(u,v,\theta).
  \end{align}
\end{widetext}

Therefore, in order to determine the three-body contributions to the pressure,
I need to measure $g^{(3)}(u,v,\theta)$, where $u=r_{12}$, $v=r_{13}$, and
$\theta=\arccos(\bm{r}_{12}\cdot\bm{r}_{13}/(r_{12}r_{13}))$.

Just to touch briefly on the numerical measurement of $g^{(3)}$, we know by
definition that
\begin{align}
  &\frac{N^3}{A^3}\hat{g}^{(3)}(\bm{r}_1,\bm{r}_2,\bm{r}_3)\nonumber\\
  =&\sum_{\alpha=1}^N\sum_{\substack{\beta=1 \\\beta\neq \alpha}}^N
  \sum_{\substack{\gamma=1 \\\gamma\neq \alpha \\
      \gamma\neq\beta}}^N\delta(\bm{r}_1-\bm{x}_{\alpha})
  \delta(\bm{r}_2-\bm{x}_{\beta})\delta(\bm{r}_3-\bm{x}_{\gamma}),
\end{align}
where $g^{(3)} = <\hat{g}^{(3)}>$.
If the system is translationally invariant, then
$g^{(3)}(\bm{r}_1,\bm{r}_2,\bm{r}_3)=g^{(3)}(\bm{u},\bm{v},\bm{v}-\bm{u})$,
with $\bm{u}=\bm{r}_2-\bm{r}_1$, $\bm{v}=\bm{r}_3-\bm{r}_1$, and so
\begin{align}
  &\frac{N^3}{A^3}\hat{g}^{(3)}(\bm{u},\bm{v},\bm{v}-\bm{u})\nonumber\\
  =&\sum_{\alpha,\beta,\gamma}^N\prime
  \delta(\bm{r}_1-\bm{x}_{\alpha})
  \delta(\bm{u}+\bm{r}_1-\bm{x}_{\beta})
  \delta(\bm{v}+\bm{r}_1-\bm{x}_{\gamma})
\end{align}
where the prime on the sum indicates that $\alpha$, $\beta$, $\gamma$ must
all take different values.
We can then integrate out $\bm{r}_1$ since the lefthand side is independent
of it, so
\begin{align}
  &A\frac{N^3}{A^3}\hat{g}^{(3)}(\bm{u},\bm{v},\bm{v}-\bm{u})\nonumber\\
  =&\int d^2r_1\sum_{\alpha,\beta,\gamma}^N\prime
  \delta(\bm{r}_1-\bm{x}_{\alpha})
  \delta(\bm{u}+\bm{r}_1-\bm{x}_{\beta})
  \delta(\bm{v}+\bm{r}_1-\bm{x}_{\gamma})\nonumber\\
  =&\sum_{\alpha,\beta,\gamma}^N\prime
  \delta(\bm{u}-(\bm{x}_{\beta}-\bm{x}_{\alpha}))
  \delta(\bm{v}-(\bm{x}_{\gamma}-\bm{x}_{\alpha}))\nonumber\\
  =&\sum_{\alpha,\beta,\gamma}^N\prime
  \delta(\bm{u}-\bm{x}_{\alpha\beta})
  \delta(\bm{v}-\bm{x}_{\alpha\gamma}).
\end{align}
Assuming that the system is also rotationally invariant, we can
define $\theta=\theta_v-\theta_u$, and integrate out $\theta_u$ since
$g^{(3)}$ is independent of it, i.e.
\begin{align}
  &2\pi\frac{N^3}{A^2}\hat{g}^{(3)}(u,v,\cos\theta)\nonumber\\
  =&\int d\theta_u\sum_{\alpha,\beta,\gamma}^N\prime
  \delta(u-x_{\alpha\beta})\delta(\theta_u-\theta_{\alpha\beta})
  \delta(v-x_{\alpha\gamma})\nonumber\\
  &\times\delta(\theta+\theta_u-\theta_{\alpha\gamma})\nonumber\\
  =&\sum_{\alpha,\beta,\gamma}^N\prime
  \delta(u-x_{\alpha\beta})
  \delta(v-x_{\alpha\gamma})
  \delta(\theta-(\theta_{\alpha\gamma}-\theta_{\alpha\beta})).
\end{align}
Finally, the numerical measurement of $\hat{g}^{(3)}$ requires us to
discretize the dirac delta functions. If we define $\delta_{i,j}$ as
the Kronecker delta function, then
\begin{align}
  &2\pi\frac{N^3}{A^2}\hat{g}^{(3)}(u,v,\cos\theta)\nonumber\\
  =&\sum_{\alpha,\beta,\gamma}^N\prime
  \frac{\delta_{u,x_{\alpha\beta}}}{\frac{(u+\Delta u)^2-u^2}{2}}
  \frac{\delta_{v,x_{\alpha\gamma}}}{\frac{(v+\Delta v)^2-v^2}{2}}
  \frac{\delta_{\theta,\theta_{\gamma\beta}}}{\Delta \theta},
\end{align}
where $\Delta u$, $\Delta v$, and $\Delta \theta$ are the sizes of the
bins in the numerical calculation.
\end{document}
